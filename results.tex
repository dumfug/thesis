\chapter{Results}
\label{cha:results}


This chapter contains the results for the analysis of networks that were generated using the proposed peer influence model.
All results are obtained from synthetic networks that were generated over \( T = 75,000 \) iterations.
The size of the networks was fixed to \(n = 5,000 \) nodes and, since the model heavily depends on events that happen at random, the reported properties of the time-varying networks are obtained by averaging the results of 40 independent runs.
The model parameter that are responsible for the formation of the community structures are set to \( p_{\Delta} = 0.90 \) for the triadic closure probability, \( \delta = 1 \) for the link reinforcement constant, and \( p_{d} = \num{5e-05} \) for the node deletion probability for every experiment.
Furthermore, the critical peer influence threshold was fixed to \( \theta = 0.10 \).
This reflects the idea that only a relatively small number of active neighbors is sufficient to increase the activity of a node significantly.

The topological properties of the integrated network, which are discussed in \cref{sec:integrated-network-properties}, are measured only for nodes that are part of the temporal network.
This means that nodes that were removed earlier due to the node deletion process do not influence the properties of the integrated network any more.
\Cref{sec:network-activity} contains an overview of the overall network activity with respect to different levels of peer influence.
The effect of the peer influence mechanism on the inter-event time distribution in the network is examined in \cref{sec:inter-event-time-dists}.
All this experiments are performed for different values for the maximum peer influence probability \( q \).
However, the last section (\cref{sec:softmax-rescaling}) keeps the peer influence level constant and discusses how different values for \( \beta \), the inverse temperature for the softmax weight re-scaling, change the peer influence effects in the network.


%% ========================================================================
%% ========================================================================


\section{Integrated Network Properties}
\label{sec:integrated-network-properties}


%% ========================================================================
%% ========================================================================


\section{Network Activity}
\label{sec:network-activity}


%% ========================================================================
%% ========================================================================


\section{Inter-event Time Distributions}
\label{sec:inter-event-time-dists}
% definition burstiness and parameter B is invariant wrt to activity homogeneity in \cite{Goh2008}
% nice explanation for B in \cite{Masuda2016}


%% ========================================================================
%% ========================================================================


\section{Softmax Weight Re-scaling}
\label{sec:softmax-rescaling}
