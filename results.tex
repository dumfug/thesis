\chapter{Results}
\label{cha:results}


This chapter discusses the basic findings of the analysis of the proposed peer influence model.
All results were obtained from synthetic networks, with a fixed size of \( n = 5,000 \) nodes, which were generated over \( T = 75,000 \) iterations.
The here presented properties of the time-varying networks were obtained by averaging the results of 40 independent runs, which is necessary due to the stochastic nature of the model.
The model parameter responsible for the formation of the community structures are set to \( p_{\Delta} = 0.90 \) for the triadic closure probability, \( \delta = 1 \) for the link reinforcement constant, and \( p_{d} = \num{5e-05} \) for the node deletion probability for every experiment.
Furthermore, the critical peer influence threshold was fixed to \( \theta = 0.10 \).
This reflects the idea that only a relatively small number of active neighbors is sufficient to increase the activity of a node in a significant way.

The time-dependent topological properties of the integrated network, which are discussed in \cref{sec:integrated-network-properties}, are measured only for nodes that are part of the temporal network.
This means that nodes which were removed earlier due to the node deletion process do not influence the properties of the integrated network any more.
\Cref{sec:network-activity} contains an overview of the overall network activity with respect to different levels of peer influence.
The effect of the peer influence mechanism on the inter-event time distribution in the network is examined in \cref{sec:inter-event-time-dists}.
All this experiments are performed for different values for the maximum peer influence probability \( q \).
However, the analysis performed in the last section (\cref{sec:softmax-rescaling}) uses a fixed value for the peer influence level and discusses how different values for \( \beta \), the inverse temperature for the softmax weight rescaling, changes the peer influence effects in the network.
The synthetic networks which are used in the first three sections adopt the average tie strength as temperature for the softmax weight rescaling.
Therefore, \( \beta \) is set to the to the inverse of the average of the weights in the integrated network in each iteration after the first one (i.e., \( \beta = 1 \) in the initial round to avoid division by zero, since no ties have been formed yet).


%% ========================================================================
%% ========================================================================


\section{Time-dependent Integrated Network Properties}
\label{sec:integrated-network-properties}

Not only the integrated network of all 75,000 previous instantaneous networks and its properties are of great interest, but also how they evolve during the simulation.
This allows to get a deeper understanding on how the model shapes the community structures in the network and the role of peer influence on it.
To make this possible is the integrated network build in an iterative fashion.
A snapshot of it is taken after the newly formed ties are added and the weights of already established links are updated in every time step.
Features like the average local clustering coefficient or the average weight of the ties are calculated for all of the 75,000 integrated network snapshots.
This allows to examine on how the topology and other measures change over time.

The first, and most interesting, measure that can be investigated in this way is the average local clustering coefficient \( C(t) \).
\Cref{fig:avg-local-cc-full} depicts the development of it over the course of the simulation for different levels of peer influence.
The graph of this function has a very distinctive pattern, which was already explained in the original work by \citet{Laurent2015}.
The average clustering coefficient is very small in the first few hundred iterations, due to the sparsity of the integrated network.
Almost all nodes are disconnected and the number of triangles which have formed is relatively small compared to the size of the network.
However, \cref{fig:avg-local-cc-full} also shows that the clustering coefficient grows very fast until it reaches its maximum value.
This rapid increase is caused by the cyclic closure mechanism of the model.
Nodes that become active in this early stage first have to introduce some ties with nodes which are selected using the focal closure mechanism.
This does not increase the average local clustering coefficient in a significant way, however, it establishes the egocentric networks.
After the first triangles are closed, the first strong ties start to develop.
These emerging strong ties amplify the biased local search of the triadic closure mechanism and result, on one hand, in in more triangles, and on the other hand, in the reinforcement of already established triangles and the associated strong ties.
This leads to a high local clustering in the established communities.
However, weak ties are eventually also introduced to the network, due to the focal closure mechanism.
They are rarely involved in the formation of new triangles, due to the bias towards strong ties, which contributes to the decrease of the average local clustering coefficient until the network reaches its stationary state.


\myfig{avg-local-cc-full}
      {width=0.75\textwidth}
      {The average local clustering coefficient \( C \) as a function of time for different maximum peer influence probabilities \( q = 0, \, 0.01, \, 0.05, \, 0.1, \, 0.15\). }
      {Average local clustering coefficient as function of time}
      {fig:avg-local-cc-full}


The peer influence mechanism seems to influence the development of the local clustering in the beginning of the simulation significantly.
\Cref{fig:avg-local-cc-start} depicts the time-depended average local clustering coefficient in the initial phase (i.e., for the first 10,000 iterations) for a range of possible values for \( q \).
It shows that the peak of \( C \) is reached faster for networks in which peer influence effects are present.
For instance, the network in which nodes are not able to influence their neighbors reaches it peak value for \( C \) after about 5,000 iterations, while the network with a maximum peer influence probability of \( q = 0.15 \) is more than 2,000 iterations faster.
However, the effect only occurs for networks with \( q > 0.05 \) and the actual maximum value of the local clustering coefficient does increase only slightly for higher levels of peer influence (c.f. \cref{tbl:max-clustering} for the precise figures).
Furthermore, it seems to have an positive effect on the development of the topological structures in the network, by accelerating the process in the beginning.
The percentage of network activity that reinforces ties in each iteration \( r(t) \) underlines this property.
In the beginning most activity is spend on forming new ties and therefore building the topological structures of the network, but after a while more and more activity is focused on reinforcing existing ties, which leads to an drastic increase of \( r(t) \).
This is also emphasized by the time required to reach the point where more than half of the total activity is spend on reinforcement.
This point is reached for the network with \( q = 0.15 \) after only 109 iterations.
The network with no peer influence takes with 229 iterations about twice as long, indicating that peer influence may play an important role in the first few iterations that shape the topology of the network.


\begin{figure}[htbp]
\centering
\begin{subfigure}[b]{0.485\textwidth}
  \includegraphics[width=\textwidth]{figures/avg-local-cc-start}
  \caption{}
  \label{fig:avg-local-cc-start}
\end{subfigure}
~
\begin{subfigure}[b]{0.485\textwidth}
  \includegraphics[width=\textwidth]{figures/avg-local-cc-end}
  \caption{}
\label{fig:avg-local-cc-end}
\end{subfigure}

\caption[Segments of the average local clustering coefficient evolution]{Segments of the evolution of the local clustering coefficient for different levels of peer influence. (\subref{fig:avg-local-cc-start}) shows the clustering coefficient for the first 10,000 iterations of the simulation, in which it reaches it maximum value and slowly starts to decrease. (\subref{fig:avg-local-cc-end}) depicts the stationary values for \( C \), which can be observed in the last 5,000 iterations.}
\label{fig:avg-local-cc-details}
\end{figure}


\begin{table}
\centering
\begin{tabular}{llllllll}
\( q \) & 0.00 & 0.01 & 0.025 & 0.05 & 0.075 & 0.10 & 0.15 \\ \hline
\( t_{\max} \) & 5,140 & 4,919 & 4,839 & 5,192 & 4,173 & 4,044 & 3,038 \\ \hline
\( C_{\max} \) & 0.5659 & 0.5689 & 0.5721 & 0.5773 & 0.5822 & 0.5895 & 0.5963
\end{tabular}

\caption[\( \max \) and \( \argmax \) of \( C(t) \)]{The maximum value for the local clustering coefficient \( C_{\max} = \max C(t) \) and the time to reach the maximum \( t_{\max} = \argmax C(t) \), for different values of \( q \).}
\label{tbl:max-clustering}
\end{table}


 The share of reinforcement activity does converge to a value of over 90\% for all levels of peer influence after a short period of time, which highlights the domination of the reinforcement process.
\Cref{fig:percentage-reinforced-ties} shows the plots of the percentage of reinforcement activity for the initial phase of the simulation and over all 75,000 iterations.
The plots of \( r(t) \) also reveal a link between the extent of reinforcement that is happening and the level of peer influence.
In the network with no peer influence effects is the proportion of reinforced to created ties on average 0.8961.
The value increases to 0.9550 in the temporal network with a maximum peer influence probability of 15\%.
This can possibly attributed to the overall increased activity within communities, due to the peer influence effects, and the associated bias towards local strong ties.
With other words, the increased local activity overshadows the introduction of random links by low-activity and/or poorly integrated nodes.


\begin{figure}[htbp]
\centering
\begin{subfigure}[b]{0.485\textwidth}
  \includegraphics[width=\textwidth]{figures/percentage-reinforced-ties-full}
  \caption{}
  \label{fig:percentage-reinforced-ties-full}
\end{subfigure}
~
\begin{subfigure}[b]{0.485\textwidth}
  \includegraphics[width=\textwidth]{figures/percentage-reinforced-ties-beginning}
  \caption{}
  \label{fig:percentage-reinforced-ties-beginning}
\end{subfigure}

\caption[Percentage of reinforced ties as function of time]{The percentage of reinforced ties \( r(t) = \frac{\#e_{r}(t)}{\#e_{r}(t) + \#e_{c}(t)} \) for different levels of peer influence as a function of time, where \( \#e_{c}(t) \) and \( \#e_{r}(t) \) are the number of created ties and the number of reinforced ties in iteration \( t \), respectively. (\subref{fig:percentage-reinforced-ties-full}) shows the ratio over all 75,000 iterations and (\subref{fig:percentage-reinforced-ties-beginning}) highlights the behavior in the beginning. Both functions were smoothed using the rolling mean method to improve the quality of the plot.}
\label{fig:percentage-reinforced-ties}
\end{figure}


The evolution of the average local clustering coefficient does dependent on the node deletion probability \( p_{d} \), due to low active nodes which are not removed fast enough and are introducing weak ties~\cite{Laurent2015}.
However, as clearly evident in \cref{fig:avg-local-cc-full}, the possible level of peer influence does influence the clustering, and therefore the community structures of the network, as well.
For example, the more likely an activation due to peer influence gets, the smaller the stationary value for \( C \) becomes.
\Cref{fig:avg-local-cc-end} highlights this effect well.
This can possibly be explained in a similar way as the effect caused by the deletion probability.
However, in this case not the decelerated removal of nodes responsible, but the overall increased activity.
The peer influence mechanism increases the activity in the network, especially in already formed communities (see \cref{sec:network-activity}), since active nodes motivate their neighbors to become active as well.
The probability for the formation of a new tie is inverse proportional to the size of a nodes' egocentric network.
Therefore, a active node which is already fully integrated in its community will reinforce one of its existing ties, or at least close a triangle, with high probability.
However, given enough tries such a node will eventually introduce new weak ties using the focal closure mechanism as well.
Therefore, the opportunities for the introduction of random links by active nodes increases, which leads to a smaller average local clustering in general.

Two additional measures of the integrated network, the average node degree and the average tie strength, were tracked over time for different magnitudes of peer influence as well.
\Cref{fig:avg-weigth-and-tie-strength} depicts the graphs for these two network properties as functions of time.
Both measures show a similar general behavior.
The average degree and the average tie strength are not independent on the maximum peer influence probability in the network and do converge after the integrated network reaches its equilibrium.
The stationary values of both do increase with increasing values for \( q \) by about the same order of magnitude, which is reasonable since both measures are related to each other.
The tie strength of a node can be seen as its weighted degree.
However, the time it takes until they converge differs.
The average degree takes longer to reach its stationary value.
This can be explained by the small probability for the creation of new ties after the egocentric networks have gained a certain size.
Every new neighbor reduces the probability for the creation of new tie in the future significantly.
The average tie strength does not suffer from this problem, due to to fast development of strong ties and the decreasing probability for the introduction of weak ties. \todo{sound explanation?}
The direct effect of the peer influence mechanism on the average degree and average weight can be explained, similar to the average local clustering coefficient, by the additional activity in the temporal network.
The nodes are getting more opportunities to add additional neighbors and to strengthen their ties until they get removed.
Note that the slightly different convergence behavior of the average tie strength for the high peer influence probability \( q = 0.15 \) cannot be explained fully at this point. It may be related to the by comparison significantly slower converge of the average degree, but it is ultimately left open for possible future studies.


\begin{figure}[htbp]
\centering
\begin{subfigure}[b]{0.485\textwidth}
  \includegraphics[width=\textwidth]{figures/avg-degree}
  \caption{}
  \label{fig:avg-degree}
\end{subfigure}
~
\begin{subfigure}[b]{0.485\textwidth}
  \includegraphics[width=\textwidth]{figures/avg-tie-strength}
  \caption{}
\label{fig:avg-tie-strength}
\end{subfigure}

\caption[Average degree and tie strength as function of time]{Plots of (\subref{fig:avg-degree}) the average node degree \( d(G_{T}) \) and (\subref{fig:avg-tie-strength}) the average tie strength \( \langle w \rangle \) in the network as a function of time for different maximum peer influence probabilities \( q = 0, \, 0.01, \, 0.05, \, 0.1, \, 0.15\). }
\label{fig:avg-weigth-and-tie-strength}
\end{figure}


%% ========================================================================
%% ========================================================================


\section{Network Activity}
\label{sec:network-activity}


One obvious implication of the proposed peer influence mechanism is an increase in the activity in the network.
Nodes can become active by them self not only due to their intrinsic activity potential but also by the influence of their peers.
The number of nodes that become active in iteration \( t \) by them self is denoted as \( \#e_{a}(t) \), and the number that become active motivated by others is denoted as \( \#e_{p}(t) \).
Therefore, the total number of contacts per iteration is \( \#e(t) = \#e_{a}(t) + \#e_{p}(t) \).
The number of peer influenced activations in a networks with \( q = 0 \) is trivially \( \#e_{p}(t) = 0 \), and the number of self-activations can be approximated by \( \#e_{a}(t) \approx n \expval{a_{i}} \).
For example, the approximated number of activations in a network with no peer influence effects and the prior specified parameters (\( \gamma = 2.7 \), \( \varepsilon = 0.001 \), and \( n = 5,000 \)) is \( \#e_{a}(t) \approx 12.05 \), which matches the observed figures.
In fact, this number is roughly the same for all levels of peer influence, since the process of node self-activation due to the intrinsic activity potential is independent on the peer influence mechanism.


\myfig{number-of-contacts}
      {width=0.75\textwidth}
      {Time dependent number of activations \( \#e(t) \) in the network for different levels of peer influence \( q \). The graphs were smoothed using the rolling mean method to improve the visualization.}
      {Time dependent number of activations}
      {fig:number-of-contacts}


The total number of activations, and therefore the gain in activity due to the peer influence, is depicted in \cref{fig:number-of-contacts}.
It shows that \( \#e(t) \) reaches a stationary value fast for small levels of peer influence.
However, the development of the total number of contacts per iterations is more complex for networks with a higher degree of peer influence.
The first phase can be described as a rapid increase in the number of activations, which stops after approximately 8,000 iterations.
After that the development of \( \#e(t) \) starts to relax and slow down.
It even starts to decreases slightly in the network with a maximum peer influence probability of \( q = 0.15 \).
Finally, the numbers slowly converge to their stationary values, which are reached after about 40,000 iterations.
This is also supported by the evolution of the fraction of the peer influenced contacts per iterations \( i(t) \), which shows the same general behavior (c.f. \cref{fig:percentage-peer-influenced-activity}).
This effect on the total activity in the network is possibly related to the development of the community structures and its implications on the peer influence mechanism.
The stagnation (or even reduction) of the activity begins after the maximum value for the average local clustering coefficient was reached (c.f. \cref{fig:avg-local-cc-full} for details), which is caused by the introduction of weak ties.
These new ties are possibly responsible for the merging of communities in the network, which in turn could reduce the effects of the peer influence mechanism.
Inactive nodes in these newly merged communities may have an greater impact on the weighted fraction of active nodes that is required for a significant influence on nodes.
This is could especially be true for weakly integrated nodes.
However, a more detailed analysis has to be performed in the future to determine and verify the true effects, which are responsible for the observed behavior in the network activity for higher levels of peer influence. \todo{future work}


\begin{figure}[htbp]
\centering
\begin{subfigure}[b]{0.485\textwidth}
  \includegraphics[width=\textwidth]{figures/percentage-influenced-activity}
 \caption{}
 \label{fig:percentage-peer-influenced-activity-full}
\end{subfigure}
~
\begin{subfigure}[b]{0.485\textwidth}
  \includegraphics[width=\textwidth]{figures/cumulative-number-of-contacts}
  \caption{}
\label{fig:percentage-peer-influenced-and-cumulative-activity}
\end{subfigure}

\caption[Percentage of peer influenced activity and cumulative activity as function of time]{Different views on the ramifications of the peer influence mechanism on the activity in the network. (\subref{fig:percentage-peer-influenced-activity-full}) depicts the evolution of the fraction of peer influenced activity \( i(t) = \sfrac{\#e_{p}(t)}{\#e(t)} \) over time for different levels of peer influence and  (\subref{fig:percentage-peer-influenced-and-cumulative-activity}) shows the aggregated number of contacts \( E(t) \) over all 75,000 iterations. The functions were smoothed using the rolling mean method to improve the quality of the plot.}
\label{fig:percentage-peer-influenced-activity}
\end{figure}


The effects of the peer influence on the activity within the time-varying network can be observed in other ways as well.
For instance, \cref{fig:percentage-peer-influenced-and-cumulative-activity} depicts the cumulative number of contacts (i.e., \( E(t) = \sum_{i \leq t} \#e(i) \)), which highlights the quantity of additional activity that was caused by different degrees of peer influence mechanism over time.
The distribution of the number of node activations in a certain time interval is another example.
There is a shift of the bulk of the distributions observable, which indicates the increased probability for a larger number of activations.


%% ========================================================================
%% ========================================================================


\section{Inter-event Time Distributions}
\label{sec:inter-event-time-dists}

As already mentioned in the motivation section of this thesis (\cref{sec:motivation}) and while discussing different ways to model user activity (\cref{sec:user-activity-models}), human activity patterns can be fairly complex.
They can be described as bursts followed by longer phases of inactivity and are usually characterized by the inter-event time distribution \( \varphi(\tau) \), which should reflect these requirements.
The inter-event times are defined, in the context of this model, as the time between two consecutive activations of a node.
The type of activation (i.e., activation due to peer influence, activity potential, or contact of another active node) is not taken into account for the here performed analysis, but it is a possible topic for future studies.
Each node \( v_{i} \) in the network has its own inter-event time distribution \( \varphi_{i} \), which depends on its activity potential and on the influence of its peers.
However, to get a better overview on how the peer influence mechanism changes the dynamics in the network in general, the union of the inter-event time distributions of all nodes is examined.
The sequence of inter-event times is determined for every node that was active between the beginning and the end of the simulation separately and then combined into one distribution.
This distribution can be seen as a mixture of distributions~\cite{Seidel2011} of all nodes, which were at some time present in the network, i.e.,

\begin{equation}
    \varphi(\tau) = \sum_{i} \pi_{i} \varphi_{i}(\tau),
\end{equation}

where \( \pi_{i} \) denotes the mixing weights, which must be selected such that \( \sum_{i} \pi_{i} = 1 \) holds.
Therefore, the mixing weights determine the importance of every individual distribution.
In this work every distribution is considered equally important and is assigned the same weight, such that the summation constraint is fulfilled.

The burstiness of human behavior is hard to describe and even harder to quantify.
It is usually done using moments of the inter-event time distribution.
For instance, the coefficient of variation~\cite{Masuda2016} is defined as the ratio between the standard deviation \( \sigma \) and the mean \( \mu \) of the inter-event time distribution, i.e., \( c_{v} = \sfrac{\sigma}{\mu} \).
It takes the value \( 0 \) when the events are occurring at a fixed, non-random, rate.
The coefficient of variation is \( 1 \) for exponentially distributed inter-event times, which is the case for events that are generated by a Poisson process, and it can get arbitrarily large for long-tailed distributions, such as power laws.
A normalized variant of the coefficient of variation was proposed by \citet{Goh2008}, which is called burstiness parameter and is defined as

\begin{equation}
    B = \frac{c_{v} - 1}{c_{v} + 1} = \frac{\sigma - \mu}{\sigma + \mu},
\end{equation}

and takes values in the range \( -1 \leq B \leq 1 \).
It is \( B = -1 \) for regularly occurring events, \( B = 0 \) for inter-event times that originated from a Poisson process, and \( B = 1 \) for a distribution that was derived from a extremely bursty sequence of events.
A nice property of this measure is that it can also be applied to mixture distributions that contain the inter-event times of, for example people, with different intrinsic activity levels.
For instance, the inter-event times of power-users that use a system heavily every day can be combined with those of users that use it only irregularly and the burstiness parameter is able to capture the level of burstiness that is present in the usage of the system regardless.
Therefore, it is also applicable for the inter-event time distributions derived from the simulations of the peer influence model.
The burstiness parameter for a network with no peer influence is about \( B = 0.19 \).
The introduction of peer influence increases this value, however, a drastic increase can only be observed for a maximum peer influence probability of \( q \ge 0.15 \).
\Cref{tbl:burstiness-parameter} contains the mean, standard derivation and burstiness parameter of the inter-event time distribution for a maximum peer influence probability in the range \( 0 \leq q \leq 0.15 \).


\begin{table}[htbp]
\centering
\begin{tabular}{llllllll}
\( q \) & 0.00 & 0.01 & 0.025 & 0.05 & 0.075 & 0.10 & 0.15 \\ \hline
\( \mu \) & 198.71 & 184.59 & 164.26 & 132.80 & 102.43 & 76.28 & 37.23 \\ \hline
\( \sigma \) & 291.32 & 270.49 & 241.40 & 197.38 & 155.09 & 118.04 & 61.22 \\ \hline
\( B \) & 0.1890 & 0.1888 & 0.1902 & 0.1956 & 0.2045 & 0.2149 & 0.2437
\end{tabular}

\caption[Burstiness of inter-event time distributions]{Mean value, standard deviation and the resulting burstiness parameter of the inter-event time distribution for different levels of peer influence.}
\label{tbl:burstiness-parameter}
\end{table}


This indicates that the peer influence mechanism influences the burstiness of the node activations.
In a network with no peer influence are two consecutive self-activations independent of each other.
Therefore, the activations happen at a certain rate that is proportional to the activity potential of a node, which leads to exponentially distributed inter-event times~\cite{Moinet2016}.
This should lead to a burstiness parameter that is close to \( B = 0 \).
However, this is not the case for the inter-event times that were generated with the proposed model even though peer influence was disabled (c.f. \cref{tbl:burstiness-parameter}).
It can be explained by the the memory effects in the model, which allow reoccurring interactions within groups of nodes and by the inclusion of passive activations due to other active nodes in the calculation of the inter-event times.
A node with a higher intrinsic activity potential will select a node from its local group with high probability.
Hence, this more active node activates less active nodes regularly, which can lead to a more bursty looking activity pattern for the other nodes and explains (at least partially) the burstiness value of \( B = 0.19 \).
The peer influence mechanism of this model should amplify this effect even further, since it increases the activity within communities.
Furthermore, the activation of nodes possibly triggers cascading activations within the communities, which makes bursts more likely as well.

One way to examine how the peer influence mechanism changes the inter-node-activation times in the network is to inspect their distribution visually.
\Cref{fig:inter-event-time-dist-loglog} depicts the inter-event distributions for a variety of peer influence levels.
The plot shows the distribution on a log-log graph, where both axis are scaled logarithmically.
This highlights that longer inter-event times are indeed possible between two consecutive activations.
The plot also shows that the level of peer influence in the network changes the shape of the distributions.
Smaller inter-event times become more likely for higher levels of peer influence, which can be seen in the mean value of the distribution as well (c.f. \cref{tbl:burstiness-parameter}).
The average time between activations decreases from about 200 in a network with no peer influence at all, to 38 in a network with \( q = 0.15 \).
The change of the distribution is even more noticeably for inter-event times less than ten time steps.
\Cref{fig:inter-event-time-dist-start} depicts the distributions for this interval of \( \tau \).
The probability for two successive activations increases drastically from less than 2\% to almost 22\% over the range of possible values for \( q \).
This change in the probabilities possibly explains the increase of the burstiness of the distributions, since a larger number of events within a small time frame become more probable due to the peer influence mechanism.
However, the tail of the inter-event time distribution changes for higher levels of peer influence as well.
Large inter-event times become more and more unlikely for larger values of \( q \), and the length of the tails get shorter as well.
The values for the standard derivation of the distributions reflect this behavior as well (c.f. \cref{tbl:burstiness-parameter}).
The standard derivation of the inter-event time in the network with high peer influence effects is only about one fifth of the standard derivation of the network without peer effects.
This of course prevents longer intervals of inactivity to a certain extend, which is the second crucial requirement for realistic activity patterns.


\myfig{inter-event-time-dist-loglog}
      {width=0.75\textwidth}
      {Log-log Plot of the inter-event time distribution for different maximum peer influence probabilities \( q = 0, \, 0.01, \, 0.05, \, 0.1, \, 0.15\).}
      {Log-log plot of the inter-event time distribution}
      {fig:inter-event-time-dist-loglog}


\myfig{inter-event-time-dist-start}
      {width=0.75\textwidth}
      {Plot of the bulk of the inter-event time distribution for small inter-event times in the range \( 1 \leq \tau \leq 10 \) for different levels of peer influence.}
      {Bulk of the inter-event time distribution}
      {fig:inter-event-time-dist-start}



%% ========================================================================
%% ========================================================================


\section{Softmax Weight Rescaling}
\label{sec:softmax-rescaling}
