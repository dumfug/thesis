% make critical peer influence threshold dependend on the egocentric network size and strucutre?

% only syntetic networks.. especially inter event dist --> show plots of real world dists

% peer influence increases porbability for small values for tau, however, it also decreases the std of the distirbution --> possibly stop the peer influence process in communities to avoid these cascading effects where one node motivates another node in the next round and then this node is motivatied by the other node as well in the node after the next and so on
% inter-event time dsit is not yet realsitic for the current model

% strange stagnation/decline in activity for high q values may be better observable for differnt weight rescaling (weight is equal for all ties) --> experiment

% experiment cascading activations

% SI process ... cascading activations as si process...

\chapter{Conclusion}
\label{cha:conclusion}

This last chapter of the thesis not only presents a review on the obtained results, but also contains a short summation of the previous chapters.
It contains a discussion of possible limitations of the proposed peer influence model and the methods that were used to evaluate and test it as well.
The chapter concludes with suggestions for possible future studies.
These suggestion include improvements for the model itself and potential interesting experiments on synthetic and real-world networks.


%% ========================================================================
%% ========================================================================


\section{Recap}
\label{sec:recap}

The thesis starts with a short motivational section, which sequentially leads to the incentive for the proposal of an extension to the activity-driven time-varying network framework~\cite{Perra2012a} that incorporates peer influence effects.
The basic idea behind the proposal is that people do not solely perform an action as a result of their intrinsic motivation to do it, but also because of the influence of their peers.
This stands in contrast to the activity-driven network model in which nodes can become active only by them self based on their activity potential.
Therefore, it provides an ideal foundation to incorporate additional peer influence effects.

Moreover, preliminaries on the basic topics of graph theory, various types of networks and related generative models, and time-varying networks are discussed.
A detailed review on the activity-driven framework itself, the properties of temporal networks it generates, and on the latest related literature that either adapts or extents it is performed as well.
Furthermore, an overview on sociological peer influence studies and network models, which utilize different types of peer influence mechanisms is also an element of this thesis.


%% ========================================================================
%% ========================================================================


\section{Limitations}
\label{sec:limitations}

\ldots


%% ========================================================================
%% ========================================================================


\section{Future Work}
\label{sec:future-work}

\ldots
