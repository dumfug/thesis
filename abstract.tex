\chapter*{Abstract}
\label{cha:abstract}

The field of time-varying networks opened up the possibility to explore dynamical systems with respect to their temporal dimension in a more precise way.
One class of these systems that is particularly interesting are networks of human interactions.
They are characterized by the complex patterns of human behavior, which can be denoted as bursty.
The recent introduction of the activity-driven time-varying network framework led to an increased effort to model these social systems more accurately.
However, all of these approaches rely solely on an intrinsic property of individuals to describe their activity patterns and disregard possible external influences entirely.
In this thesis an activity-driven network model is proposed, which introduces a peer influence mechanism into the network dynamics, and thus allowing individuals to motivate their neighbors in the social network to become active as well.
The ramifications of this mechanism on the topological and activity-related properties of synthetically generated networks are examined and reveal its complex influence on the dynamics.
Furthermore, the results indicate a positive effect on the emerging activity patterns.


\textbf{Keywords:} \mykeywords
