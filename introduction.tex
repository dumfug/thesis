\chapter{Introduction}
\label{cha:introduction}


\section{Motivation}
\label{sec:motivation}




%% ========================================================================
%% ========================================================================


\section{Outline}
\label{sec:outline}

This content of this thesis is structured as follows.
Chapter~\ref{cha:related-work} contains in the beginning some basic graph-theoretic definitions (\autoref{sec:graph-theory-basics}) and an introduction into the topics of social and temporal networks (\autoref{sec:social-networks} and \autoref{sec:time-varying-networks} respectively).
Additionally, an overview of some important generative network models is given in \autoref{sec:network-models} and related work in the context of user behavioral models, especially for models that use temporal networks, is discussed in \autoref{sec:user-activity-models}.
The last section (\autoref{sec:peer-influence}) of this chapter is related to peer influence and its ramifications and applications in different fields.

In \autoref{cha:model} the proposed model is discussed in great detail.
First, the model and ideas that it is based upon are outlined in \autoref{sec:base-model}.
The extension that incorporates peer influence effects into the base model and the idea behind is described in \autoref{sec:peer-influence-model}.
Chapter~\ref{cha:results} contains the evaluation and analysis of the proposed model on synthetic networks

The last chapter of this document (\autoref{cha:conclusion}) includes a summary of the archived results and a conclusion.
Furthermore, different applications and possible extensions of the peer influence model are discussed in the last section of this thesis.
