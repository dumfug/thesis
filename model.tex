\chapter{Model}


\section{The Base Model}

\todo{rewrite!}
% explain in short the relation to the activity-driven framework (i.e., the activity potentials) and note the distribution that is used to draw the samples and note that w.l.o.g. delta t = eta = 1 and epsilon is 10^-3 in the beginning of this section


To generate artificial contact sequences, a extension of an already established activity driven model by \citet{Laurent2015} was used.
This model, which is based the activity-driven framework by \citet{Perra2012a} (see \autoref{subsec:activity-driven-models} for details), introduces additional social mechanisms and memory effects to provide a more realistic view on temporal networks.
It produces adjustable community structures and weight-topological correlations by adding the following three mechanisms:

\begin{enumerate}
    \item Memory effects
    \item Closure processes
    \item Node deletion
\end{enumerate}

The first mechanism introduces memory to the nodes. \todo{this is based on the idea in: time varying networks and the weakness of strong ties (explain more in detail)}
A node remembers all previous interactions with other nodes (i.e., its neighbors) and is prone to repeat them.
The memory of a node is represented by a weighted egocentric network, that includes all other nodes, which were already part of an interactions in the past, and the weight represent the number of previous interactions.
The choice of forming a new tie or reinforcing a existing one depends on how many nodes are already in the egocentric network.
Let \(k_{i} = d(v_{i})\) be the degree of the node \(v_{i}\), then the probability for a new tie is \(p(k_{i}) = \frac{c}{k_{i} + c}\), and the probability to interact with a already established tie is \(\bar{p}(k_{i}) = 1 - p(k_{i}) = 1 - \frac{c}{k_{i} + c} = \frac{k_{i}}{k_{i} + c}\), where the constant \(c\) determines how strong the memory of a user is (cf.  \autoref{fig:reinforcement-process-prob-plot} for details).
Therefore, the probability for new ties decays exponentially with the degree of the node.
This corresponds to the observations of actual social interaction dynamics, where actors tend to communicate within their social cycle, which has a limited size due to cognitive capacities of the actors.
The memory mechanism allows the introduction of dependencies of successive interactions between nodes and replaces the approach of selecting a communication partner uniformly at random when a node becomes active in the original activity-driven framework.

\myfig{reinforcement-process-prob-function}
      {width=0.75\textwidth}
      {Plots of the function that determines the probability for the formation of a new tie based on the degree of a node \(p(k)\) for different values of the scaling factor for the memory strength \(c\). This constant can help to model different types of users. Larger values may correspond to social explorers, that are more prone to form new ties, and smaller values are related to social keepers, which communicate almost exclusively to peers in their social circle.}
      {Probability distribution for the formation of new ties.}
      {fig:reinforcement-process-prob-plot}

The second mechanism introduces two different closure processes to the model.
The first one, cyclic closure, assures the formation of triangles (i.e., cliques between three nodes), which are linked to the formation of community structures in the network.
If a node tries to form a new tie, it performs with probability \(p_{\Delta}\) a cyclic closure by interacting with a neighbor of a neighbor at random.
The second closure process, focal closure, tries to emulate the social dynamic that users tend to form ties with other users that are similar (e.g., they have common interests).
This process is also performed when a new tie should be formed and has a probability of \(1 - p_{\Delta}\) and is simply done by selecting a new node to interact with uniformly at random (i.e., by a random search in the network).

The third, and last, mechanism that is included into the model is the deletion of nodes.
In the original activity-driven framework, nodes live forever and are forever part of the network.
In this extension, nodes have an intrinsic probability to die \(p_{d}\) in every time step.
This ensures that the network can reach a stable state in which the structural characteristics (e.g., the community structures) become invariant in time.
However, every time a node is removed from the network, a new one joins to keep the size of the network constant.

The probabilities \(p_{\Delta}\) \(p_{d}\) are fixed and the same for every node in the network.
The memory strength is fixed to \(c = 1\).
Other parameters of the activity-driven framework are \(\eta = \Delta t = 1\), the lower activity bound \(\varepsilon = 0.001\), and the number of interactions per node \(m = 1\).

\todo{describe used activity dist but with peer influence}


%% ========================================================================
%% ========================================================================
