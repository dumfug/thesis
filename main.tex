\newcommand{\mypapersize}{A4}
%% e.g., "A4", "letter", "legal", "executive", ...
%% The size of the paper of the resulting PDF file.

\newcommand{\mylaterality}{oneside} %TODO: change to twoside when finished
%% "oneside" or "twoside"
%% Either you are creating a document which is printed on both, left pages
%% and right pages (twoside) or you create a document which is printed
%% on right pages only (oneside).

\newcommand{\mydraft}{true}
%% "true" or "false"
%% Use draft mode? If true, included graphics are replaced by empty
%% rectangles (of same size) and overfull boxes (in margin space) are
%% marked with black box (-> easy to spot!)

\newcommand{\myparskip}{full}
%% e.g., "no", "full", "half", ...
%% How to separate paragraphs: indention ("no") or spacing ("half",
%% "full", ...).

\newcommand{\myBCOR}{0mm}
%% Inner binding correction. This value depends on the method which is
%% being used to bind your printed result. Some techniques do not
%% require a binding correction at all ("0mm"), other require for
%% example "5mm". Refer to KOMA script documentation for a detailed
%% explanation what a binding correction is and how to measure it.

\newcommand{\myfontsize}{12pt}
%% e.g., 10pt, 11pt, 12pt
%% The font size of the main text in pt (points).

\newcommand{\mylinespread}{1.0}
%% e.g., 1.0, 1.5, 2.0
%% Line spacing in %/100. For example 1.5 means 150% of the usual line
%% spacing. Please use with caution: 100% ("1.0") is fine because the
%% font was designed for it.

\newcommand{\mylanguage}{ngerman,american}
%% "english,ngerman", "ngerman,english", ...
%% NOTE: The *last* language is the active one!
%% See babel documentation for further details.

%% BibLaTeX-settings: (see biblatex reference for further description)
\newcommand{\mybiblatexstyle}{alphabetic}
%% e.g., "alphabetic", "authoryear", ...
%% The biblatex style which is being used for referencing. See
%% biblatex documentation for further details and more values.
%%
%% CAUTION: if you change the style, please check for (in)compatible
%%          "biblatex" package options in the file
%%          "template/preamble.tex"! For example: "alphabetic" does
%%          not have an option "dashed=..." and causes an error if it
%%          does not get removed from the list of options.

\newcommand{\mybiblatexdashed}{false}  %% "true" or "false"
%% If true: replace recurring reference authors with a dash.

\newcommand{\mybiblatexbackref}{true}  %% "true" or "false"
%% If true: create backward links from reference to citations.

\newcommand{\mybiblatexfile}{references-biblatex.bib}
%% Name of the biblatex file that holds the references.

\newcommand{\mydispositioncolor}{0,0,0}
%% e.g., "30,103,182" (blue/turquois), "0,0,0" (black), ...
%% Color of the headings and so forth in RGB (red,green,blue) values.
%% NOTE: if you are using "0,0,0" for black, printers might still
%%       recognize pages as color pages. In case this is a problem
%%       (paying for color print-outs vs. paying for b/w-printouts)
%%       please edit file "template/preamble.tex" and change
%%       "\definecolor{DispositionColor}{RGB}{\mydispositioncolor}"
%%       to "\definecolor{DispositionColor}{gray}{0}" and thus
%%       overwriting the value of \mydispositioncolor above.

\newcommand{\mycolorlinks}{true}  %% "true" or "false"
%% Enables or disables colored links (hyperref package).

\newcommand{\mytitlepage}{template/title_Thesis_TU_Graz}
%% Your own or one of following pre-defined title pages:
%% "template/title_plain_maketitle": simple maketitle page
%% "template/title_Diplomarbeit_KF_Uni_Graz.tex": fancy (german) title page for KF Uni Graz
%% "template/title_Thesis_TU_Graz":
%%             titlepage for Graz University of Technology (correct
%%             (old?) Corporate Design) by Karl Voit (2012)
%% "template/title_Thesis_TU_Graz_-_kazemakase":
%%             titlepage for Graz University of Technology
%%             (correct new Corporate Design) by kazemakase (2013):
%%             see https://github.com/novoid/LaTeX-KOMA-template/issues/5
%% "template/title_VWA": titlepage for Vorwissenschaftliche Arbeit

\newcommand{\mytodonotesoptions}{}
%% e.g., "" (empty), "disable", ...
%% Options for the todonotes-package. If "disable", all todonotes will
%% be hidden (including listoftodos).

%% Load main settings for document preamble:
%% Time-stamp: <2015-04-30 17:23:24 vk>
%%%% === Disclaimer: =======================================================
%% created by
%%
%%      Karl Voit
%%
%% using GNU/Linux, GNU Emacs & LaTeX 2e
%%

%doc% %% overriding preamble/preamble.tex %%
%doc% \newcommand{\mylinespread}{1.0}  \newcommand{\mycolorlinks}{true}
%doc% \documentclass[12pt,paper=a4,parskip=half,DIV=calc,oneside,%%
%doc% headinclude,footinclude=false,open=right,bibliography=totoc]{scrartcl}
%doc% \usepackage[utf8]{inputenc}\usepackage[ngerman,american]{babel}\usepackage{scrpage2}
%doc% \usepackage{ifthen}\usepackage{eurosym}\usepackage{xspace}\usepackage[usenames,dvipsnames]{xcolor}
%doc% \usepackage[protrusion=true,factor=900]{microtype}
%doc% \usepackage{enumitem}
%doc% \usepackage[pdftex]{graphicx}
%doc% \usepackage{todonotes}
%doc% \usepackage{dingbat,bbding} %% special characters
%doc% \definecolor{DispositionColor}{RGB}{30,103,182}
%doc%
%doc% \usepackage[backend=biber,style=authoryear,dashed=false,natbib=true,hyperref=true%%
%doc% ]{biblatex}
%doc%
%doc% \addbibresource{references-biblatex.bib} %% remove, if using BibTeX instead of biblatex
%doc%
%doc% %% overriding userdata %%
%doc% \newcommand{\myauthor}{Karl Voit}\newcommand{\mytitle}{LaTeX Template Documentation}
%doc% \newcommand{\mysubject}{A Comprehensive Guide to Use the
%doc% Template from https://github.com/novoid/LaTeX-KOMA-template}
%doc% \newcommand{\mykeywords}{LaTeX, pdflatex, template, documentation, biber, biblatex}
%doc%
%doc% \newcommand{\myLaT}{\LaTeX{}@TUG\xspace}
%doc%
%doc% %% for future use?
%doc% % \usepackage{filecontents}
%doc% % \begin{filecontents}{filename.example}
%doc% %
%doc% % \end{filecontents}
%doc%
%doc%
%doc% %% using existing TeX files %%
%doc% %% Time-stamp: <2015-04-30 17:19:58 vk>
%%%% === Disclaimer: =======================================================
%% created by
%%
%%      Karl Voit
%%
%% using GNU/Linux, GNU Emacs & LaTeX 2e
%%

%doc%
%doc% \section{\texttt{mycommands.tex} --- various definitions}\myinteresting
%doc% \label{sec:mycommands}
%doc%
%doc% In file \verb#template/mycommands.tex# many useful commands are being
%doc% defined.
%doc%
%doc% \paragraph{What should I do with this file?} Please take a look at its
%doc% content to get the most out of your document.
%doc%

%doc%
%doc% One of the best advantages of \LaTeX{} compared to \myacro{WYSIWYG} software products is
%doc% the possibility to define and use macros within text. This empowers the user to
%doc% a great extend.  Many things can be defined using \verb#\newcommand{}# and
%doc% automates repeating tasks. It is recommended to use macros not only for
%doc% repetitive tasks but also for separating form from content such as \myacro{CSS}
%doc% does for \myacro{XHTML}. Think of including graphics in your document: after
%doc% writing your book, you might want to change all captions to the upper side of
%doc% each figure. In this case you either have to modify all
%doc% \texttt{includegraphics} commands or you were clever enough to define something
%doc% like \verb#\myfig#\footnote{See below for a detailed description}. Using a
%doc% macro for including graphics enables you to modify the position caption on only
%doc% \emph{one} place: at the definition of the macro.
%doc%
%doc% The following section describes some macros that came with this document template
%doc% from \myLaT and you are welcome to modify or extend them or to create
%doc% your own macros!
%doc%

%doc%
%doc% \subsection{\texttt{myfig} --- including graphics made easy}
%doc%
%doc% The classic: you can easily add graphics to your document with \verb#\myfig#:
%doc% \begin{verbatim}
%doc%  \myfig{flower}%% filename w/o extension in the folder figures
%doc%        {width=0.7\textwidth}%% maximum width/height, aspect ratio will be kept
%doc%        {This flower was photographed at my home town in 2010}%% caption
%doc%        {Home town flower}%% optional (short) caption for list of figures
%doc%        {fig:flower}%% label
%doc% \end{verbatim}
%doc%
%doc% There are many advantages of this command (compared to manual
%doc% \texttt{figure} environments and \texttt{includegraphics} commands:
%doc% \begin{itemize}
%doc% \item consistent style throughout the whole document
%doc% \item easy to change; for example move caption on top
%doc% \item much less characters to type (faster, error prone)
%doc% \item less visual clutter in the \TeX{}-files
%doc% \end{itemize}
%doc%
%doc%
\newcommand{\myfig}[5]{
%% example:
% \myfig{}%% filename in figures folder
%       {width=0.5\textwidth,height=0.5\textheight}%% maximum width/height, aspect ratio will be kept
%       {}%% caption
%       {}%% optional (short) caption for list of figures
%       {}%% label
\begin{figure}[thp]
  \centering
  \includegraphics[keepaspectratio,#2]{figures/#1}
  \caption[#4]{#3}
  \label{#5} %% NOTE: always label *after* caption!
\end{figure}
}


%doc%
%doc% \subsection{\texttt{myclone} --- repeat things!}
%doc%
%doc% Using \verb#\myclone[42]{foobar}# results the text \enquote{foobar} printed 42 times.
%doc% But you can not only repeat text output with \texttt{myclone}.
%doc%
%doc% Default argument
%doc% for the optional parameter \enquote{number of times} (like \enquote{42} in the example above)
%doc% is set to two.
%doc%
%% \myclone[x]{text}
\newcounter{myclonecnt}
\newcommand{\myclone}[2][2]{%
  \setcounter{myclonecnt}{#1}%
  \whiledo{\value{myclonecnt}>0}{#2\addtocounter{myclonecnt}{-1}}%
}

%old% %d oc%
%old% %d oc% \subsection{\texttt{fixxme} --- sidemark something as unfinished}
%old% %d oc%
%old% %d oc% You know it: something has to be fixed and you can not do it right
%old% %d oc% now. In order to \texttt{not} forget about it, you might want to add a
%old% %d oc% note like \verb+\fixxme{check again}+ which inserts a note on the page
%old% %d oc% margin such as this\fixxme{check again} example.
%old% %d oc%
%old% \newcommand{\fixxme}[1]{%%
%old% \textcolor{red}{FIXXME}\marginpar{\textcolor{red}{#1}}%%
%old% }


%%%% End
%%% Local Variables:
%%% mode: latex
%%% mode: auto-fill
%%% mode: flyspell
%%% eval: (ispell-change-dictionary "en_US")
%%% TeX-master: "../main"
%%% End:
%% vim:foldmethod=expr
%% vim:fde=getline(v\:lnum)=~'^%%%%'?0\:getline(v\:lnum)=~'^%doc.*\ .\\%(sub\\)\\?section{.\\+'?'>1'\:'1':

%doc% %%%% Time-stamp: <2015-08-22 17:20:32 vk>
%%%% === Disclaimer: =======================================================
%% created by
%%
%%      Karl Voit
%%
%% using GNU/Linux, GNU Emacs & LaTeX 2e
%%
%doc%
%doc% \section{\texttt{typographic\_settings.tex} --- Typographic finetuning}
%doc%
%doc% The settings of file \verb#template/typographic_settings.tex# contain
%doc% typographic finetuning related to things mentioned in literature.  The
%doc% settings in this file relates to personal taste and most of all:
%doc% \emph{typographic experience}.
%doc%
%doc% \paragraph{What should I do with this file?} You might as well skip the whole
%doc% file by excluding the \verb#%%%% Time-stamp: <2015-08-22 17:20:32 vk>
%%%% === Disclaimer: =======================================================
%% created by
%%
%%      Karl Voit
%%
%% using GNU/Linux, GNU Emacs & LaTeX 2e
%%
%doc%
%doc% \section{\texttt{typographic\_settings.tex} --- Typographic finetuning}
%doc%
%doc% The settings of file \verb#template/typographic_settings.tex# contain
%doc% typographic finetuning related to things mentioned in literature.  The
%doc% settings in this file relates to personal taste and most of all:
%doc% \emph{typographic experience}.
%doc%
%doc% \paragraph{What should I do with this file?} You might as well skip the whole
%doc% file by excluding the \verb#\input{template/typographic_settings.tex}# command
%doc% in \texttt{main.tex}.  For standard usage it is recommended to stay with the
%doc% default settings.
%doc%
%doc%
%% ========================================================================

%doc%
%doc% Some basic microtypographic settings are provided by the
%doc% \texttt{microtype}
%doc% package\footnote{\url{http://ctan.org/pkg/microtype}}. This template
%doc% uses the rather conservative package parameters: \texttt{protrusion=true,factor=900}.
\usepackage[protrusion=true,factor=900]{microtype}

%doc%
%doc% \subsection{French spacing}
%doc%
%doc% \paragraph{Why?} see~\textcite[p.\,28, p.\,30]{Bringhurst1993}: `2.1.4 Use a single word space between sentences.'
%doc%
%doc% \paragraph{How?} see~\textcite[p.\,185]{Eijkhout2008}:\\
%doc% \verb#\frenchspacing  %% Macro to switch off extra space after punctuation.# \\
\frenchspacing  %% Macro to switch off extra space after punctuation.
%doc%
%doc% Note: This setting might be default for \myacro{KOMA} script.
%doc%


%doc%
%doc% \subsection{Font}
%doc%
%doc% This template is using the Palatino font (package \texttt{mathpazo}) which results
%doc% in a legible document and matching mathematical fonts for printout.
%doc%
%doc% It is highly recommended that you either stick to the Palatino font or use the
%doc% \LaTeX{} default fonts (by removing the package \texttt{mathpazo}).
%doc%
%doc% Chosing different fonts is not
%doc% an easy task. Please leave this to people with good knowledge on this subject.
%doc%
%doc% One valid reason to change the default fonts is when your document is mainly
%doc% read on a computer screen. In this case it is recommended to switch to a font
%doc% \textsf{which is sans-serif like this}. This template contains several alternative
%doc% font packages which can be activated in this file.
%doc%

% for changing the default font, please go to the next subsection!

%doc%
%doc% \subsection{Text figures}
%doc%
%doc% \ldots also called old style numbers such as 0123456789.
%doc% (German: \enquote{Mediäval\-ziffern\footnote{\url{https://secure.wikimedia.org/wikibooks/de/wiki/LaTeX-W\%C3\%B6rterbuch:\_Medi\%C3\%A4valziffern}}})
%doc% \paragraph{Why?} see~\textcite[p.\,44f]{Bringhurst1993}:
%doc% \begin{quote}
%doc% `3.2.1 If the font includes both text figures and titling figures, use
%doc%  titling figures only with full caps, and text figures in all other
%doc%  circumstances.'
%doc% \end{quote}
%doc%
%doc% \paragraph{How?}
%doc% Quoted from Wikibooks\footnote{\url{https://secure.wikimedia.org/wikibooks/en/wiki/LaTeX/Formatting\#Text\_figures\_.28.22old\_style.22\_numerals.29}}:
%doc% \begin{quote}
%doc% Some fonts do not have text figures built in; the textcomp package attempts to
%doc% remedy this by effectively generating text figures from the currently-selected
%doc% font. Put \verb#\usepackage{textcomp}# in your preamble. textcomp also allows you to
%doc% use decimal points, properly formatted dollar signs, etc. within
%doc% \verb#\oldstylenums{}#.
%doc% \end{quote}
%doc% \ldots but proposed \LaTeX{} method does not work out well. Instead use:\\
%doc% \verb#\usepackage{hfoldsty}#  (enables text figures using additional font) or \\
%doc% \verb#\usepackage[sc,osf]{mathpazo}# (switches to Palatino font with small caps and old style figures enabled).
%doc%
%\usepackage{hfoldsty}  %% enables text figures using additional font
%% ... OR use ...
\usepackage[sc,osf]{mathpazo} %% switches to Palatino with small caps and old style figures

%% Font selection from:
%%     http://www.matthiaspospiech.de/latex/vorlagen/allgemein/preambel/fonts/
%% use following lines *instead* of the mathpazo package above:
%% ===== Serif =========================================================
%% for Computer Modern (LaTeX default font), simply remove the mathpazo above
%\usepackage{charter}\linespread{1.05} %% Charter
%\usepackage{bookman}                  %% Bookman (laedt Avant Garde !!)
%\usepackage{newcent}                  %% New Century Schoolbook (laedt Avant Garde !!)
%% ===== Sans Serif ====================================================
%\renewcommand{\familydefault}{\sfdefault}  %% this one in *combination* with the default mathpazo package
%\usepackage{cmbright}                  %% CM-Bright (eigntlich eine Familie)
%\usepackage{tpslifonts}                %% tpslifonts % Font for Slides


%doc%
%doc% \subsection{\texttt{myacro} --- Abbrevations using \textsc{small caps}}\myinteresting
%doc% \label{sec:myacro}
%doc%
%doc% \paragraph{Why?} see~\textcite[p.\,45f]{Bringhurst1993}: `3.2.2 For abbrevations and
%doc% acronyms in the midst of normal text, use spaced small caps.'
%doc%
%doc% \paragraph{How?} Using the predefined macro \verb#\myacro{}# for things like
%doc% \myacro{UNO} or \myacro{UNESCO} using \verb#\myacro{UNO}# or \verb#\myacro{UNESCO}#.
%doc%
\DeclareRobustCommand{\myacro}[1]{\textsc{\lowercase{#1}}} %%  abbrevations using small caps


%doc%
%doc% \subsection{Colorized headings and links}
%doc%
%doc% This document template is able to generate an output that uses colorized
%doc% headings, captions, page numbers, and links. The color named `DispositionColor'
%doc% used in this document is defined near the definition of package \texttt{color}
%doc% in the preamble (see section~\ref{subsec:miscpackages}). The changes required
%doc% for headings, page numbers, and captions are defined here.
%doc%
%doc% Settings for colored links are handled by the definitions of the
%doc% \texttt{hyperref} package (see section~\ref{sec:pdf}).
%doc%
\setheadsepline{.4pt}[\color{DispositionColor}]
\renewcommand{\headfont}{\normalfont\sffamily\color{DispositionColor}}
\renewcommand{\pnumfont}{\normalfont\sffamily\color{DispositionColor}}
\addtokomafont{disposition}{\color{DispositionColor}}
\addtokomafont{caption}{\color{DispositionColor}\footnotesize}
\addtokomafont{captionlabel}{\color{DispositionColor}}

%doc%
%doc% \subsection{No figures or tables below footnotes}
%doc%
%doc% \LaTeX{} places floating environments below footnotes if \texttt{b}
%doc% (bottom) is used as (default) placement algorithm. This is certainly
%doc% not appealing for most people and is deactivated in this template by
%doc% using the package \texttt{footmisc} with its option \texttt{bottom}.
%doc%
%% see also: http://www.komascript.de/node/858 (German description)
\usepackage[bottom]{footmisc}

%doc%
%doc% \subsection{Spacings of list environments}
%doc%
%doc% By default, \LaTeX{} is using vertical spaces between items of enumerate,
%doc% itemize and description environments. This is fine for multi-line items.
%doc% Many times, the user does just write single-line items where the larger
%doc% vertical space is inappropriate. The \href{http://ctan.org/pkg/enumitem}{enumitem}
%doc% package provides replacements for the pre-defined list environments and
%doc% offers many options to modify their appearances.
%doc% This template is using the package option for \texttt{noitemsep} which
%doc% mimimizes the vertical space between list items.
%doc%
\usepackage{enumitem}
%\setlist{noitemsep}   %% kills the space between items

%doc%
%doc% \subsection{\texttt{csquotes} --- Correct quotation marks}\myinteresting
%doc% \label{sub:csquotes}
%doc%
%doc% \emph{Never} use quotation marks found on your keyboard.
%doc% They end up in strange characters or false looking quotation marks.
%doc%
%doc% In \LaTeX{} you are able to use typographically correct quotation marks. The package
%doc% \href{http://www.ctan.org/pkg/csquotes}{\texttt{csquotes}} offers you with
%doc% \verb#\enquote{foobar}# a command to get correct quotation marks around \enquote{foobar}.
%doc% Please do check the package options in order to modify
%doc% its settings according to the language used\footnote{most of the time in
%doc% combination with the language set in the options of the \texttt{babel} package}.
%doc%
%doc% \href{http://www.ctan.org/pkg/csquotes}{\texttt{csquotes}} is also recommended
%doc% by \texttt{biblatex} (see Section~\ref{sec:references}).
\usepackage[babel=true,strict=true,english=american,german=guillemets]{csquotes}

%doc%
%doc% \subsection{Line spread}
%doc%
%doc% If you have to enlarge the distance between two lines of text, you can
%doc% increase it using the \texttt{\mylinespread} command in \texttt{main.tex}. By default, it is
%doc% deactivated (set to 100~percent). Modify only with caution since it influences the
%doc% page layout and could lead to ugly looking documents.
\linespread{\mylinespread}

%doc%
%doc% \subsection{Optional: Lines above and below the chapter head}
%doc%
%doc% This is not quite something typographic but rather a matter of taste.
%doc% \myacro{KOMA} Script offers \href{http://www.komascript.de/node/24}{a method to
%doc% add lines above and below chapter head} which is disabled by
%doc% default. If you want to enable this feature, remove corresponding
%doc% comment characters from the settings.
%doc%
%% Source: http://www.komascript.de/node/24
%disabled% %% 1st get a new command
%disabled% \newcommand*{\ORIGchapterheadstartvskip}{}%
%disabled% %% 2nd save the original definition to the new command
%disabled% \let\ORIGchapterheadstartvskip=\chapterheadstartvskip
%disabled% %% 3rd redefine the command using the saved original command
%disabled% \renewcommand*{\chapterheadstartvskip}{%
%disabled%   \ORIGchapterheadstartvskip
%disabled%   {%
%disabled%     \setlength{\parskip}{0pt}%
%disabled%     \noindent\color{DispositionColor}\rule[.3\baselineskip]{\linewidth}{1pt}\par
%disabled%   }%
%disabled% }
%disabled% %% see above
%disabled% \newcommand*{\ORIGchapterheadendvskip}{}%
%disabled% \let\ORIGchapterheadendvskip=\chapterheadendvskip
%disabled% \renewcommand*{\chapterheadendvskip}{%
%disabled%   {%
%disabled%     \setlength{\parskip}{0pt}%
%disabled%     \noindent\color{DispositionColor}\rule[.3\baselineskip]{\linewidth}{1pt}\par
%disabled%   }%
%disabled%   \ORIGchapterheadendvskip
%disabled% }

%doc%
%doc% \subsection{Optional: Chapter thumbs}
%doc%
%doc% This is not quite something typographic but rather a matter of taste.
%doc% \myacro{KOMA} Script offers \href{http://www.komascript.de/chapterthumbs-example}{a method to
%doc% add chapter thumbs} (in combination with the package \texttt{scrpage2}) which is disabled by
%doc% default. If you want to enable this feature, remove corresponding
%doc% comment characters from the settings.
%doc%
%disabled% \makeatletter
%disabled% % Safty first
%disabled% \@ifundefined{chapter}{\let\chapter\undefined
%disabled%   \chapter must be defined to use chapter thumbs!}{%
%disabled%
%disabled% % Two new commands for the width and height of the boxes with the
%disabled% % chapter number at the thumbs (use of commands instead of lengths
%disabled% % for sparing registers)
%disabled% \newcommand*{\chapterthumbwidth}{2em}
%disabled% \newcommand*{\chapterthumbheight}{1em}
%disabled%
%disabled% % Two new commands for the colors of the box background and the
%disabled% % chapter numbers of the thumbs
%disabled% \newcommand*{\chapterthumbboxcolor}{black}
%disabled% \newcommand*{\chapterthumbtextcolor}{white}
%disabled%
%disabled% % New command to set a chapter thumb. I'm using a group at this
%disabled% % command, because I'm changing the temporary dimension \@tempdima
%disabled% \newcommand*{\putchapterthumb}{%
%disabled%   \begingroup
%disabled%     \Large
%disabled%     % calculate the horizontal possition of the right paper border
%disabled%     % (I ignore \hoffset, because I interprete \hoffset moves the page
%disabled%     % at the paper e.g. if you are using cropmarks)
%disabled%     \setlength{\@tempdima}{\@oddheadshift}% (internal from scrpage2)
%disabled%     \setlength{\@tempdima}{-\@tempdima}%
%disabled%     \addtolength{\@tempdima}{\paperwidth}%
%disabled%     \addtolength{\@tempdima}{-\oddsidemargin}%
%disabled%     \addtolength{\@tempdima}{-1in}%
%disabled%     % putting the thumbs should not change the horizontal
%disabled%     % possition
%disabled%     \rlap{%
%disabled%       % move to the calculated horizontal possition
%disabled%       \hspace*{\@tempdima}%
%disabled%       % putting the thumbs should not change the vertical
%disabled%       % possition
%disabled%       \vbox to 0pt{%
%disabled%         % calculate the vertical possition of the thumbs (I ignore
%disabled%         % \voffset for the same reasons told above)
%disabled%         \setlength{\@tempdima}{\chapterthumbwidth}%
%disabled%         \multiply\@tempdima by\value{chapter}%
%disabled%         \addtolength{\@tempdima}{-\chapterthumbwidth}%
%disabled%         \addtolength{\@tempdima}{-\baselineskip}%
%disabled%         % move to the calculated vertical possition
%disabled%         \vspace*{\@tempdima}%
%disabled%         % put the thumbs left so the current horizontal possition
%disabled%         \llap{%
%disabled%           % and rotate them
%disabled%           \rotatebox{90}{\colorbox{\chapterthumbboxcolor}{%
%disabled%               \parbox[c][\chapterthumbheight][c]{\chapterthumbwidth}{%
%disabled%                 \centering
%disabled%                 \textcolor{\chapterthumbtextcolor}{%
%disabled%                   \strut\thechapter}\\
%disabled%               }%
%disabled%             }%
%disabled%           }%
%disabled%         }%
%disabled%         % avoid overfull \vbox messages
%disabled%         \vss
%disabled%       }%
%disabled%     }%
%disabled%   \endgroup
%disabled% }
%disabled%
%disabled% % New command, which works like \lohead but also puts the thumbs (you
%disabled% % cannot use \ihead with this definition but you may change this, if
%disabled% % you use more internal scrpage2 commands)
%disabled% \newcommand*{\loheadwithchapterthumbs}[2][]{%
%disabled%   \lohead[\putchapterthumb#1]{\putchapterthumb#2}%
%disabled% }
%disabled%
%disabled% % initial use
%disabled% \loheadwithchapterthumbs{}
%disabled% \pagestyle{scrheadings}
%disabled%
%disabled% }
%disabled% \makeatother

%%%% END
%%% Local Variables:
%%% mode: latex
%%% mode: auto-fill
%%% mode: flyspell
%%% eval: (ispell-change-dictionary "en_US")
%%% TeX-master: "../main"
%%% End:
%% vim:foldmethod=expr
%% vim:fde=getline(v\:lnum)=~'^%%%%'?0\:getline(v\:lnum)=~'^%doc.*\ .\\%(sub\\)\\?section{.\\+'?'>1'\:'1':
# command
%doc% in \texttt{main.tex}.  For standard usage it is recommended to stay with the
%doc% default settings.
%doc%
%doc%
%% ========================================================================

%doc%
%doc% Some basic microtypographic settings are provided by the
%doc% \texttt{microtype}
%doc% package\footnote{\url{http://ctan.org/pkg/microtype}}. This template
%doc% uses the rather conservative package parameters: \texttt{protrusion=true,factor=900}.
\usepackage[protrusion=true,factor=900]{microtype}

%doc%
%doc% \subsection{French spacing}
%doc%
%doc% \paragraph{Why?} see~\textcite[p.\,28, p.\,30]{Bringhurst1993}: `2.1.4 Use a single word space between sentences.'
%doc%
%doc% \paragraph{How?} see~\textcite[p.\,185]{Eijkhout2008}:\\
%doc% \verb#\frenchspacing  %% Macro to switch off extra space after punctuation.# \\
\frenchspacing  %% Macro to switch off extra space after punctuation.
%doc%
%doc% Note: This setting might be default for \myacro{KOMA} script.
%doc%


%doc%
%doc% \subsection{Font}
%doc%
%doc% This template is using the Palatino font (package \texttt{mathpazo}) which results
%doc% in a legible document and matching mathematical fonts for printout.
%doc%
%doc% It is highly recommended that you either stick to the Palatino font or use the
%doc% \LaTeX{} default fonts (by removing the package \texttt{mathpazo}).
%doc%
%doc% Chosing different fonts is not
%doc% an easy task. Please leave this to people with good knowledge on this subject.
%doc%
%doc% One valid reason to change the default fonts is when your document is mainly
%doc% read on a computer screen. In this case it is recommended to switch to a font
%doc% \textsf{which is sans-serif like this}. This template contains several alternative
%doc% font packages which can be activated in this file.
%doc%

% for changing the default font, please go to the next subsection!

%doc%
%doc% \subsection{Text figures}
%doc%
%doc% \ldots also called old style numbers such as 0123456789.
%doc% (German: \enquote{Mediäval\-ziffern\footnote{\url{https://secure.wikimedia.org/wikibooks/de/wiki/LaTeX-W\%C3\%B6rterbuch:\_Medi\%C3\%A4valziffern}}})
%doc% \paragraph{Why?} see~\textcite[p.\,44f]{Bringhurst1993}:
%doc% \begin{quote}
%doc% `3.2.1 If the font includes both text figures and titling figures, use
%doc%  titling figures only with full caps, and text figures in all other
%doc%  circumstances.'
%doc% \end{quote}
%doc%
%doc% \paragraph{How?}
%doc% Quoted from Wikibooks\footnote{\url{https://secure.wikimedia.org/wikibooks/en/wiki/LaTeX/Formatting\#Text\_figures\_.28.22old\_style.22\_numerals.29}}:
%doc% \begin{quote}
%doc% Some fonts do not have text figures built in; the textcomp package attempts to
%doc% remedy this by effectively generating text figures from the currently-selected
%doc% font. Put \verb#\usepackage{textcomp}# in your preamble. textcomp also allows you to
%doc% use decimal points, properly formatted dollar signs, etc. within
%doc% \verb#\oldstylenums{}#.
%doc% \end{quote}
%doc% \ldots but proposed \LaTeX{} method does not work out well. Instead use:\\
%doc% \verb#\usepackage{hfoldsty}#  (enables text figures using additional font) or \\
%doc% \verb#\usepackage[sc,osf]{mathpazo}# (switches to Palatino font with small caps and old style figures enabled).
%doc%
%\usepackage{hfoldsty}  %% enables text figures using additional font
%% ... OR use ...
\usepackage[sc,osf]{mathpazo} %% switches to Palatino with small caps and old style figures

%% Font selection from:
%%     http://www.matthiaspospiech.de/latex/vorlagen/allgemein/preambel/fonts/
%% use following lines *instead* of the mathpazo package above:
%% ===== Serif =========================================================
%% for Computer Modern (LaTeX default font), simply remove the mathpazo above
%\usepackage{charter}\linespread{1.05} %% Charter
%\usepackage{bookman}                  %% Bookman (laedt Avant Garde !!)
%\usepackage{newcent}                  %% New Century Schoolbook (laedt Avant Garde !!)
%% ===== Sans Serif ====================================================
%\renewcommand{\familydefault}{\sfdefault}  %% this one in *combination* with the default mathpazo package
%\usepackage{cmbright}                  %% CM-Bright (eigntlich eine Familie)
%\usepackage{tpslifonts}                %% tpslifonts % Font for Slides


%doc%
%doc% \subsection{\texttt{myacro} --- Abbrevations using \textsc{small caps}}\myinteresting
%doc% \label{sec:myacro}
%doc%
%doc% \paragraph{Why?} see~\textcite[p.\,45f]{Bringhurst1993}: `3.2.2 For abbrevations and
%doc% acronyms in the midst of normal text, use spaced small caps.'
%doc%
%doc% \paragraph{How?} Using the predefined macro \verb#\myacro{}# for things like
%doc% \myacro{UNO} or \myacro{UNESCO} using \verb#\myacro{UNO}# or \verb#\myacro{UNESCO}#.
%doc%
\DeclareRobustCommand{\myacro}[1]{\textsc{\lowercase{#1}}} %%  abbrevations using small caps


%doc%
%doc% \subsection{Colorized headings and links}
%doc%
%doc% This document template is able to generate an output that uses colorized
%doc% headings, captions, page numbers, and links. The color named `DispositionColor'
%doc% used in this document is defined near the definition of package \texttt{color}
%doc% in the preamble (see section~\ref{subsec:miscpackages}). The changes required
%doc% for headings, page numbers, and captions are defined here.
%doc%
%doc% Settings for colored links are handled by the definitions of the
%doc% \texttt{hyperref} package (see section~\ref{sec:pdf}).
%doc%
\setheadsepline{.4pt}[\color{DispositionColor}]
\renewcommand{\headfont}{\normalfont\sffamily\color{DispositionColor}}
\renewcommand{\pnumfont}{\normalfont\sffamily\color{DispositionColor}}
\addtokomafont{disposition}{\color{DispositionColor}}
\addtokomafont{caption}{\color{DispositionColor}\footnotesize}
\addtokomafont{captionlabel}{\color{DispositionColor}}

%doc%
%doc% \subsection{No figures or tables below footnotes}
%doc%
%doc% \LaTeX{} places floating environments below footnotes if \texttt{b}
%doc% (bottom) is used as (default) placement algorithm. This is certainly
%doc% not appealing for most people and is deactivated in this template by
%doc% using the package \texttt{footmisc} with its option \texttt{bottom}.
%doc%
%% see also: http://www.komascript.de/node/858 (German description)
\usepackage[bottom]{footmisc}

%doc%
%doc% \subsection{Spacings of list environments}
%doc%
%doc% By default, \LaTeX{} is using vertical spaces between items of enumerate,
%doc% itemize and description environments. This is fine for multi-line items.
%doc% Many times, the user does just write single-line items where the larger
%doc% vertical space is inappropriate. The \href{http://ctan.org/pkg/enumitem}{enumitem}
%doc% package provides replacements for the pre-defined list environments and
%doc% offers many options to modify their appearances.
%doc% This template is using the package option for \texttt{noitemsep} which
%doc% mimimizes the vertical space between list items.
%doc%
\usepackage{enumitem}
%\setlist{noitemsep}   %% kills the space between items

%doc%
%doc% \subsection{\texttt{csquotes} --- Correct quotation marks}\myinteresting
%doc% \label{sub:csquotes}
%doc%
%doc% \emph{Never} use quotation marks found on your keyboard.
%doc% They end up in strange characters or false looking quotation marks.
%doc%
%doc% In \LaTeX{} you are able to use typographically correct quotation marks. The package
%doc% \href{http://www.ctan.org/pkg/csquotes}{\texttt{csquotes}} offers you with
%doc% \verb#\enquote{foobar}# a command to get correct quotation marks around \enquote{foobar}.
%doc% Please do check the package options in order to modify
%doc% its settings according to the language used\footnote{most of the time in
%doc% combination with the language set in the options of the \texttt{babel} package}.
%doc%
%doc% \href{http://www.ctan.org/pkg/csquotes}{\texttt{csquotes}} is also recommended
%doc% by \texttt{biblatex} (see Section~\ref{sec:references}).
\usepackage[babel=true,strict=true,english=american,german=guillemets]{csquotes}

%doc%
%doc% \subsection{Line spread}
%doc%
%doc% If you have to enlarge the distance between two lines of text, you can
%doc% increase it using the \texttt{\mylinespread} command in \texttt{main.tex}. By default, it is
%doc% deactivated (set to 100~percent). Modify only with caution since it influences the
%doc% page layout and could lead to ugly looking documents.
\linespread{\mylinespread}

%doc%
%doc% \subsection{Optional: Lines above and below the chapter head}
%doc%
%doc% This is not quite something typographic but rather a matter of taste.
%doc% \myacro{KOMA} Script offers \href{http://www.komascript.de/node/24}{a method to
%doc% add lines above and below chapter head} which is disabled by
%doc% default. If you want to enable this feature, remove corresponding
%doc% comment characters from the settings.
%doc%
%% Source: http://www.komascript.de/node/24
%disabled% %% 1st get a new command
%disabled% \newcommand*{\ORIGchapterheadstartvskip}{}%
%disabled% %% 2nd save the original definition to the new command
%disabled% \let\ORIGchapterheadstartvskip=\chapterheadstartvskip
%disabled% %% 3rd redefine the command using the saved original command
%disabled% \renewcommand*{\chapterheadstartvskip}{%
%disabled%   \ORIGchapterheadstartvskip
%disabled%   {%
%disabled%     \setlength{\parskip}{0pt}%
%disabled%     \noindent\color{DispositionColor}\rule[.3\baselineskip]{\linewidth}{1pt}\par
%disabled%   }%
%disabled% }
%disabled% %% see above
%disabled% \newcommand*{\ORIGchapterheadendvskip}{}%
%disabled% \let\ORIGchapterheadendvskip=\chapterheadendvskip
%disabled% \renewcommand*{\chapterheadendvskip}{%
%disabled%   {%
%disabled%     \setlength{\parskip}{0pt}%
%disabled%     \noindent\color{DispositionColor}\rule[.3\baselineskip]{\linewidth}{1pt}\par
%disabled%   }%
%disabled%   \ORIGchapterheadendvskip
%disabled% }

%doc%
%doc% \subsection{Optional: Chapter thumbs}
%doc%
%doc% This is not quite something typographic but rather a matter of taste.
%doc% \myacro{KOMA} Script offers \href{http://www.komascript.de/chapterthumbs-example}{a method to
%doc% add chapter thumbs} (in combination with the package \texttt{scrpage2}) which is disabled by
%doc% default. If you want to enable this feature, remove corresponding
%doc% comment characters from the settings.
%doc%
%disabled% \makeatletter
%disabled% % Safty first
%disabled% \@ifundefined{chapter}{\let\chapter\undefined
%disabled%   \chapter must be defined to use chapter thumbs!}{%
%disabled%
%disabled% % Two new commands for the width and height of the boxes with the
%disabled% % chapter number at the thumbs (use of commands instead of lengths
%disabled% % for sparing registers)
%disabled% \newcommand*{\chapterthumbwidth}{2em}
%disabled% \newcommand*{\chapterthumbheight}{1em}
%disabled%
%disabled% % Two new commands for the colors of the box background and the
%disabled% % chapter numbers of the thumbs
%disabled% \newcommand*{\chapterthumbboxcolor}{black}
%disabled% \newcommand*{\chapterthumbtextcolor}{white}
%disabled%
%disabled% % New command to set a chapter thumb. I'm using a group at this
%disabled% % command, because I'm changing the temporary dimension \@tempdima
%disabled% \newcommand*{\putchapterthumb}{%
%disabled%   \begingroup
%disabled%     \Large
%disabled%     % calculate the horizontal possition of the right paper border
%disabled%     % (I ignore \hoffset, because I interprete \hoffset moves the page
%disabled%     % at the paper e.g. if you are using cropmarks)
%disabled%     \setlength{\@tempdima}{\@oddheadshift}% (internal from scrpage2)
%disabled%     \setlength{\@tempdima}{-\@tempdima}%
%disabled%     \addtolength{\@tempdima}{\paperwidth}%
%disabled%     \addtolength{\@tempdima}{-\oddsidemargin}%
%disabled%     \addtolength{\@tempdima}{-1in}%
%disabled%     % putting the thumbs should not change the horizontal
%disabled%     % possition
%disabled%     \rlap{%
%disabled%       % move to the calculated horizontal possition
%disabled%       \hspace*{\@tempdima}%
%disabled%       % putting the thumbs should not change the vertical
%disabled%       % possition
%disabled%       \vbox to 0pt{%
%disabled%         % calculate the vertical possition of the thumbs (I ignore
%disabled%         % \voffset for the same reasons told above)
%disabled%         \setlength{\@tempdima}{\chapterthumbwidth}%
%disabled%         \multiply\@tempdima by\value{chapter}%
%disabled%         \addtolength{\@tempdima}{-\chapterthumbwidth}%
%disabled%         \addtolength{\@tempdima}{-\baselineskip}%
%disabled%         % move to the calculated vertical possition
%disabled%         \vspace*{\@tempdima}%
%disabled%         % put the thumbs left so the current horizontal possition
%disabled%         \llap{%
%disabled%           % and rotate them
%disabled%           \rotatebox{90}{\colorbox{\chapterthumbboxcolor}{%
%disabled%               \parbox[c][\chapterthumbheight][c]{\chapterthumbwidth}{%
%disabled%                 \centering
%disabled%                 \textcolor{\chapterthumbtextcolor}{%
%disabled%                   \strut\thechapter}\\
%disabled%               }%
%disabled%             }%
%disabled%           }%
%disabled%         }%
%disabled%         % avoid overfull \vbox messages
%disabled%         \vss
%disabled%       }%
%disabled%     }%
%disabled%   \endgroup
%disabled% }
%disabled%
%disabled% % New command, which works like \lohead but also puts the thumbs (you
%disabled% % cannot use \ihead with this definition but you may change this, if
%disabled% % you use more internal scrpage2 commands)
%disabled% \newcommand*{\loheadwithchapterthumbs}[2][]{%
%disabled%   \lohead[\putchapterthumb#1]{\putchapterthumb#2}%
%disabled% }
%disabled%
%disabled% % initial use
%disabled% \loheadwithchapterthumbs{}
%disabled% \pagestyle{scrheadings}
%disabled%
%disabled% }
%disabled% \makeatother

%%%% END
%%% Local Variables:
%%% mode: latex
%%% mode: auto-fill
%%% mode: flyspell
%%% eval: (ispell-change-dictionary "en_US")
%%% TeX-master: "../main"
%%% End:
%% vim:foldmethod=expr
%% vim:fde=getline(v\:lnum)=~'^%%%%'?0\:getline(v\:lnum)=~'^%doc.*\ .\\%(sub\\)\\?section{.\\+'?'>1'\:'1':

%doc% %%%% Time-stamp: <2014-03-23 13:40:59 vk>
%%%% === Disclaimer: =======================================================
%% created by
%%
%%      Karl Voit
%%
%% using GNU/Linux, GNU Emacs & LaTeX 2e
%%

%doc%
%doc% \section{\texttt{pdf\_settings.tex} --- Settings related to PDF output}
%doc% \label{sec:pdf}
%doc% 
%doc% The file \verb#template/pdf_settings.tex# basically contains the definitions for
%doc% the \href{http://tug.org/applications/hyperref/}{\texttt{hyperref} package}
%doc% including the
%doc% \href{http://www.ctan.org/tex-archive/macros/latex/required/graphics/}{\texttt{graphicx}
%doc% package}. Since these settings should be the last things of any \LaTeX{}
%doc% preamble, they got their own \TeX{} file which is included in \texttt{main.tex}.
%doc% 
%doc% \paragraph{What should I do with this file?} The settings in this file are
%doc% important for \myacro{PDF} output and including graphics. Do not exclude the
%doc% related \texttt{input} command in \texttt{main.tex}. But you might want to
%doc% modify some settings after you read the
%doc% \href{http://tug.org/applications/hyperref/}{documentation of the \texttt{hyperref} package}.
%doc% 


%% Fix positioning of images in PDF viewers. (disabled by
%% default; see https://github.com/novoid/LaTeX-KOMA-template/issues/4
%% for more information) 
%% I do not have time to read about possible side-effect of this
%% package for now.
% \usepackage[hypcap]{caption}

%% Declarations of hyperref should be the last definitions of the preamble:
%% FIXXME: black-and-white-version for printing!

\pdfcompresslevel=9

\usepackage[%
unicode=true, % loads with unicode support
%a4paper=true, %
pdftex=true, %
backref, %
pagebackref=false, % creates backward references too
bookmarks=false, %
bookmarksopen=false, % when starting with AcrobatReader, the Bookmarkcolumn is opened
pdfpagemode=None,% None, UseOutlines, UseThumbs, FullScreen
plainpages=false, % correct, if pdflatex complains: ``destination with same identifier already exists''
%% colors: https://secure.wikimedia.org/wikibooks/en/wiki/LaTeX/Colors
urlcolor=DispositionColor, %%
linkcolor=DispositionColor, %%
pagecolor=DispositionColor, %%
citecolor=DispositionColor, %%
anchorcolor=DispositionColor, %%
colorlinks=\mycolorlinks, % turn on/off colored links (on: better for
                          % on-screen reading; off: better for printout versions)
]{hyperref}

%% all strings need to be loaded after hyperref was loaded with unicode support
%% if not the field is garbled in the output for characters like ČŽĆŠĐ
\hypersetup{
pdftitle={\mytitle}, %
pdfauthor={\myauthor}, %
pdfsubject={\mysubject}, %
pdfcreator={Accomplished with: pdfLaTeX, biber, and hyperref-package. No animals, MS-EULA or BSA-rules were harmed.},
pdfproducer={\myauthor},
pdfkeywords={\mykeywords}
}

%\DeclareGraphicsExtensions{.pdf}

%%%% END
%%% Local Variables:
%%% TeX-master: "../main"
%%% mode: latex
%%% mode: auto-fill
%%% mode: flyspell
%%% eval: (ispell-change-dictionary "en_US")
%%% End:
%% vim:foldmethod=expr
%% vim:fde=getline(v\:lnum)=~'^%%%%'?0\:getline(v\:lnum)=~'^%doc.*\ .\\%(sub\\)\\?section{.\\+'?'>1'\:'1':

%doc%
%doc% \begin{document}
%doc% %% title page %%
%doc% \title{\mytitle}\subtitle{\mysubject}
%doc% \author{\myauthor}
%doc% \date{\today}
%doc%
%doc% \maketitle\newpage
%doc%
%doc% \tableofcontents\newpage
%doc% %%---------------------------------------%%

%doc%
%doc% \section{How to use this \LaTeX{} document template}
%doc%
%doc% This \LaTeX{} document template from
%doc% \myLaT\footnote{\url{http://LaTeX.TUGraz.at}} is based on \myacro{KOMA}
%doc% script\footnote{\url{http://komascript.de/}}. You don't need any
%doc% special \myacro{KOMA} knowledge (but it woun't hurt either). It provides an easy to use and
%doc% easy to modify template. All settings are documented and many references to
%doc% additional information sources are given.
%doc%

%doc% In general, there should not be any reason to modify a file in
%doc% the \texttt{template} folder. \emph{All important settings are
%doc% accessible in the main folder, mostly in the \texttt{main.tex}
%doc% file.} This way, it is easy to get what you need and you can update
%doc% the template independent of the content of the document.
%doc%
%doc% \newcommand{\myimportant}{%% mark important chapters
%doc%   \marginpar{\vspace{-1em}\rightpointleft}
%doc% }
%doc% \newcommand{\myinteresting}{\marginpar{\vspace{-2em}\PencilLeftDown}}

%doc%
%doc% The \emph{absolute minimum you should read} is listed below and
%doc% marked with the hand symbol:\myimportant
%doc% \begin{itemize}
%doc% \item Section~\ref{sec:modifytemplate}: basic configuration of this template.
%doc% \item Section~\ref{sec:howtocompile}: how to generate the \myacro{PDF} file
%doc% \item Section~\ref{sec:references}: using biblatex (instead of bibtex)
%doc% \end{itemize}
%doc%
%doc% In order to get a perfect resulting document and to get an
%doc% exciting experience with this template, you should definitely consider reading
%doc% following sections which are also marked with the pencil symbol:\myinteresting
%doc% \begin{itemize}
%doc% \item Section~\ref{sec:extending-template}: extend the template with
%doc%   your own usepackages, newcommands, and so forth
%doc% \item Section~\ref{sec:mycommands}: pre-defined commands to make your life easier (e.g., including graphics)
%doc% \item Section~\ref{sec:myacro}: how to do acronyms (like \myacro{ACME}) beautifully
%doc% \item Section~\ref{sub:csquotes}: how to \enquote{quote} text and use parentheses correctly
%doc% \end{itemize}
%doc%
%doc% The other sections describe all other settings for the sake of completeness. This is
%doc% interesting for learning more about \LaTeX{} and modifying this template to a higher level of detail.

%doc%
%doc% \newpage
%doc% \subsection{Six Steps to Customize Your Document}\myimportant
%doc% \label{sec:modifytemplate}
%doc%
%doc% This template is optimized to get to the first draft of your thesis
%doc% very quickly. Follow these instructions and you get most of your
%doc% customizing done in a few minutes:
%doc%
%doc% \newcommand{\myfile}[1]{\texttt{\href{file:#1}{#1}}}
%doc%
%doc% \begin{enumerate}
%doc% \item Modify settings in \texttt{main.tex} to meet your requirements:
%doc%   \begin{itemize}
%doc%   \item Basic settings
%doc%     \begin{itemize}
%doc%     \item Paper size, languages, font size, citation style,
%doc%           title page, and so forth
%doc%     \end{itemize}
%doc%   \item Document metadata
%doc%     \begin{itemize}
%doc%     \item Preferences like \verb+myauthor+, \verb+mytitle+, and so forth
%doc%     \end{itemize}
%doc%   \end{itemize}
%doc% \item Replace \myfile{figures/institution.pdf} with the logo of
%doc% your institution in either \myacro{PDF} or \myacro{PNG}
%doc% format.\footnote{Avoid \myacro{JPEG} format for
%doc% computer-generated (pixcel-oriented) graphics like logos or
%doc% screenshots in general. The \myacro{JEPG} format is for
%doc% photographs \emph{only}.}
%doc% \item Further down in \myfile{main.tex}:
%doc%   \begin{itemize}
%doc%   \item Create your desired structure for the chapters
%doc%         (\verb+\chapter{Introduction}
\label{cha:introduction}


\section{Motivation}
\label{sec:motivation}

Plenty of problems that arise in the real world can be solved by first abstracting them in a more general form and solve them by using already established and well-known tools afterwards.
One very famous example for this process is attributed to the Swiss mathematician Leonhard Euler.
The city Königsberg (now called Kaliningrad) was split into four parts in the 18\textsuperscript{th} century, due to the pathway of the river Pregel.
These four areas of the city were connected by seven bridges.
The residents of Königsberg had an ongoing challenge to find a way through the city that crosses each bridge exactly once and ends once more at the starting point of the tour.
This is known as the Königsberg bridge problem~\cite{Paoletti2011, Cook2012}.
Euler tackled this challenge by eliminating irrelevant details (e.g., the length of the brides, or the size of the size of the areas) and abstracting the problem to its essence.
The regions of the city became points and the bridges between two areas became line segments that connect the corresponding points.
This abstracted topological view on the problem allowed Euler to solve the Königsberg problem and show that in fact no such tour through the city (also known as an Eulerian cycle) exists.
For it to be possible requires the number of bridges that are accessible in each region to be even, which is not the case for the Königsberg problem, where the number is odd in each region.

This particular way of abstracting real world objects and their connections into points and lines and applying mathematical methods and reasoning to it laid the foundation for the mathematical field of graph theory.
It is used today in countless applications, from finding efficient ways to manufacture circuit boards~\cite{Cook2012} to calculating fast routes for the data packets that are sent over the internet~\cite{Wang1999}.
Graph theory also spawned the field of networks science~\cite{Newman2010}, which deals with large real world systems that can be modeled using graphs.
For instance, large infrastructure networks, such as power grids can be represented by graphs.
These graphs can relate parts of the infrastructure, like power stations or transformers, that are connected to each other via power lines~\cite{Watts1998}.
Another field that benefits from this as well is sociology, and in particular social network analysis, which investigates the complex behavior of people and groups in the context of social interactions~\cite{Newman2010}.
Sociological studies are often performed with small sample sizes, since the collection of the required data is time consuming and usually done by hand using methods like questionnaires, interviews, or by simply observing the people that are part of the study~\cite{Wasserman1994}.
However, ever since the emerge of the web, people can interact with each other more easily.
Websites like online social networks (e.g., Facebook, Twitter, Reddit,\ldots), or collaboration forums (e.g., StackOverflow) seem to be very popular.
Most of these websites are ranked on top positions on Alexas' list of top 500 visited websites globally~\cite{Alexa2017}.
\citet{Wellman2001} showed that these websites (and the internet in general) does not increase or decrease the social capital of people (i.e., their relationships with friends and family or their commitment to participate in organizations), but supplements it by providing easier ways to organize and plan real-world activities in an online setting.

This availability of large amounts of data that are generated on these websites can be beneficial in the context of social network analysis.
For example, StackExchange, a website that maintains a variety of communities in which users can ask and answer questions to different topics (e.g., StackOverflow for programming related questions), provides an easy access\footnote{\url{https://data.stackexchange.com/}} to the user data.
This data is used in many studies on a wide range of topics (e.g.,~\cite{Danescu2013,Walk2016, Hasani-Mavriqi2016}).
A large mobile phone calls (MPC) data set is used in a variety of papers~\cite{Onnela2007, Karsai2014, Murase2015, Laurent2015} as well.
It was collected by an European mobile phone provider with approximately 20\% market share and consists of over 630 million logged phone calls between more than six million people.
Another interesting data set was used in the study by \citet{Sekara2016}, in which they propose a framework to describe gatherings of people and their (temporal) properties.
The data set consists of data from various sources for 1,000 people that was collected over a period of 36 months in short intervals (i.e., with a five minute resolution).
The data set contains information about the phone calls, text messages, social media activity, geolocation, and the proximity to other study participants for every subject.

All of these large data sets share another crucial feature: they all include temporal information.
Every post, tweet, phone call, or text message has a timestamp attached to it.
This allows to pin these activities to users at specific points in time and to infer a chronological order between them.
However, not all studies include this additional information into their work.
The reasons for this can vary.
For instance, some use-cases only require quantitative information (i.e., how often something happens between two objects) and not exactly when it did  (e.g.,~\cite{Kumpula2007, Bagler2008}).
This corresponds to the elimination of irrelevant details in the abstraction of a problem.
Another reason is that graphs by them self are not able to include the temporal information, since they only represent static relationships between objects.
However, there is an extension that provides the possibility to integrate the time information when required.
This type of graphs are called temporal or time-varying networks~\cite{Holme2012, Holme2015} and extent graphs in a way that relationships between objects become time depended.
This basically means that the connection between objects is only present at certain points in time, which leads to a more realistic but also more complex abstraction.

An area that definitively requires the incorporation of time information is the modeling of human behavioral patterns.
It has been shown that activities performed by people (e.g., the writing of e-mails, text messages, tweets,\ldots) are not randomly distributed in time, but follow certain patterns instead~\cite{Barabasi2005}.
Human activity can usually be described as bursty.
For instance, it is evident that people tend to write multiple e-mails in a relatively short period of time.
This high activity phase is then generally followed by longer periods of inactivity, in which, for example, no e-mails are written at all.
The best way to describe these patterns is by using a probability distribution of the times between two consecutive activities (i.e., the inter-event time distribution).
The distribution is characterized by its high probabilities for short inter-event times and its long tail that allows for the longer phases with no activity, which can be modeled using a power-law distribution of the form \( p(\tau) \sim \tau^{-\gamma} \).

Inter-event time distributions are also relevant in the context of time-varying networks.
\citet{Lambiotte2013} study in their work the effects of the inter-event time distribution of the link activations on dynamic spreading processes in temporal networks.
The framework proposed by \citet{Perra2012a} shifts the focus from the activation of the connections to the activations of the objects.
It is based on the simple idea that each entity in the network can become active based on some inherent activity potential and subsequently connect to others in each time step.
Therefore, it can, for example, be used to model the activity of users in an entire social network.
However, this model has some disadvantages and due to its simplicity and unrealistic assumptions.
For example, it is not able to reproduce inter-event time distributions of the activations and topological properties of the network that have the same characteristics as observed in so many real-world networks.
Nevertheless, there are a lot of extensions to this basic framework that address these issues (e.g.,~\cite{Laurent2015, Moinet2015, Moinet2016}).
Overall initiated this activity-driven temporal network model a wide range of followup work in many different areas.

It was the foundation that started this thesis as well.
All models that originated from the original paper by \citet{Perra2012a} so far are based on the idea that entities can only become active due to their intrinsic activity potential.
 However, in the real world people are often heavily influenced by their peers and friends~\cite{Walk2016}.
They are not doing things solely because of their own determination to do so, but also because their friends and the associated peer pressure.
Therefore, in the context of social and collaboration networks could more active user help to motivate their peers to become active as well.
The aim of this work is it to define a model based on the activity-driven time-varying network framework that is able to include peer influence effects in the activation process of entities and to study the implication of this mechanism of the network in general.


%% ========================================================================
%% ========================================================================


\section{Outline}
\label{sec:outline}

The content of this thesis is structured as follows.
\Cref{cha:related-work} starts with some basic graph-theoretic definitions (\cref{sec:graph-theory-basics}) and a detailed introduction into the topics of social and temporal networks (\cref{sec:social-networks} and \cref{sec:time-varying-networks} respectively).
Additionally, an overview of some important generative network models and their properties is given in \cref{sec:network-models}.
 \Cref{sec:user-activity-models} discusses related work in the context of user behavioral models.
 It provides a possible explanation for the origin of bursty human behavior with the queuing model and provides an overview of activity models that are based on time-varying networks, with focus on work based on the activity-driven temporal network framework.
The last section (\cref{sec:peer-influence}) of this chapter is related to peer influence and its ramifications and applications in different fields.

In \cref{cha:model} the proposed time-varying peer influence network model is discussed in great detail.
First, the model and the ideas on which it is based are outlined in \cref{sec:base-model}.
The extension that incorporates peer influence effects into the base model and the idea behind it is described in \cref{sec:peer-influence-model}.
\Cref{cha:results} contains the evaluation and analysis of the proposed model on synthetic networks.

The last chapter of this document (\cref{cha:conclusion}) includes a summary of the archived results and a conclusion.
Furthermore, different applications and possible extensions of the peer influence model are discussed in the last section of this thesis.
+, \verb+\include{evaluation}+, \ldots)
%doc%   \end{itemize}
%doc% \item Create the \TeX{} files and fill your content into these files you defined in the previous step.
%doc% \item Optionally: Modify \myfile{colophon.tex} to meet your situation.
%doc%   \begin{itemize}
%doc%   \item Please spend a couple of minutes and think about putting your work
%doc%         under an open license\footnote{\url{https://creativecommons.org/licenses/}}
%doc%         in order to follow the spirit of Open Science\footnote{\url{https://en.wikipedia.org/wiki/Open_science}}.
%doc%   \end{itemize}
%doc% \item In case you are using \myacro{GNU} make\footnote{If you
%doc%       don't know, what \myacro{GNU} make is, you are not using it (yet).}:
%doc%       Put your desired \myacro{PDF} file name in the second line of file
%doc%    \myfile{Makefile}
%doc%    \begin{itemize}
%doc%    \item replace \enquote{Projectname} with your filename
%doc%    \item do not use any file extension like \texttt{.tex} or \texttt{.pdf}
%doc%    \end{itemize}
%doc% \end{enumerate}
%doc%
%doc%

%doc%
%doc% \subsection{License}\myimportant
%doc% \label{sec:license}
%doc%
%doc% This template is licensed under a Creative Commons Attribution-ShareAlike 3.0 Unported (CC BY-SA 3.0)
%doc%         license\footnote{\url{https://creativecommons.org/licenses/by-sa/3.0/}}:
%doc%     \begin{itemize}
%doc%     \item You can share (to copy, distribute and transmit) this template.
%doc%     \item You can remix (adapt) this template.
%doc%     \item You can make commercial use of the template.
%doc%     \item In case you modify this template and share the derived
%doc%           template: You must attribute the template such that you do not
%doc%           remove (co-)authorship of Karl Voit and you must not remove
%doc%           the URL to the original repository on
%doc%           github\footnote{\url{https://github.com/novoid/LaTeX-KOMA-template}}.
%doc%     \item If you alter, transform, or build a new template upon
%doc%           this template, you may distribute the resulting
%doc%           template only under the same or similar license to this one.
%doc%     \item There are \emph{no restrictions} of any kind, however, related to the
%doc%           resulting (PDF) document!
%doc%     \item You may remove the colophon (but it's not recommended).
%doc%     \end{itemize}


%doc%
%doc%
%doc% \subsection{How to compile this document}\myimportant
%doc% \label{sec:howtocompile}
%doc%
%doc% I assume that compiling \LaTeX{} documents within your software
%doc% environment is something you have already learned. This template is
%doc% almost like any other \LaTeX{} document except it uses
%doc% state-of-the-art tools for generating things like the list of
%doc% references using biblatex/biber (see
%doc% Section~\ref{sec:references} for details). Unfortunately, some \LaTeX{} editors
%doc% do not support this much better way of working with bibliography
%doc% references yet. This section describes how to compile this template.
%doc%
%doc% \subsubsection{Compiling Using a \LaTeX{} Editor}
%doc%
%doc% Please do select \myfile{main.tex} as the \enquote{main project file} or make
%doc% sure to compile/run only \myfile{main.tex} (and not \myfile{introduction.tex}
%doc% or other \TeX{} files of this template).
%doc%
%doc% Choose \texttt{biber} for generating the references. Modern LaTeX{}
%doc% environments offer this option. Older tools might not be that up to
%doc% date yet.
%doc%

%doc% \subsubsection{Activating \texttt{biber} in the \LaTeX{} editor TeXworks}
%doc% \label{sec:biberTeXworks}
%doc%
%doc% The \href{https://www.tug.org/texworks/}{TeXworks} editor is a very
%doc% basic (but fine) \LaTeX{} editor to start with. It is included in
%doc% \href{http://miktex.org/}{MiKTeX} and
%doc% \href{http://miktex.org/portable}{MiKTeX portable} and supports
%doc% \href{https://en.wikipedia.org/wiki/Syntax_highlighting}{syntax
%doc%   highlighting} and
%doc% \href{http://itexmac.sourceforge.net/SyncTeX.html}{SyncTeX} to
%doc% synchronize \myacro{PDF} output and \LaTeX{} source code.
%doc%
%doc% Unfortunately, TeXworks shipped with MiKTeX does not support compiling
%doc% using \texttt{biber} (biblatex) out of the box. Here is a solution to
%doc% this issue. Go to TeXworks: \texttt{Edit} $\rightarrow$
%doc% \texttt{Preferences~\ldots} $\rightarrow$ \texttt{Typesetting} $\rightarrow$
%doc% \texttt{Processing tools} and add a new entry (using the plus icon):
%doc%
%doc% \begin{tabbing}
%doc%   Arguments: \= foobar  \kill
%doc%   Name:      \> \verb#pdflatex+biber# \\
%doc%   Program:   \> \emph{find the \texttt{template/pdflatex+biber.bat} file from your disk} \\
%doc%   Arguments: \> \verb+$fullname+ \\
%doc%              \> \verb+$basename+
%doc% \end{tabbing}
%doc%
%doc% Activate the \enquote{View PDF after running} option.
%doc%
%doc% Close the preferences dialog and you will now have an additional
%doc% choice in the drop down list for compiling your document. Choose the
%doc% new entry called \verb#pdflatex+biber# and start a happier life with
%doc% \texttt{biber}.
%doc%
%doc% In case your TeXworks has a German user interface, here the key
%doc% aspects in German as well:
%doc%
%doc% \begin{otherlanguage}{ngerman}
%doc%
%doc%   \texttt{Bearbeiten} $\rightarrow$ \texttt{Einstellungen~\ldots} $\rightarrow$
%doc%   \texttt{Textsatz} $\rightarrow$ \texttt{Verarbeitungsprogramme} $\rightarrow$
%doc%   + \emph{(neues Verarbeitungsprogramm)}:
%doc%
%doc% \begin{tabbing}
%doc%   Befehl/Datei: \= foobar  \kill
%doc%     Name: \> pdflatex+biber \\
%doc%     Befehl/Datei: \> \emph{die \texttt{template/pdflatex+biber.bat} im Laufwerk suchen} \\
%doc%     Argumente: \> \verb+$fullname+ \\
%doc%                \> \verb+$basename+
%doc% \end{tabbing}
%doc%
%doc% \enquote{PDF nach Beendigung anzeigen} aktivieren.
%doc%
%doc% \end{otherlanguage}
%doc%

%doc% \subsubsection{Compiling Using \myacro{GNU} make}
%doc%
%doc% With \myacro{GNU}
%doc% make\footnote{\url{https://secure.wikimedia.org/wikipedia/en/wiki/Make\_\%28software\%29}}
%doc% it is just simple as that: \texttt{make pdf}
%doc%
%doc% Several other targets are available. You can check them out by
%doc% executing: \texttt{make help}
%doc%
%doc% In case you are using TeXLive (instead of MiKTeX as I do), you might
%doc% want to modify the line \texttt{PDFLATEX\_CMD = pdflatex} within
%doc% the file \texttt{Makefile} to: \texttt{PDFLATEX\_CMD = pdflatex -synctex=1 -undump=pdflatex}
%doc%
%doc%

%doc% \subsubsection{Compiling in a Text-Shell}
%doc%
%doc% To generate a document using \texttt{Biber}, you can stick to
%doc% following example:
%doc% \begin{verbatim}
%doc% pdflatex main.tex
%doc% biber main
%doc% pdflatex main.tex
%doc% pdflatex main.tex
%doc% \end{verbatim}
%doc% 
%doc% Users of TeXLive with Microsoft Windows might want to try the
%doc% following script\footnote{Thanks to Florian Brucker for provinding
%doc%   this script.} which could be stored as, e.g., \texttt{compile.bat}:
%doc% \begin{verbatim}
%doc% REM call pdflatex using parameters suitable for TeXLive:
%doc% pdflatex.exe  "main.tex"
%doc% REM generate the references metadata for biblatex (using biber):
%doc% biber.exe "main"
%doc% REM call pdflatex twice to compile the references and finalize PDF:
%doc% pdflatex.exe  "main.tex"
%doc% pdflatex.exe -synctex=-1 -interaction=nonstopmode "main.tex"
%doc% \end{verbatim}
%doc% 


%doc%
%doc% \subsection{How to get rid of the template documentation}
%doc%
%doc% Simply remove the files \verb#Template_Documentation.pdf# and
%doc% \verb#Template_Documentation.tex# (if it exists) in the main folder
%doc% of this template.
%doc%
%doc% \subsection{What about modifying or extending the template?}\myinteresting
%doc% \label{sec:extending-template}
%doc%
%doc% This template provides an easy to start \LaTeX{} document template with sound
%doc% default settings. You can modify each setting any time. It is recommended that
%doc% you are familiar with the documentation of the command whose settings you want
%doc% to modify.
%doc%
%doc% It is recommended that for \emph{adding} things to the preambel (newcommands,
%doc% setting variables, defining headers, \dots) you should use the file
%doc% \texttt{main.tex}.
%doc% There are comment lines which help you find the right spot.
%doc% This way you still have the chance to update your \texttt{template}
%doc% folder from the template repository without losing your own added things.
%doc%
%doc% The following sections describe the settings and commands of this template and
%doc% give a short overview of its features.

%doc% \subsection{How to change the title page}
%doc%
%doc% This template comes with a variety of title pages. They are located in
%doc% the folder \texttt{template}. You can switch to a specific title
%doc% page by including the corresponding title page file in the file
%doc% \texttt{main.tex}.
%doc%
%doc% Please note that you may not need to modify any title page document by
%doc% yourself since all relevant information is defined in the file
%doc% \texttt{main.tex}.

%doc%
%doc% \section{\texttt{preamble.tex} --- Main preamble file}
%doc%
%doc% In the file \verb#preamble/preamble.tex# you will find the basic
%doc% definitions related to your document. This template uses the \myacro{KOMA} script
%doc% extension package of \LaTeX{}.
%doc%
%doc% There are comments added to the \verb#\documentclass{}# definitions. Please
%doc% refer to the great documentation of \myacro{KOMA}\footnote{\texttt{scrguide.pdf} for
%doc% German users} for further details.
%doc%
%doc% \paragraph{What should I do with this file?} For standard purposes you might
%doc% use the default values it provides. You must not remove its \texttt{include} command
%doc% in \texttt{main.tex} since it contains important definitions. This file contains
%doc% settings which are documented well and can be modified according to your needs.
%doc% It is recommended that you fully understand each setting you modify in order to
%doc% get a good document result. However, you can set basic values in the
%doc% \texttt{main.tex} file: font size, paper size,
%doc% paragraph separation mode, draft mode, binding correction, and whether
%doc% your document will be a one sided document or you are planning to
%doc% create a document which is printed on both, left side and right side.
%doc%

\documentclass[%
fontsize=\myfontsize,%% size of the main text
paper=\mypapersize,  %% paper format
parskip=\myparskip,  %% vertical space between paragraphs (instead of indenting first par-line)
DIV=calc,            %% calculates a good DIV value for type area; 66 characters/line is great
headinclude=true,    %% is header part of margin space or part of page content?
footinclude=false,   %% is footer part of margin space or part of page content?
open=right,          %% "right" or "left": start new chapter on right or left page
appendixprefix=true, %% adds appendix prefix; only for book-classes with \backmatter
bibliography=totoc,  %% adds the bibliography to table of contents (without number)
draft=\mydraft,      %% if true: included graphics are omitted and black boxes
                     %%          mark overfull boxes in margin space
BCOR=\myBCOR,        %% binding correction (depends on how you bind
                     %% the resulting printout.
\mylaterality        %% oneside: document is not printed on left and right sides, only right side
                     %% twoside: document is printed on left and right sides
]{scrbook}  %% article class of KOMA: "scrartcl", "scrreprt", or "scrbook".
            %% CAUTION: If documentclass will be changed, *many* other things
            %%          change as well like heading structure, ...



% FIXXME: adopting class usage:
% from scrbook -> scrartcl OR scrreport:
% - remove appendixprefix from class options
% - remove \frontmatter \mainmatter \backmatter \appendix from main.tex

% FIXXME: adopting language:
% add or modify language parameter of package »babel« and use language switches described in babel-documentation

%doc%
%doc% \subsection{\texttt{inputenc}: UTF8 as input charset}
%doc%
%doc% You are able and should use \myacro{UTF8} character settings for writing these \TeX{}-files.
%doc%
%\usepackage{ucs}             %% UTF8 as input characters; UCS incompatible to biblatex
\usepackage[utf8]{inputenc} %% UTF8 as input characters
%% Source: http://latex.tugraz.at/latex/tutorial#laden_von_paketen


%doc%
%doc% \subsection{\texttt{babel}: Language settings}
%doc%
%doc% The default setting of the language is American. Please change settings for
%doc% additional or alternative languages used in \texttt{main.tex}.
%doc%
%doc% Please note that the default language of the document is the \emph{last} language
%doc% which is added to the package options.
%doc%
%doc% To set only parts of your document in a different language as the rest, use for example\newline
%doc% \verb+\foreignlanguage{ngerman}{Beispieltext in deutscher Sprache}+\newline
%doc% For using foreign language quotes, please refer to the \verb+\foreignquote+,
%doc% \verb+\foreigntextquote+, or \verb+\foreignblockquote+ provided by
%doc% \texttt{csquotes} (see Section~\ref{sub:csquotes}).
%doc%
\usepackage[\mylanguage]{babel}  %% used languages; default language is *last* language of options

%doc%
%doc% \subsection{\texttt{scrpage2}: Headers and footers}
%doc%
%doc% Since this template is based on \myacro{KOMA} script it uses its great \texttt{scrpage2}
%doc% package for defining header and footer information. Please refer to the \myacro{KOMA}
%doc% script documentation how to use this package.
%doc%
\usepackage{scrpage2} %%  advanced page style using KOMA


%doc%
%doc% \subsection{References}\myimportant
%doc% \label{sec:references}
%doc%
%doc% This template is using
%doc% \href{http://www.tex.ac.uk/tex-archive/info/translations/biblatex/de/}{\texttt{biblatex}}
%doc% and \href{http://en.wikipedia.org/wiki/Biber_(LaTeX)}{\texttt{Biber}}
%doc% instead of
%doc% \href{http://en.wikipedia.org/wiki/BibTeX}{\textsc{Bib}\TeX{}}. This has the following
%doc% advantages:
%doc% \begin{itemize}
%doc% \item better documentation
%doc% \item Unicode-support like German umlauts (ö, ä, ü, ß) for references
%doc% \item flexible definition of citation styles
%doc% \item multiple bibliographies e.\,g. for printed and online resources
%doc% \item cleaner reference definition e.\,g. inheriting information from
%doc%   \texttt{Proceedings} to all related \texttt{InProceedings}
%doc% \item modern implementation
%doc% \end{itemize}
%doc%
%doc% In short, \texttt{biblatex} is able to handle your \texttt{bib}-files
%doc% and offers additional features. To get the most out of
%doc% \texttt{biblatex}, you should read the very good package
%doc% documentation. Be warned: you'll probably never want to change back
%doc% to \textsc{Bib}\TeX{} again.
%doc%
%doc% Take a look at the files \texttt{references-bibtex.bib} and
%doc% \texttt{references-biblatex.bib}: they contain the three
%doc% references \texttt{tagstore}, \texttt{Voit2009}, and
%doc% \texttt{Voit2011}.
%doc% The second file is optimized for \texttt{biblatex} and
%doc% takes advantage of some features that are not possible with
%doc% \textsc{Bib}\TeX{}.
%doc%
%doc% This template is ready to use \texttt{biblatex} with \texttt{Biber} as
%doc% reference compiler. You should make sure that you have installed an up
%doc% to date binary of \texttt{Biber} from its
%doc% homepage\footnote{\url{http://biblatex-biber.sourceforge.net/}}.
%doc%
%doc%
%doc% In \texttt{main.tex} you can define several general \texttt{biblatex}
%doc% options: citation style, whether or not multiple occurrences of
%doc% authors are replaced with dashes, or if backward references (from
%doc% references to citations) should be added.
%doc%
%doc%
%doc% If you are using the LaTeX{} editor TeXworks, please make sure that
%doc% you have read Section~\ref{sec:biberTeXworks} in order to use
%doc% \texttt{biber}.
%doc%

%doc% \subsubsection{Example citation commands}
%doc%
%doc% This section demonstrates some example citations using the style \texttt{authoryear}.
%doc% You can change the citation style in \texttt{main.tex} (\texttt{mybiblatexstyle}).
%doc%
%doc% \begin{itemize}
%doc% \item cite \cite{Eijkhout2008} and cite \cite{Bringhurst1993, Eijkhout2008}.
%doc% \item citet \citet{Eijkhout2008} and citet \citet{Bringhurst1993, Eijkhout2008}.
%doc% \item autocite \autocite{Eijkhout2008} and autocite \autocite{Bringhurst1993, Eijkhout2008}.
%doc% \item autocites \autocites{Eijkhout2008} and autocites \autocites{Bringhurst1993, Eijkhout2008}.
%doc% \item citeauthor \citeauthor{Eijkhout2008} and citeauthor \citeauthor{Bringhurst1993, Eijkhout2008}.
%doc% \item citetitle \citetitle{Eijkhout2008} and citetitle \citetitle{Bringhurst1993, Eijkhout2008}.
%doc% \item citeyear \citeyear{Eijkhout2008} and citeyear \citeyear{Bringhurst1993, Eijkhout2008}.
%doc% \item textcite \textcite{Eijkhout2008} and textcite \textcite{Bringhurst1993, Eijkhout2008}.
%doc% \item smartcite \smartcite{Eijkhout2008} and smartcite \smartcite{Bringhurst1993, Eijkhout2008}.
%doc% \item footcite \footcite{Eijkhout2008} and footcite \footcite{Bringhurst1993, Eijkhout2008}.
%doc% \item footcite with page \footcite[p.42]{Eijkhout2008} and footcite with page \footcite[compare][p.\,42]{Eijkhout2008}.
%doc% \item fullcite \fullcite{Eijkhout2008} and fullcite \fullcite{Bringhurst1993, Eijkhout2008}.
%doc% \end{itemize}
%doc%
%doc% Please note that the citation style as well as the bibliography style
%doc% can be changed very easily. Refer to the settings in
%doc% \texttt{main.tex} as well as the very good documentation of \texttt{biblatex}.
%doc%

%doc% \subsubsection{Using this template with \myacro{APA} style}
%doc%
%doc% First, you have to have the \myacro{APA} biblatex style
%doc% installed. Modern \LaTeX{} distributions do come with
%doc% \texttt{biblatex} and \myacro{APA} style. If so, you will find the
%doc% files \texttt{biblatex-apa.pdf} (style documentation) and
%doc% \texttt{biblatex-apa-test.pdf} (file with citation examples) on your
%doc% hard disk.
%doc%
%doc% \begin{enumerate}
%doc% \item Change the style according to \verb#\newcommand{\mybiblatexstyle}{apa}#
%doc% \item Add \verb#\DeclareLanguageMapping{american}{american-apa}# or \\
%doc%   \verb#\DeclareLanguageMapping{german}{german-apa}# to your
%doc%   preamble\footnote{You might want to use section \enquote{MISC
%doc%       self-defined commands and settings} for this.}
%doc% \end{enumerate}
%doc%
%doc% These steps change the biblatex style to \myacro{APA} style

%doc%
%doc% \subsubsection{Using this template with \textsc{Bib}\TeX{}}
%doc%
%doc% If you do not want to use \texttt{Biber} and \texttt{biblatex}, you
%doc% have to change several things:
%doc% \begin{itemize}
%doc% \item in \verb#preamble/preamble.tex#
%doc%   \begin{itemize}
%doc%   \item remove the usepackage command of \texttt{biblatex}
%doc%   \item remove the \verb#\addbibresource{...}# command
%doc%   \end{itemize}
%doc% \item in \verb#main.tex#
%doc%   \begin{itemize}
%doc%   \item replace \verb=\printbibliography= with the usual
%doc%     \verb=\bibliographystyle{yourstyle}= and \verb=\bibliography{yourbibfile}=
%doc%   \end{itemize}
%doc% \item if you are using \myacro{GNU} \texttt{make}: modify \verb=Makefile=
%doc%   \begin{itemize}
%doc%   \item replace \verb#BIBTEX_CMD = biber# with \verb#BIBTEX_CMD = bibtex#
%doc%   \end{itemize}
%doc% \item Use the reference file \texttt{references-bibtex.bib}
%doc%   instead of \texttt{references-biblatex.bib}
%doc% \end{itemize}
%doc%
%doc%
\usepackage[backend=biber, %% using "biber" to compile references (instead of "biblatex")
style=\mybiblatexstyle, %% see biblatex documentation
%style=alphabetic, %% see biblatex documentation
%dashed=\mybiblatexdashed, %% do *not* replace recurring reference authors with a dash
backref=\mybiblatexbackref, %% create backlings from references to citations
natbib=true, %% offering natbib-compatible commands
hyperref=true, %% using hyperref-package references
]{biblatex}  %% remove, if using BibTeX instead of biblatex

\addbibresource{\mybiblatexfile} %% remove, if using BibTeX instead of biblatex



%doc%
%doc% \subsection{Miscellaneous packages} \label{subsec:miscpackages}
%doc%
%doc% There are several packages included by default. You might want to activate or
%doc% deactivate them according to your requirements:
%doc%
%doc% \begin{enumerate}

%doc% \item[\texttt{\href{http://www.ctan.org/pkg/graphicx}{%%
%doc% graphicx%%
%doc% }}]
%doc% The widely used package to use graphical images within a \LaTeX{} document.
\usepackage[pdftex]{graphicx}

%doc% \item[\texttt{\href{https://secure.wikimedia.org/wikibooks/en/wiki/LaTeX/Formatting\#Other\_symbols}{%%
%doc% pifont%%
%doc% }}]
%doc% For additional special characters available by \verb#\ding{}#
\usepackage{pifont}


%doc% \item[\texttt{\href{http://ctan.org/pkg/ifthen}{%%
%doc% ifthen%%
%doc% }}]
%doc% For using if/then/else statements for example in macros
\usepackage{ifthen}

%% pre-define ifthen-boolean variables:
\newboolean{myaddcolophon}
\newboolean{myaddlistoftodos}


%doc% \item[\texttt{\href{http://www.ctan.org/tex-archive/fonts/eurosym}{%%
%doc% eurosym%%
%doc% }}]
%doc% Using the character for Euro with \verb#\officialeuro{}#
%\usepackage{eurosym}

%doc% \item[\texttt{\href{http://www.ctan.org/tex-archive/help/Catalogue/entries/xspace.html}{%%
%doc% xspace%%
%doc% }}]
%doc% This package is required for intelligent spacing after commands
\usepackage{xspace}

%doc% \item[\texttt{\href{https://secure.wikimedia.org/wikibooks/en/wiki/LaTeX/Colors}{%%
%doc% xcolor%%
%doc% }}]
%doc% This package defines basic colors. If you want to get rid of colored links and headings
%doc% please change corresponding value in \texttt{main.tex} to \{0,0,0\}.
\usepackage[usenames,dvipsnames]{xcolor}
\definecolor{DispositionColor}{RGB}{\mydispositioncolor} %% used for links and so forth in screen-version

%doc% \item[\texttt{\href{http://www.ctan.org/pkg/ulem}{%%
%doc% ulem%%
%doc% }}]
%doc% This package offers strikethrough command \verb+\sout{foobar}+.
\usepackage[normalem]{ulem}

%doc% \item[\texttt{\href{http://www.ctan.org/pkg/framed}{%%
%doc% framed%%
%doc% }}]
%doc% Create framed, shaded, or differently highlighted regions that can
%doc% break across pages.  The environments defined are
%doc% \begin{itemize}
%doc%   \item framed: ordinary frame box (\verb+\fbox+) with edge at margin
%doc%   \item shaded: shaded background (\verb+\colorbox+) bleeding into margin
%doc%   \item snugshade: similar
%doc%   \item leftbar: thick vertical line in left margin
%doc% \end{itemize}
\usepackage{framed}

%doc% \item[\texttt{\href{http://www.ctan.org/pkg/eso-pic}{%%
%doc% eso-pic%%
%doc% }}]
%doc% For example on title pages you might want to have a logo on the upper right corner of
%doc% the first page (only). The package \texttt{eso-pic} is able to place things on absolute
%doc% and relative positions on the whole page.
\usepackage{eso-pic}

%doc% \item[\texttt{\href{http://ctan.org/pkg/enumitem}{%%
%doc% enumitem%%
%doc% }}]
%doc% This package replaces the built-in definitions for enumerate, itemize and description.
%doc% With \texttt{enumitem} the user has more control over the layout of those environments.
\usepackage{enumitem}

%doc% \item[\texttt{\href{http://www.ctan.org/tex-archive/macros/latex/contrib/todonotes/}{%%
%doc% todonotes%%
%doc% }}]
%doc% This packages is \emph{very} handy to add notes\footnote{\texttt{todonotes} replaced
%doc% the \texttt{fixxme}-command which previously was defined in the
%doc% \texttt{preamble\_mycommands.tex} file.}. Using for example \verb#\todo{check}#
%doc% results in something like this \todo{check} in the document. Do read the
%doc% great package documentation for usage of other very helpful commands such as
%doc% \verb#\missingfigure{}# and \verb#\listoftodos#. The latter one creates an index of all
%doc% open todos which is very useful for getting an overview of open issues.
%doc% The package \texttt{todonotes} require the packages \texttt{ifthen}, \texttt{xkeyval}, \texttt{xcolor},
%doc% \texttt{tikz}, \texttt{calc}, and \texttt{graphicx}. Activate
%doc% and configure \verb#\listoftodos# in \texttt{main.tex}.
%\usepackage{todonotes}
\usepackage[\mytodonotesoptions]{todonotes}  %% option "disable" removes all todonotes output from resulting document

%disabled% \item[\texttt{\href{http://www.ctan.org/tex-archive/macros/latex/contrib/blindtext}{%%
%disabled% blindtext%%
%disabled% }}]
%disabled% This package is used to generate blind text for demonstration purposes.
%disabled% %% This is undocumented due to problems using american english; author informed
%disabled% \usepackage{blindtext}  %% provides commands for blind text:
%disabled% %% \blindtext creates some text,
%disabled% %% \Blindtext creates more text.
%disabled% %% \blinddocument creates a small document with sections, lists...
%disabled% %% \Blinddocument creates a large document with sections, lists...
%% 2012-03-10: vk: author published a corrected version which is able to handle "american english" as well. Did not have time to check new package version for this template here.

%doc% \item[\texttt{\href{http://ctan.org/tex-archive/macros/latex/contrib/units}{%%
%doc% units%%
%doc% }}]
%doc% For setting correctly typesetted units and nice fractions with \verb+\unit[42]{m}+ and \verb+\unitfrac[100]{km}{h}+.
\usepackage{units}


%doc% \end{enumerate}




%%%% End
%%% Local Variables:
%%% TeX-master: "../main"
%%% mode: latex
%%% mode: auto-fill
%%% mode: flyspell
%%% eval: (ispell-change-dictionary "en_US")
%%% End:
%% vim:foldmethod=expr
%% vim:fde=getline(v\:lnum)=~'^%%%%'?0\:getline(v\:lnum)=~'^%doc.*\ .\\%(sub\\)\\?section{.\\+'?'>1'\:'1':
%% DO NOT REMOVE THIS LINE!

\setboolean{myaddcolophon}{true}  %% "true" or "false"
%% If set to "true": a colophon (with notes about this document
%% template, LaTeX, ...) is added after the title page.
%% Please do not set to "false" without a good reason. The colophon
%% helps your readers to get in touch with LaTeX and to find this template.

\setboolean{myaddlistoftodos}{true}  %% "true" or "false"
%% If set to "true": the current list of open todos is added after the
%% table of contents. If \mytodonotesoptions is set to "disable", no
%% list of todos is added, independent of this setting here.



%% ========================================================================
%%%% Document metadata
%% ========================================================================

%% general metadata:
\newcommand{\myauthor}{Matthias Wölbitsch}  %% also used for PDF metadata (hyperref)
\newcommand{\myauthorwithexistingtitles}{\myauthor{}, BSc}  %% including university degree already held
\newcommand{\mytitle}{Working title}  % possible titile: Introducing peer influence mechanisms to activity-driven network models??%% also used for PDF metadata (hyperref)
\newcommand{\mysubject}{Master's Thesis}  %% also used for PDF metadata (hyperref)
\newcommand{\mykeywords}{TODO}  %% also used for PDF metadata (hyperref)
%% this information is used only for generating the title page:
\newcommand{\myworktitle}{Master's Thesis}  %% official type of work like ``Master theses''
\newcommand{\mygrade}{Diplom-Ingenieur} %% title you are getting with this work like ``Master of ...''
\newcommand{\mystudy}{Computer Science} %% your study like ``Arts''
\newcommand{\mydegreeprogramme}{Master's degree programme: \mystudy} %% Master's or PhD degree programme
\newcommand{\myuniversity}{Graz University of Technology} %% your university/school
\newcommand{\myinstitute}{Knowledge Technologies Institute} %% affiliation
\newcommand{\myinstitutehead}{Univ.-Prof.\,Dr.~Stefanie Lindstaedt} %% head of institute
\newcommand{\mysupervisor}{Assoc.Prof.\,Dipl.-Ing.\,Dr.techn.~Denis Helic} %% your supervisor
\newcommand{\myhomestreet}{Jakominigürtel~5a} %% your home street (with house number)
\newcommand{\myhometown}{Graz} %% your home town
\newcommand{\myhomepostalnumber}{8010} %% your postal number of home town
\newcommand{\mysubmissionmonth}{sometime} %% month you are handing in
\newcommand{\mysubmissionyear}{2017} %% year you are handing in
\newcommand{\mysubmissiontown}{\myhometown} %% town of handing in (or \myhometown)

%% additional information for generic_documentation title page
\newcommand{\myid}{113099} %% Matrikelnummer
\newcommand{\mylecture}{Master's Thesis} %%


%% ========================================================================
%%%% MISC command definitions
%% ========================================================================
%% Time-stamp: <2015-04-30 17:19:58 vk>
%%%% === Disclaimer: =======================================================
%% created by
%%
%%      Karl Voit
%%
%% using GNU/Linux, GNU Emacs & LaTeX 2e
%%

%doc%
%doc% \section{\texttt{mycommands.tex} --- various definitions}\myinteresting
%doc% \label{sec:mycommands}
%doc%
%doc% In file \verb#template/mycommands.tex# many useful commands are being
%doc% defined.
%doc%
%doc% \paragraph{What should I do with this file?} Please take a look at its
%doc% content to get the most out of your document.
%doc%

%doc%
%doc% One of the best advantages of \LaTeX{} compared to \myacro{WYSIWYG} software products is
%doc% the possibility to define and use macros within text. This empowers the user to
%doc% a great extend.  Many things can be defined using \verb#\newcommand{}# and
%doc% automates repeating tasks. It is recommended to use macros not only for
%doc% repetitive tasks but also for separating form from content such as \myacro{CSS}
%doc% does for \myacro{XHTML}. Think of including graphics in your document: after
%doc% writing your book, you might want to change all captions to the upper side of
%doc% each figure. In this case you either have to modify all
%doc% \texttt{includegraphics} commands or you were clever enough to define something
%doc% like \verb#\myfig#\footnote{See below for a detailed description}. Using a
%doc% macro for including graphics enables you to modify the position caption on only
%doc% \emph{one} place: at the definition of the macro.
%doc%
%doc% The following section describes some macros that came with this document template
%doc% from \myLaT and you are welcome to modify or extend them or to create
%doc% your own macros!
%doc%

%doc%
%doc% \subsection{\texttt{myfig} --- including graphics made easy}
%doc%
%doc% The classic: you can easily add graphics to your document with \verb#\myfig#:
%doc% \begin{verbatim}
%doc%  \myfig{flower}%% filename w/o extension in the folder figures
%doc%        {width=0.7\textwidth}%% maximum width/height, aspect ratio will be kept
%doc%        {This flower was photographed at my home town in 2010}%% caption
%doc%        {Home town flower}%% optional (short) caption for list of figures
%doc%        {fig:flower}%% label
%doc% \end{verbatim}
%doc%
%doc% There are many advantages of this command (compared to manual
%doc% \texttt{figure} environments and \texttt{includegraphics} commands:
%doc% \begin{itemize}
%doc% \item consistent style throughout the whole document
%doc% \item easy to change; for example move caption on top
%doc% \item much less characters to type (faster, error prone)
%doc% \item less visual clutter in the \TeX{}-files
%doc% \end{itemize}
%doc%
%doc%
\newcommand{\myfig}[5]{
%% example:
% \myfig{}%% filename in figures folder
%       {width=0.5\textwidth,height=0.5\textheight}%% maximum width/height, aspect ratio will be kept
%       {}%% caption
%       {}%% optional (short) caption for list of figures
%       {}%% label
\begin{figure}[thp]
  \centering
  \includegraphics[keepaspectratio,#2]{figures/#1}
  \caption[#4]{#3}
  \label{#5} %% NOTE: always label *after* caption!
\end{figure}
}


%doc%
%doc% \subsection{\texttt{myclone} --- repeat things!}
%doc%
%doc% Using \verb#\myclone[42]{foobar}# results the text \enquote{foobar} printed 42 times.
%doc% But you can not only repeat text output with \texttt{myclone}.
%doc%
%doc% Default argument
%doc% for the optional parameter \enquote{number of times} (like \enquote{42} in the example above)
%doc% is set to two.
%doc%
%% \myclone[x]{text}
\newcounter{myclonecnt}
\newcommand{\myclone}[2][2]{%
  \setcounter{myclonecnt}{#1}%
  \whiledo{\value{myclonecnt}>0}{#2\addtocounter{myclonecnt}{-1}}%
}

%old% %d oc%
%old% %d oc% \subsection{\texttt{fixxme} --- sidemark something as unfinished}
%old% %d oc%
%old% %d oc% You know it: something has to be fixed and you can not do it right
%old% %d oc% now. In order to \texttt{not} forget about it, you might want to add a
%old% %d oc% note like \verb+\fixxme{check again}+ which inserts a note on the page
%old% %d oc% margin such as this\fixxme{check again} example.
%old% %d oc%
%old% \newcommand{\fixxme}[1]{%%
%old% \textcolor{red}{FIXXME}\marginpar{\textcolor{red}{#1}}%%
%old% }


%%%% End
%%% Local Variables:
%%% mode: latex
%%% mode: auto-fill
%%% mode: flyspell
%%% eval: (ispell-change-dictionary "en_US")
%%% TeX-master: "../main"
%%% End:
%% vim:foldmethod=expr
%% vim:fde=getline(v\:lnum)=~'^%%%%'?0\:getline(v\:lnum)=~'^%doc.*\ .\\%(sub\\)\\?section{.\\+'?'>1'\:'1':


%% ========================================================================
%%%% Typographic settings
%% ========================================================================
%%%% Time-stamp: <2015-08-22 17:20:32 vk>
%%%% === Disclaimer: =======================================================
%% created by
%%
%%      Karl Voit
%%
%% using GNU/Linux, GNU Emacs & LaTeX 2e
%%
%doc%
%doc% \section{\texttt{typographic\_settings.tex} --- Typographic finetuning}
%doc%
%doc% The settings of file \verb#template/typographic_settings.tex# contain
%doc% typographic finetuning related to things mentioned in literature.  The
%doc% settings in this file relates to personal taste and most of all:
%doc% \emph{typographic experience}.
%doc%
%doc% \paragraph{What should I do with this file?} You might as well skip the whole
%doc% file by excluding the \verb#%%%% Time-stamp: <2015-08-22 17:20:32 vk>
%%%% === Disclaimer: =======================================================
%% created by
%%
%%      Karl Voit
%%
%% using GNU/Linux, GNU Emacs & LaTeX 2e
%%
%doc%
%doc% \section{\texttt{typographic\_settings.tex} --- Typographic finetuning}
%doc%
%doc% The settings of file \verb#template/typographic_settings.tex# contain
%doc% typographic finetuning related to things mentioned in literature.  The
%doc% settings in this file relates to personal taste and most of all:
%doc% \emph{typographic experience}.
%doc%
%doc% \paragraph{What should I do with this file?} You might as well skip the whole
%doc% file by excluding the \verb#%%%% Time-stamp: <2015-08-22 17:20:32 vk>
%%%% === Disclaimer: =======================================================
%% created by
%%
%%      Karl Voit
%%
%% using GNU/Linux, GNU Emacs & LaTeX 2e
%%
%doc%
%doc% \section{\texttt{typographic\_settings.tex} --- Typographic finetuning}
%doc%
%doc% The settings of file \verb#template/typographic_settings.tex# contain
%doc% typographic finetuning related to things mentioned in literature.  The
%doc% settings in this file relates to personal taste and most of all:
%doc% \emph{typographic experience}.
%doc%
%doc% \paragraph{What should I do with this file?} You might as well skip the whole
%doc% file by excluding the \verb#\input{template/typographic_settings.tex}# command
%doc% in \texttt{main.tex}.  For standard usage it is recommended to stay with the
%doc% default settings.
%doc%
%doc%
%% ========================================================================

%doc%
%doc% Some basic microtypographic settings are provided by the
%doc% \texttt{microtype}
%doc% package\footnote{\url{http://ctan.org/pkg/microtype}}. This template
%doc% uses the rather conservative package parameters: \texttt{protrusion=true,factor=900}.
\usepackage[protrusion=true,factor=900]{microtype}

%doc%
%doc% \subsection{French spacing}
%doc%
%doc% \paragraph{Why?} see~\textcite[p.\,28, p.\,30]{Bringhurst1993}: `2.1.4 Use a single word space between sentences.'
%doc%
%doc% \paragraph{How?} see~\textcite[p.\,185]{Eijkhout2008}:\\
%doc% \verb#\frenchspacing  %% Macro to switch off extra space after punctuation.# \\
\frenchspacing  %% Macro to switch off extra space after punctuation.
%doc%
%doc% Note: This setting might be default for \myacro{KOMA} script.
%doc%


%doc%
%doc% \subsection{Font}
%doc%
%doc% This template is using the Palatino font (package \texttt{mathpazo}) which results
%doc% in a legible document and matching mathematical fonts for printout.
%doc%
%doc% It is highly recommended that you either stick to the Palatino font or use the
%doc% \LaTeX{} default fonts (by removing the package \texttt{mathpazo}).
%doc%
%doc% Chosing different fonts is not
%doc% an easy task. Please leave this to people with good knowledge on this subject.
%doc%
%doc% One valid reason to change the default fonts is when your document is mainly
%doc% read on a computer screen. In this case it is recommended to switch to a font
%doc% \textsf{which is sans-serif like this}. This template contains several alternative
%doc% font packages which can be activated in this file.
%doc%

% for changing the default font, please go to the next subsection!

%doc%
%doc% \subsection{Text figures}
%doc%
%doc% \ldots also called old style numbers such as 0123456789.
%doc% (German: \enquote{Mediäval\-ziffern\footnote{\url{https://secure.wikimedia.org/wikibooks/de/wiki/LaTeX-W\%C3\%B6rterbuch:\_Medi\%C3\%A4valziffern}}})
%doc% \paragraph{Why?} see~\textcite[p.\,44f]{Bringhurst1993}:
%doc% \begin{quote}
%doc% `3.2.1 If the font includes both text figures and titling figures, use
%doc%  titling figures only with full caps, and text figures in all other
%doc%  circumstances.'
%doc% \end{quote}
%doc%
%doc% \paragraph{How?}
%doc% Quoted from Wikibooks\footnote{\url{https://secure.wikimedia.org/wikibooks/en/wiki/LaTeX/Formatting\#Text\_figures\_.28.22old\_style.22\_numerals.29}}:
%doc% \begin{quote}
%doc% Some fonts do not have text figures built in; the textcomp package attempts to
%doc% remedy this by effectively generating text figures from the currently-selected
%doc% font. Put \verb#\usepackage{textcomp}# in your preamble. textcomp also allows you to
%doc% use decimal points, properly formatted dollar signs, etc. within
%doc% \verb#\oldstylenums{}#.
%doc% \end{quote}
%doc% \ldots but proposed \LaTeX{} method does not work out well. Instead use:\\
%doc% \verb#\usepackage{hfoldsty}#  (enables text figures using additional font) or \\
%doc% \verb#\usepackage[sc,osf]{mathpazo}# (switches to Palatino font with small caps and old style figures enabled).
%doc%
%\usepackage{hfoldsty}  %% enables text figures using additional font
%% ... OR use ...
\usepackage[sc,osf]{mathpazo} %% switches to Palatino with small caps and old style figures

%% Font selection from:
%%     http://www.matthiaspospiech.de/latex/vorlagen/allgemein/preambel/fonts/
%% use following lines *instead* of the mathpazo package above:
%% ===== Serif =========================================================
%% for Computer Modern (LaTeX default font), simply remove the mathpazo above
%\usepackage{charter}\linespread{1.05} %% Charter
%\usepackage{bookman}                  %% Bookman (laedt Avant Garde !!)
%\usepackage{newcent}                  %% New Century Schoolbook (laedt Avant Garde !!)
%% ===== Sans Serif ====================================================
%\renewcommand{\familydefault}{\sfdefault}  %% this one in *combination* with the default mathpazo package
%\usepackage{cmbright}                  %% CM-Bright (eigntlich eine Familie)
%\usepackage{tpslifonts}                %% tpslifonts % Font for Slides


%doc%
%doc% \subsection{\texttt{myacro} --- Abbrevations using \textsc{small caps}}\myinteresting
%doc% \label{sec:myacro}
%doc%
%doc% \paragraph{Why?} see~\textcite[p.\,45f]{Bringhurst1993}: `3.2.2 For abbrevations and
%doc% acronyms in the midst of normal text, use spaced small caps.'
%doc%
%doc% \paragraph{How?} Using the predefined macro \verb#\myacro{}# for things like
%doc% \myacro{UNO} or \myacro{UNESCO} using \verb#\myacro{UNO}# or \verb#\myacro{UNESCO}#.
%doc%
\DeclareRobustCommand{\myacro}[1]{\textsc{\lowercase{#1}}} %%  abbrevations using small caps


%doc%
%doc% \subsection{Colorized headings and links}
%doc%
%doc% This document template is able to generate an output that uses colorized
%doc% headings, captions, page numbers, and links. The color named `DispositionColor'
%doc% used in this document is defined near the definition of package \texttt{color}
%doc% in the preamble (see section~\ref{subsec:miscpackages}). The changes required
%doc% for headings, page numbers, and captions are defined here.
%doc%
%doc% Settings for colored links are handled by the definitions of the
%doc% \texttt{hyperref} package (see section~\ref{sec:pdf}).
%doc%
\setheadsepline{.4pt}[\color{DispositionColor}]
\renewcommand{\headfont}{\normalfont\sffamily\color{DispositionColor}}
\renewcommand{\pnumfont}{\normalfont\sffamily\color{DispositionColor}}
\addtokomafont{disposition}{\color{DispositionColor}}
\addtokomafont{caption}{\color{DispositionColor}\footnotesize}
\addtokomafont{captionlabel}{\color{DispositionColor}}

%doc%
%doc% \subsection{No figures or tables below footnotes}
%doc%
%doc% \LaTeX{} places floating environments below footnotes if \texttt{b}
%doc% (bottom) is used as (default) placement algorithm. This is certainly
%doc% not appealing for most people and is deactivated in this template by
%doc% using the package \texttt{footmisc} with its option \texttt{bottom}.
%doc%
%% see also: http://www.komascript.de/node/858 (German description)
\usepackage[bottom]{footmisc}

%doc%
%doc% \subsection{Spacings of list environments}
%doc%
%doc% By default, \LaTeX{} is using vertical spaces between items of enumerate,
%doc% itemize and description environments. This is fine for multi-line items.
%doc% Many times, the user does just write single-line items where the larger
%doc% vertical space is inappropriate. The \href{http://ctan.org/pkg/enumitem}{enumitem}
%doc% package provides replacements for the pre-defined list environments and
%doc% offers many options to modify their appearances.
%doc% This template is using the package option for \texttt{noitemsep} which
%doc% mimimizes the vertical space between list items.
%doc%
\usepackage{enumitem}
%\setlist{noitemsep}   %% kills the space between items

%doc%
%doc% \subsection{\texttt{csquotes} --- Correct quotation marks}\myinteresting
%doc% \label{sub:csquotes}
%doc%
%doc% \emph{Never} use quotation marks found on your keyboard.
%doc% They end up in strange characters or false looking quotation marks.
%doc%
%doc% In \LaTeX{} you are able to use typographically correct quotation marks. The package
%doc% \href{http://www.ctan.org/pkg/csquotes}{\texttt{csquotes}} offers you with
%doc% \verb#\enquote{foobar}# a command to get correct quotation marks around \enquote{foobar}.
%doc% Please do check the package options in order to modify
%doc% its settings according to the language used\footnote{most of the time in
%doc% combination with the language set in the options of the \texttt{babel} package}.
%doc%
%doc% \href{http://www.ctan.org/pkg/csquotes}{\texttt{csquotes}} is also recommended
%doc% by \texttt{biblatex} (see Section~\ref{sec:references}).
\usepackage[babel=true,strict=true,english=american,german=guillemets]{csquotes}

%doc%
%doc% \subsection{Line spread}
%doc%
%doc% If you have to enlarge the distance between two lines of text, you can
%doc% increase it using the \texttt{\mylinespread} command in \texttt{main.tex}. By default, it is
%doc% deactivated (set to 100~percent). Modify only with caution since it influences the
%doc% page layout and could lead to ugly looking documents.
\linespread{\mylinespread}

%doc%
%doc% \subsection{Optional: Lines above and below the chapter head}
%doc%
%doc% This is not quite something typographic but rather a matter of taste.
%doc% \myacro{KOMA} Script offers \href{http://www.komascript.de/node/24}{a method to
%doc% add lines above and below chapter head} which is disabled by
%doc% default. If you want to enable this feature, remove corresponding
%doc% comment characters from the settings.
%doc%
%% Source: http://www.komascript.de/node/24
%disabled% %% 1st get a new command
%disabled% \newcommand*{\ORIGchapterheadstartvskip}{}%
%disabled% %% 2nd save the original definition to the new command
%disabled% \let\ORIGchapterheadstartvskip=\chapterheadstartvskip
%disabled% %% 3rd redefine the command using the saved original command
%disabled% \renewcommand*{\chapterheadstartvskip}{%
%disabled%   \ORIGchapterheadstartvskip
%disabled%   {%
%disabled%     \setlength{\parskip}{0pt}%
%disabled%     \noindent\color{DispositionColor}\rule[.3\baselineskip]{\linewidth}{1pt}\par
%disabled%   }%
%disabled% }
%disabled% %% see above
%disabled% \newcommand*{\ORIGchapterheadendvskip}{}%
%disabled% \let\ORIGchapterheadendvskip=\chapterheadendvskip
%disabled% \renewcommand*{\chapterheadendvskip}{%
%disabled%   {%
%disabled%     \setlength{\parskip}{0pt}%
%disabled%     \noindent\color{DispositionColor}\rule[.3\baselineskip]{\linewidth}{1pt}\par
%disabled%   }%
%disabled%   \ORIGchapterheadendvskip
%disabled% }

%doc%
%doc% \subsection{Optional: Chapter thumbs}
%doc%
%doc% This is not quite something typographic but rather a matter of taste.
%doc% \myacro{KOMA} Script offers \href{http://www.komascript.de/chapterthumbs-example}{a method to
%doc% add chapter thumbs} (in combination with the package \texttt{scrpage2}) which is disabled by
%doc% default. If you want to enable this feature, remove corresponding
%doc% comment characters from the settings.
%doc%
%disabled% \makeatletter
%disabled% % Safty first
%disabled% \@ifundefined{chapter}{\let\chapter\undefined
%disabled%   \chapter must be defined to use chapter thumbs!}{%
%disabled%
%disabled% % Two new commands for the width and height of the boxes with the
%disabled% % chapter number at the thumbs (use of commands instead of lengths
%disabled% % for sparing registers)
%disabled% \newcommand*{\chapterthumbwidth}{2em}
%disabled% \newcommand*{\chapterthumbheight}{1em}
%disabled%
%disabled% % Two new commands for the colors of the box background and the
%disabled% % chapter numbers of the thumbs
%disabled% \newcommand*{\chapterthumbboxcolor}{black}
%disabled% \newcommand*{\chapterthumbtextcolor}{white}
%disabled%
%disabled% % New command to set a chapter thumb. I'm using a group at this
%disabled% % command, because I'm changing the temporary dimension \@tempdima
%disabled% \newcommand*{\putchapterthumb}{%
%disabled%   \begingroup
%disabled%     \Large
%disabled%     % calculate the horizontal possition of the right paper border
%disabled%     % (I ignore \hoffset, because I interprete \hoffset moves the page
%disabled%     % at the paper e.g. if you are using cropmarks)
%disabled%     \setlength{\@tempdima}{\@oddheadshift}% (internal from scrpage2)
%disabled%     \setlength{\@tempdima}{-\@tempdima}%
%disabled%     \addtolength{\@tempdima}{\paperwidth}%
%disabled%     \addtolength{\@tempdima}{-\oddsidemargin}%
%disabled%     \addtolength{\@tempdima}{-1in}%
%disabled%     % putting the thumbs should not change the horizontal
%disabled%     % possition
%disabled%     \rlap{%
%disabled%       % move to the calculated horizontal possition
%disabled%       \hspace*{\@tempdima}%
%disabled%       % putting the thumbs should not change the vertical
%disabled%       % possition
%disabled%       \vbox to 0pt{%
%disabled%         % calculate the vertical possition of the thumbs (I ignore
%disabled%         % \voffset for the same reasons told above)
%disabled%         \setlength{\@tempdima}{\chapterthumbwidth}%
%disabled%         \multiply\@tempdima by\value{chapter}%
%disabled%         \addtolength{\@tempdima}{-\chapterthumbwidth}%
%disabled%         \addtolength{\@tempdima}{-\baselineskip}%
%disabled%         % move to the calculated vertical possition
%disabled%         \vspace*{\@tempdima}%
%disabled%         % put the thumbs left so the current horizontal possition
%disabled%         \llap{%
%disabled%           % and rotate them
%disabled%           \rotatebox{90}{\colorbox{\chapterthumbboxcolor}{%
%disabled%               \parbox[c][\chapterthumbheight][c]{\chapterthumbwidth}{%
%disabled%                 \centering
%disabled%                 \textcolor{\chapterthumbtextcolor}{%
%disabled%                   \strut\thechapter}\\
%disabled%               }%
%disabled%             }%
%disabled%           }%
%disabled%         }%
%disabled%         % avoid overfull \vbox messages
%disabled%         \vss
%disabled%       }%
%disabled%     }%
%disabled%   \endgroup
%disabled% }
%disabled%
%disabled% % New command, which works like \lohead but also puts the thumbs (you
%disabled% % cannot use \ihead with this definition but you may change this, if
%disabled% % you use more internal scrpage2 commands)
%disabled% \newcommand*{\loheadwithchapterthumbs}[2][]{%
%disabled%   \lohead[\putchapterthumb#1]{\putchapterthumb#2}%
%disabled% }
%disabled%
%disabled% % initial use
%disabled% \loheadwithchapterthumbs{}
%disabled% \pagestyle{scrheadings}
%disabled%
%disabled% }
%disabled% \makeatother

%%%% END
%%% Local Variables:
%%% mode: latex
%%% mode: auto-fill
%%% mode: flyspell
%%% eval: (ispell-change-dictionary "en_US")
%%% TeX-master: "../main"
%%% End:
%% vim:foldmethod=expr
%% vim:fde=getline(v\:lnum)=~'^%%%%'?0\:getline(v\:lnum)=~'^%doc.*\ .\\%(sub\\)\\?section{.\\+'?'>1'\:'1':
# command
%doc% in \texttt{main.tex}.  For standard usage it is recommended to stay with the
%doc% default settings.
%doc%
%doc%
%% ========================================================================

%doc%
%doc% Some basic microtypographic settings are provided by the
%doc% \texttt{microtype}
%doc% package\footnote{\url{http://ctan.org/pkg/microtype}}. This template
%doc% uses the rather conservative package parameters: \texttt{protrusion=true,factor=900}.
\usepackage[protrusion=true,factor=900]{microtype}

%doc%
%doc% \subsection{French spacing}
%doc%
%doc% \paragraph{Why?} see~\textcite[p.\,28, p.\,30]{Bringhurst1993}: `2.1.4 Use a single word space between sentences.'
%doc%
%doc% \paragraph{How?} see~\textcite[p.\,185]{Eijkhout2008}:\\
%doc% \verb#\frenchspacing  %% Macro to switch off extra space after punctuation.# \\
\frenchspacing  %% Macro to switch off extra space after punctuation.
%doc%
%doc% Note: This setting might be default for \myacro{KOMA} script.
%doc%


%doc%
%doc% \subsection{Font}
%doc%
%doc% This template is using the Palatino font (package \texttt{mathpazo}) which results
%doc% in a legible document and matching mathematical fonts for printout.
%doc%
%doc% It is highly recommended that you either stick to the Palatino font or use the
%doc% \LaTeX{} default fonts (by removing the package \texttt{mathpazo}).
%doc%
%doc% Chosing different fonts is not
%doc% an easy task. Please leave this to people with good knowledge on this subject.
%doc%
%doc% One valid reason to change the default fonts is when your document is mainly
%doc% read on a computer screen. In this case it is recommended to switch to a font
%doc% \textsf{which is sans-serif like this}. This template contains several alternative
%doc% font packages which can be activated in this file.
%doc%

% for changing the default font, please go to the next subsection!

%doc%
%doc% \subsection{Text figures}
%doc%
%doc% \ldots also called old style numbers such as 0123456789.
%doc% (German: \enquote{Mediäval\-ziffern\footnote{\url{https://secure.wikimedia.org/wikibooks/de/wiki/LaTeX-W\%C3\%B6rterbuch:\_Medi\%C3\%A4valziffern}}})
%doc% \paragraph{Why?} see~\textcite[p.\,44f]{Bringhurst1993}:
%doc% \begin{quote}
%doc% `3.2.1 If the font includes both text figures and titling figures, use
%doc%  titling figures only with full caps, and text figures in all other
%doc%  circumstances.'
%doc% \end{quote}
%doc%
%doc% \paragraph{How?}
%doc% Quoted from Wikibooks\footnote{\url{https://secure.wikimedia.org/wikibooks/en/wiki/LaTeX/Formatting\#Text\_figures\_.28.22old\_style.22\_numerals.29}}:
%doc% \begin{quote}
%doc% Some fonts do not have text figures built in; the textcomp package attempts to
%doc% remedy this by effectively generating text figures from the currently-selected
%doc% font. Put \verb#\usepackage{textcomp}# in your preamble. textcomp also allows you to
%doc% use decimal points, properly formatted dollar signs, etc. within
%doc% \verb#\oldstylenums{}#.
%doc% \end{quote}
%doc% \ldots but proposed \LaTeX{} method does not work out well. Instead use:\\
%doc% \verb#\usepackage{hfoldsty}#  (enables text figures using additional font) or \\
%doc% \verb#\usepackage[sc,osf]{mathpazo}# (switches to Palatino font with small caps and old style figures enabled).
%doc%
%\usepackage{hfoldsty}  %% enables text figures using additional font
%% ... OR use ...
\usepackage[sc,osf]{mathpazo} %% switches to Palatino with small caps and old style figures

%% Font selection from:
%%     http://www.matthiaspospiech.de/latex/vorlagen/allgemein/preambel/fonts/
%% use following lines *instead* of the mathpazo package above:
%% ===== Serif =========================================================
%% for Computer Modern (LaTeX default font), simply remove the mathpazo above
%\usepackage{charter}\linespread{1.05} %% Charter
%\usepackage{bookman}                  %% Bookman (laedt Avant Garde !!)
%\usepackage{newcent}                  %% New Century Schoolbook (laedt Avant Garde !!)
%% ===== Sans Serif ====================================================
%\renewcommand{\familydefault}{\sfdefault}  %% this one in *combination* with the default mathpazo package
%\usepackage{cmbright}                  %% CM-Bright (eigntlich eine Familie)
%\usepackage{tpslifonts}                %% tpslifonts % Font for Slides


%doc%
%doc% \subsection{\texttt{myacro} --- Abbrevations using \textsc{small caps}}\myinteresting
%doc% \label{sec:myacro}
%doc%
%doc% \paragraph{Why?} see~\textcite[p.\,45f]{Bringhurst1993}: `3.2.2 For abbrevations and
%doc% acronyms in the midst of normal text, use spaced small caps.'
%doc%
%doc% \paragraph{How?} Using the predefined macro \verb#\myacro{}# for things like
%doc% \myacro{UNO} or \myacro{UNESCO} using \verb#\myacro{UNO}# or \verb#\myacro{UNESCO}#.
%doc%
\DeclareRobustCommand{\myacro}[1]{\textsc{\lowercase{#1}}} %%  abbrevations using small caps


%doc%
%doc% \subsection{Colorized headings and links}
%doc%
%doc% This document template is able to generate an output that uses colorized
%doc% headings, captions, page numbers, and links. The color named `DispositionColor'
%doc% used in this document is defined near the definition of package \texttt{color}
%doc% in the preamble (see section~\ref{subsec:miscpackages}). The changes required
%doc% for headings, page numbers, and captions are defined here.
%doc%
%doc% Settings for colored links are handled by the definitions of the
%doc% \texttt{hyperref} package (see section~\ref{sec:pdf}).
%doc%
\setheadsepline{.4pt}[\color{DispositionColor}]
\renewcommand{\headfont}{\normalfont\sffamily\color{DispositionColor}}
\renewcommand{\pnumfont}{\normalfont\sffamily\color{DispositionColor}}
\addtokomafont{disposition}{\color{DispositionColor}}
\addtokomafont{caption}{\color{DispositionColor}\footnotesize}
\addtokomafont{captionlabel}{\color{DispositionColor}}

%doc%
%doc% \subsection{No figures or tables below footnotes}
%doc%
%doc% \LaTeX{} places floating environments below footnotes if \texttt{b}
%doc% (bottom) is used as (default) placement algorithm. This is certainly
%doc% not appealing for most people and is deactivated in this template by
%doc% using the package \texttt{footmisc} with its option \texttt{bottom}.
%doc%
%% see also: http://www.komascript.de/node/858 (German description)
\usepackage[bottom]{footmisc}

%doc%
%doc% \subsection{Spacings of list environments}
%doc%
%doc% By default, \LaTeX{} is using vertical spaces between items of enumerate,
%doc% itemize and description environments. This is fine for multi-line items.
%doc% Many times, the user does just write single-line items where the larger
%doc% vertical space is inappropriate. The \href{http://ctan.org/pkg/enumitem}{enumitem}
%doc% package provides replacements for the pre-defined list environments and
%doc% offers many options to modify their appearances.
%doc% This template is using the package option for \texttt{noitemsep} which
%doc% mimimizes the vertical space between list items.
%doc%
\usepackage{enumitem}
%\setlist{noitemsep}   %% kills the space between items

%doc%
%doc% \subsection{\texttt{csquotes} --- Correct quotation marks}\myinteresting
%doc% \label{sub:csquotes}
%doc%
%doc% \emph{Never} use quotation marks found on your keyboard.
%doc% They end up in strange characters or false looking quotation marks.
%doc%
%doc% In \LaTeX{} you are able to use typographically correct quotation marks. The package
%doc% \href{http://www.ctan.org/pkg/csquotes}{\texttt{csquotes}} offers you with
%doc% \verb#\enquote{foobar}# a command to get correct quotation marks around \enquote{foobar}.
%doc% Please do check the package options in order to modify
%doc% its settings according to the language used\footnote{most of the time in
%doc% combination with the language set in the options of the \texttt{babel} package}.
%doc%
%doc% \href{http://www.ctan.org/pkg/csquotes}{\texttt{csquotes}} is also recommended
%doc% by \texttt{biblatex} (see Section~\ref{sec:references}).
\usepackage[babel=true,strict=true,english=american,german=guillemets]{csquotes}

%doc%
%doc% \subsection{Line spread}
%doc%
%doc% If you have to enlarge the distance between two lines of text, you can
%doc% increase it using the \texttt{\mylinespread} command in \texttt{main.tex}. By default, it is
%doc% deactivated (set to 100~percent). Modify only with caution since it influences the
%doc% page layout and could lead to ugly looking documents.
\linespread{\mylinespread}

%doc%
%doc% \subsection{Optional: Lines above and below the chapter head}
%doc%
%doc% This is not quite something typographic but rather a matter of taste.
%doc% \myacro{KOMA} Script offers \href{http://www.komascript.de/node/24}{a method to
%doc% add lines above and below chapter head} which is disabled by
%doc% default. If you want to enable this feature, remove corresponding
%doc% comment characters from the settings.
%doc%
%% Source: http://www.komascript.de/node/24
%disabled% %% 1st get a new command
%disabled% \newcommand*{\ORIGchapterheadstartvskip}{}%
%disabled% %% 2nd save the original definition to the new command
%disabled% \let\ORIGchapterheadstartvskip=\chapterheadstartvskip
%disabled% %% 3rd redefine the command using the saved original command
%disabled% \renewcommand*{\chapterheadstartvskip}{%
%disabled%   \ORIGchapterheadstartvskip
%disabled%   {%
%disabled%     \setlength{\parskip}{0pt}%
%disabled%     \noindent\color{DispositionColor}\rule[.3\baselineskip]{\linewidth}{1pt}\par
%disabled%   }%
%disabled% }
%disabled% %% see above
%disabled% \newcommand*{\ORIGchapterheadendvskip}{}%
%disabled% \let\ORIGchapterheadendvskip=\chapterheadendvskip
%disabled% \renewcommand*{\chapterheadendvskip}{%
%disabled%   {%
%disabled%     \setlength{\parskip}{0pt}%
%disabled%     \noindent\color{DispositionColor}\rule[.3\baselineskip]{\linewidth}{1pt}\par
%disabled%   }%
%disabled%   \ORIGchapterheadendvskip
%disabled% }

%doc%
%doc% \subsection{Optional: Chapter thumbs}
%doc%
%doc% This is not quite something typographic but rather a matter of taste.
%doc% \myacro{KOMA} Script offers \href{http://www.komascript.de/chapterthumbs-example}{a method to
%doc% add chapter thumbs} (in combination with the package \texttt{scrpage2}) which is disabled by
%doc% default. If you want to enable this feature, remove corresponding
%doc% comment characters from the settings.
%doc%
%disabled% \makeatletter
%disabled% % Safty first
%disabled% \@ifundefined{chapter}{\let\chapter\undefined
%disabled%   \chapter must be defined to use chapter thumbs!}{%
%disabled%
%disabled% % Two new commands for the width and height of the boxes with the
%disabled% % chapter number at the thumbs (use of commands instead of lengths
%disabled% % for sparing registers)
%disabled% \newcommand*{\chapterthumbwidth}{2em}
%disabled% \newcommand*{\chapterthumbheight}{1em}
%disabled%
%disabled% % Two new commands for the colors of the box background and the
%disabled% % chapter numbers of the thumbs
%disabled% \newcommand*{\chapterthumbboxcolor}{black}
%disabled% \newcommand*{\chapterthumbtextcolor}{white}
%disabled%
%disabled% % New command to set a chapter thumb. I'm using a group at this
%disabled% % command, because I'm changing the temporary dimension \@tempdima
%disabled% \newcommand*{\putchapterthumb}{%
%disabled%   \begingroup
%disabled%     \Large
%disabled%     % calculate the horizontal possition of the right paper border
%disabled%     % (I ignore \hoffset, because I interprete \hoffset moves the page
%disabled%     % at the paper e.g. if you are using cropmarks)
%disabled%     \setlength{\@tempdima}{\@oddheadshift}% (internal from scrpage2)
%disabled%     \setlength{\@tempdima}{-\@tempdima}%
%disabled%     \addtolength{\@tempdima}{\paperwidth}%
%disabled%     \addtolength{\@tempdima}{-\oddsidemargin}%
%disabled%     \addtolength{\@tempdima}{-1in}%
%disabled%     % putting the thumbs should not change the horizontal
%disabled%     % possition
%disabled%     \rlap{%
%disabled%       % move to the calculated horizontal possition
%disabled%       \hspace*{\@tempdima}%
%disabled%       % putting the thumbs should not change the vertical
%disabled%       % possition
%disabled%       \vbox to 0pt{%
%disabled%         % calculate the vertical possition of the thumbs (I ignore
%disabled%         % \voffset for the same reasons told above)
%disabled%         \setlength{\@tempdima}{\chapterthumbwidth}%
%disabled%         \multiply\@tempdima by\value{chapter}%
%disabled%         \addtolength{\@tempdima}{-\chapterthumbwidth}%
%disabled%         \addtolength{\@tempdima}{-\baselineskip}%
%disabled%         % move to the calculated vertical possition
%disabled%         \vspace*{\@tempdima}%
%disabled%         % put the thumbs left so the current horizontal possition
%disabled%         \llap{%
%disabled%           % and rotate them
%disabled%           \rotatebox{90}{\colorbox{\chapterthumbboxcolor}{%
%disabled%               \parbox[c][\chapterthumbheight][c]{\chapterthumbwidth}{%
%disabled%                 \centering
%disabled%                 \textcolor{\chapterthumbtextcolor}{%
%disabled%                   \strut\thechapter}\\
%disabled%               }%
%disabled%             }%
%disabled%           }%
%disabled%         }%
%disabled%         % avoid overfull \vbox messages
%disabled%         \vss
%disabled%       }%
%disabled%     }%
%disabled%   \endgroup
%disabled% }
%disabled%
%disabled% % New command, which works like \lohead but also puts the thumbs (you
%disabled% % cannot use \ihead with this definition but you may change this, if
%disabled% % you use more internal scrpage2 commands)
%disabled% \newcommand*{\loheadwithchapterthumbs}[2][]{%
%disabled%   \lohead[\putchapterthumb#1]{\putchapterthumb#2}%
%disabled% }
%disabled%
%disabled% % initial use
%disabled% \loheadwithchapterthumbs{}
%disabled% \pagestyle{scrheadings}
%disabled%
%disabled% }
%disabled% \makeatother

%%%% END
%%% Local Variables:
%%% mode: latex
%%% mode: auto-fill
%%% mode: flyspell
%%% eval: (ispell-change-dictionary "en_US")
%%% TeX-master: "../main"
%%% End:
%% vim:foldmethod=expr
%% vim:fde=getline(v\:lnum)=~'^%%%%'?0\:getline(v\:lnum)=~'^%doc.*\ .\\%(sub\\)\\?section{.\\+'?'>1'\:'1':
# command
%doc% in \texttt{main.tex}.  For standard usage it is recommended to stay with the
%doc% default settings.
%doc%
%doc%
%% ========================================================================

%doc%
%doc% Some basic microtypographic settings are provided by the
%doc% \texttt{microtype}
%doc% package\footnote{\url{http://ctan.org/pkg/microtype}}. This template
%doc% uses the rather conservative package parameters: \texttt{protrusion=true,factor=900}.
\usepackage[protrusion=true,factor=900]{microtype}

%doc%
%doc% \subsection{French spacing}
%doc%
%doc% \paragraph{Why?} see~\textcite[p.\,28, p.\,30]{Bringhurst1993}: `2.1.4 Use a single word space between sentences.'
%doc%
%doc% \paragraph{How?} see~\textcite[p.\,185]{Eijkhout2008}:\\
%doc% \verb#\frenchspacing  %% Macro to switch off extra space after punctuation.# \\
\frenchspacing  %% Macro to switch off extra space after punctuation.
%doc%
%doc% Note: This setting might be default for \myacro{KOMA} script.
%doc%


%doc%
%doc% \subsection{Font}
%doc%
%doc% This template is using the Palatino font (package \texttt{mathpazo}) which results
%doc% in a legible document and matching mathematical fonts for printout.
%doc%
%doc% It is highly recommended that you either stick to the Palatino font or use the
%doc% \LaTeX{} default fonts (by removing the package \texttt{mathpazo}).
%doc%
%doc% Chosing different fonts is not
%doc% an easy task. Please leave this to people with good knowledge on this subject.
%doc%
%doc% One valid reason to change the default fonts is when your document is mainly
%doc% read on a computer screen. In this case it is recommended to switch to a font
%doc% \textsf{which is sans-serif like this}. This template contains several alternative
%doc% font packages which can be activated in this file.
%doc%

% for changing the default font, please go to the next subsection!

%doc%
%doc% \subsection{Text figures}
%doc%
%doc% \ldots also called old style numbers such as 0123456789.
%doc% (German: \enquote{Mediäval\-ziffern\footnote{\url{https://secure.wikimedia.org/wikibooks/de/wiki/LaTeX-W\%C3\%B6rterbuch:\_Medi\%C3\%A4valziffern}}})
%doc% \paragraph{Why?} see~\textcite[p.\,44f]{Bringhurst1993}:
%doc% \begin{quote}
%doc% `3.2.1 If the font includes both text figures and titling figures, use
%doc%  titling figures only with full caps, and text figures in all other
%doc%  circumstances.'
%doc% \end{quote}
%doc%
%doc% \paragraph{How?}
%doc% Quoted from Wikibooks\footnote{\url{https://secure.wikimedia.org/wikibooks/en/wiki/LaTeX/Formatting\#Text\_figures\_.28.22old\_style.22\_numerals.29}}:
%doc% \begin{quote}
%doc% Some fonts do not have text figures built in; the textcomp package attempts to
%doc% remedy this by effectively generating text figures from the currently-selected
%doc% font. Put \verb#\usepackage{textcomp}# in your preamble. textcomp also allows you to
%doc% use decimal points, properly formatted dollar signs, etc. within
%doc% \verb#\oldstylenums{}#.
%doc% \end{quote}
%doc% \ldots but proposed \LaTeX{} method does not work out well. Instead use:\\
%doc% \verb#\usepackage{hfoldsty}#  (enables text figures using additional font) or \\
%doc% \verb#\usepackage[sc,osf]{mathpazo}# (switches to Palatino font with small caps and old style figures enabled).
%doc%
%\usepackage{hfoldsty}  %% enables text figures using additional font
%% ... OR use ...
\usepackage[sc,osf]{mathpazo} %% switches to Palatino with small caps and old style figures

%% Font selection from:
%%     http://www.matthiaspospiech.de/latex/vorlagen/allgemein/preambel/fonts/
%% use following lines *instead* of the mathpazo package above:
%% ===== Serif =========================================================
%% for Computer Modern (LaTeX default font), simply remove the mathpazo above
%\usepackage{charter}\linespread{1.05} %% Charter
%\usepackage{bookman}                  %% Bookman (laedt Avant Garde !!)
%\usepackage{newcent}                  %% New Century Schoolbook (laedt Avant Garde !!)
%% ===== Sans Serif ====================================================
%\renewcommand{\familydefault}{\sfdefault}  %% this one in *combination* with the default mathpazo package
%\usepackage{cmbright}                  %% CM-Bright (eigntlich eine Familie)
%\usepackage{tpslifonts}                %% tpslifonts % Font for Slides


%doc%
%doc% \subsection{\texttt{myacro} --- Abbrevations using \textsc{small caps}}\myinteresting
%doc% \label{sec:myacro}
%doc%
%doc% \paragraph{Why?} see~\textcite[p.\,45f]{Bringhurst1993}: `3.2.2 For abbrevations and
%doc% acronyms in the midst of normal text, use spaced small caps.'
%doc%
%doc% \paragraph{How?} Using the predefined macro \verb#\myacro{}# for things like
%doc% \myacro{UNO} or \myacro{UNESCO} using \verb#\myacro{UNO}# or \verb#\myacro{UNESCO}#.
%doc%
\DeclareRobustCommand{\myacro}[1]{\textsc{\lowercase{#1}}} %%  abbrevations using small caps


%doc%
%doc% \subsection{Colorized headings and links}
%doc%
%doc% This document template is able to generate an output that uses colorized
%doc% headings, captions, page numbers, and links. The color named `DispositionColor'
%doc% used in this document is defined near the definition of package \texttt{color}
%doc% in the preamble (see section~\ref{subsec:miscpackages}). The changes required
%doc% for headings, page numbers, and captions are defined here.
%doc%
%doc% Settings for colored links are handled by the definitions of the
%doc% \texttt{hyperref} package (see section~\ref{sec:pdf}).
%doc%
\setheadsepline{.4pt}[\color{DispositionColor}]
\renewcommand{\headfont}{\normalfont\sffamily\color{DispositionColor}}
\renewcommand{\pnumfont}{\normalfont\sffamily\color{DispositionColor}}
\addtokomafont{disposition}{\color{DispositionColor}}
\addtokomafont{caption}{\color{DispositionColor}\footnotesize}
\addtokomafont{captionlabel}{\color{DispositionColor}}

%doc%
%doc% \subsection{No figures or tables below footnotes}
%doc%
%doc% \LaTeX{} places floating environments below footnotes if \texttt{b}
%doc% (bottom) is used as (default) placement algorithm. This is certainly
%doc% not appealing for most people and is deactivated in this template by
%doc% using the package \texttt{footmisc} with its option \texttt{bottom}.
%doc%
%% see also: http://www.komascript.de/node/858 (German description)
\usepackage[bottom]{footmisc}

%doc%
%doc% \subsection{Spacings of list environments}
%doc%
%doc% By default, \LaTeX{} is using vertical spaces between items of enumerate,
%doc% itemize and description environments. This is fine for multi-line items.
%doc% Many times, the user does just write single-line items where the larger
%doc% vertical space is inappropriate. The \href{http://ctan.org/pkg/enumitem}{enumitem}
%doc% package provides replacements for the pre-defined list environments and
%doc% offers many options to modify their appearances.
%doc% This template is using the package option for \texttt{noitemsep} which
%doc% mimimizes the vertical space between list items.
%doc%
\usepackage{enumitem}
%\setlist{noitemsep}   %% kills the space between items

%doc%
%doc% \subsection{\texttt{csquotes} --- Correct quotation marks}\myinteresting
%doc% \label{sub:csquotes}
%doc%
%doc% \emph{Never} use quotation marks found on your keyboard.
%doc% They end up in strange characters or false looking quotation marks.
%doc%
%doc% In \LaTeX{} you are able to use typographically correct quotation marks. The package
%doc% \href{http://www.ctan.org/pkg/csquotes}{\texttt{csquotes}} offers you with
%doc% \verb#\enquote{foobar}# a command to get correct quotation marks around \enquote{foobar}.
%doc% Please do check the package options in order to modify
%doc% its settings according to the language used\footnote{most of the time in
%doc% combination with the language set in the options of the \texttt{babel} package}.
%doc%
%doc% \href{http://www.ctan.org/pkg/csquotes}{\texttt{csquotes}} is also recommended
%doc% by \texttt{biblatex} (see Section~\ref{sec:references}).
\usepackage[babel=true,strict=true,english=american,german=guillemets]{csquotes}

%doc%
%doc% \subsection{Line spread}
%doc%
%doc% If you have to enlarge the distance between two lines of text, you can
%doc% increase it using the \texttt{\mylinespread} command in \texttt{main.tex}. By default, it is
%doc% deactivated (set to 100~percent). Modify only with caution since it influences the
%doc% page layout and could lead to ugly looking documents.
\linespread{\mylinespread}

%doc%
%doc% \subsection{Optional: Lines above and below the chapter head}
%doc%
%doc% This is not quite something typographic but rather a matter of taste.
%doc% \myacro{KOMA} Script offers \href{http://www.komascript.de/node/24}{a method to
%doc% add lines above and below chapter head} which is disabled by
%doc% default. If you want to enable this feature, remove corresponding
%doc% comment characters from the settings.
%doc%
%% Source: http://www.komascript.de/node/24
%disabled% %% 1st get a new command
%disabled% \newcommand*{\ORIGchapterheadstartvskip}{}%
%disabled% %% 2nd save the original definition to the new command
%disabled% \let\ORIGchapterheadstartvskip=\chapterheadstartvskip
%disabled% %% 3rd redefine the command using the saved original command
%disabled% \renewcommand*{\chapterheadstartvskip}{%
%disabled%   \ORIGchapterheadstartvskip
%disabled%   {%
%disabled%     \setlength{\parskip}{0pt}%
%disabled%     \noindent\color{DispositionColor}\rule[.3\baselineskip]{\linewidth}{1pt}\par
%disabled%   }%
%disabled% }
%disabled% %% see above
%disabled% \newcommand*{\ORIGchapterheadendvskip}{}%
%disabled% \let\ORIGchapterheadendvskip=\chapterheadendvskip
%disabled% \renewcommand*{\chapterheadendvskip}{%
%disabled%   {%
%disabled%     \setlength{\parskip}{0pt}%
%disabled%     \noindent\color{DispositionColor}\rule[.3\baselineskip]{\linewidth}{1pt}\par
%disabled%   }%
%disabled%   \ORIGchapterheadendvskip
%disabled% }

%doc%
%doc% \subsection{Optional: Chapter thumbs}
%doc%
%doc% This is not quite something typographic but rather a matter of taste.
%doc% \myacro{KOMA} Script offers \href{http://www.komascript.de/chapterthumbs-example}{a method to
%doc% add chapter thumbs} (in combination with the package \texttt{scrpage2}) which is disabled by
%doc% default. If you want to enable this feature, remove corresponding
%doc% comment characters from the settings.
%doc%
%disabled% \makeatletter
%disabled% % Safty first
%disabled% \@ifundefined{chapter}{\let\chapter\undefined
%disabled%   \chapter must be defined to use chapter thumbs!}{%
%disabled%
%disabled% % Two new commands for the width and height of the boxes with the
%disabled% % chapter number at the thumbs (use of commands instead of lengths
%disabled% % for sparing registers)
%disabled% \newcommand*{\chapterthumbwidth}{2em}
%disabled% \newcommand*{\chapterthumbheight}{1em}
%disabled%
%disabled% % Two new commands for the colors of the box background and the
%disabled% % chapter numbers of the thumbs
%disabled% \newcommand*{\chapterthumbboxcolor}{black}
%disabled% \newcommand*{\chapterthumbtextcolor}{white}
%disabled%
%disabled% % New command to set a chapter thumb. I'm using a group at this
%disabled% % command, because I'm changing the temporary dimension \@tempdima
%disabled% \newcommand*{\putchapterthumb}{%
%disabled%   \begingroup
%disabled%     \Large
%disabled%     % calculate the horizontal possition of the right paper border
%disabled%     % (I ignore \hoffset, because I interprete \hoffset moves the page
%disabled%     % at the paper e.g. if you are using cropmarks)
%disabled%     \setlength{\@tempdima}{\@oddheadshift}% (internal from scrpage2)
%disabled%     \setlength{\@tempdima}{-\@tempdima}%
%disabled%     \addtolength{\@tempdima}{\paperwidth}%
%disabled%     \addtolength{\@tempdima}{-\oddsidemargin}%
%disabled%     \addtolength{\@tempdima}{-1in}%
%disabled%     % putting the thumbs should not change the horizontal
%disabled%     % possition
%disabled%     \rlap{%
%disabled%       % move to the calculated horizontal possition
%disabled%       \hspace*{\@tempdima}%
%disabled%       % putting the thumbs should not change the vertical
%disabled%       % possition
%disabled%       \vbox to 0pt{%
%disabled%         % calculate the vertical possition of the thumbs (I ignore
%disabled%         % \voffset for the same reasons told above)
%disabled%         \setlength{\@tempdima}{\chapterthumbwidth}%
%disabled%         \multiply\@tempdima by\value{chapter}%
%disabled%         \addtolength{\@tempdima}{-\chapterthumbwidth}%
%disabled%         \addtolength{\@tempdima}{-\baselineskip}%
%disabled%         % move to the calculated vertical possition
%disabled%         \vspace*{\@tempdima}%
%disabled%         % put the thumbs left so the current horizontal possition
%disabled%         \llap{%
%disabled%           % and rotate them
%disabled%           \rotatebox{90}{\colorbox{\chapterthumbboxcolor}{%
%disabled%               \parbox[c][\chapterthumbheight][c]{\chapterthumbwidth}{%
%disabled%                 \centering
%disabled%                 \textcolor{\chapterthumbtextcolor}{%
%disabled%                   \strut\thechapter}\\
%disabled%               }%
%disabled%             }%
%disabled%           }%
%disabled%         }%
%disabled%         % avoid overfull \vbox messages
%disabled%         \vss
%disabled%       }%
%disabled%     }%
%disabled%   \endgroup
%disabled% }
%disabled%
%disabled% % New command, which works like \lohead but also puts the thumbs (you
%disabled% % cannot use \ihead with this definition but you may change this, if
%disabled% % you use more internal scrpage2 commands)
%disabled% \newcommand*{\loheadwithchapterthumbs}[2][]{%
%disabled%   \lohead[\putchapterthumb#1]{\putchapterthumb#2}%
%disabled% }
%disabled%
%disabled% % initial use
%disabled% \loheadwithchapterthumbs{}
%disabled% \pagestyle{scrheadings}
%disabled%
%disabled% }
%disabled% \makeatother

%%%% END
%%% Local Variables:
%%% mode: latex
%%% mode: auto-fill
%%% mode: flyspell
%%% eval: (ispell-change-dictionary "en_US")
%%% TeX-master: "../main"
%%% End:
%% vim:foldmethod=expr
%% vim:fde=getline(v\:lnum)=~'^%%%%'?0\:getline(v\:lnum)=~'^%doc.*\ .\\%(sub\\)\\?section{.\\+'?'>1'\:'1':



%% ========================================================================
%%%% MISC usepackages
%% ========================================================================
\usepackage{amsmath}
\usepackage{tikz}
\usepackage{caption}
\usepackage{subcaption}
\usepackage{xfrac}
\usepackage{siunitx}
\sisetup{output-exponent-marker=\ensuremath{\mathrm{e}}}




%% ========================================================================
%%%% MISC self-defined commands and settings
%% ========================================================================
\newcommand{ \prob}[1]{ \mathbb{P}[{#1}] }
\newcommand{ \expval}[1]{ \mathbb{E}[{#1}] }

\DeclareMathOperator*{\argmax}{arg\,max}
%% ... it's OK to put here your own newcommand/newenvironment-definitions ...




\newcommand{\myLaT}{\LaTeX{}@TUG\xspace} %% LaTeX@TUG text "logo"

\hyphenation{ex-am-ple hy-phen-ate}  %% in order to use German umlauts
%% here (Ver-\"of-fent-li-chung), you have to check for
%% activated \usepackage[T1]{fontenc} in the preamble

%% override default language of babel: (be sure to know, what you're
%% doing here)
%\selectlanguage{american}
%\selectlanguage{ngerman}

%% ========================================================================
%%%% Templates
%% ========================================================================

%% template for inserting figures:
% \myfig{}%% filename
%       {}%% width/height
%       {}%% caption
%       {}%% optional (short) caption for list of figures
%       {fig:}%% label

%% acronyms in small caps: \myacro{UNESCO}


%%%% Time-stamp: <2014-03-23 13:40:59 vk>
%%%% === Disclaimer: =======================================================
%% created by
%%
%%      Karl Voit
%%
%% using GNU/Linux, GNU Emacs & LaTeX 2e
%%

%doc%
%doc% \section{\texttt{pdf\_settings.tex} --- Settings related to PDF output}
%doc% \label{sec:pdf}
%doc% 
%doc% The file \verb#template/pdf_settings.tex# basically contains the definitions for
%doc% the \href{http://tug.org/applications/hyperref/}{\texttt{hyperref} package}
%doc% including the
%doc% \href{http://www.ctan.org/tex-archive/macros/latex/required/graphics/}{\texttt{graphicx}
%doc% package}. Since these settings should be the last things of any \LaTeX{}
%doc% preamble, they got their own \TeX{} file which is included in \texttt{main.tex}.
%doc% 
%doc% \paragraph{What should I do with this file?} The settings in this file are
%doc% important for \myacro{PDF} output and including graphics. Do not exclude the
%doc% related \texttt{input} command in \texttt{main.tex}. But you might want to
%doc% modify some settings after you read the
%doc% \href{http://tug.org/applications/hyperref/}{documentation of the \texttt{hyperref} package}.
%doc% 


%% Fix positioning of images in PDF viewers. (disabled by
%% default; see https://github.com/novoid/LaTeX-KOMA-template/issues/4
%% for more information) 
%% I do not have time to read about possible side-effect of this
%% package for now.
% \usepackage[hypcap]{caption}

%% Declarations of hyperref should be the last definitions of the preamble:
%% FIXXME: black-and-white-version for printing!

\pdfcompresslevel=9

\usepackage[%
unicode=true, % loads with unicode support
%a4paper=true, %
pdftex=true, %
backref, %
pagebackref=false, % creates backward references too
bookmarks=false, %
bookmarksopen=false, % when starting with AcrobatReader, the Bookmarkcolumn is opened
pdfpagemode=None,% None, UseOutlines, UseThumbs, FullScreen
plainpages=false, % correct, if pdflatex complains: ``destination with same identifier already exists''
%% colors: https://secure.wikimedia.org/wikibooks/en/wiki/LaTeX/Colors
urlcolor=DispositionColor, %%
linkcolor=DispositionColor, %%
pagecolor=DispositionColor, %%
citecolor=DispositionColor, %%
anchorcolor=DispositionColor, %%
colorlinks=\mycolorlinks, % turn on/off colored links (on: better for
                          % on-screen reading; off: better for printout versions)
]{hyperref}

%% all strings need to be loaded after hyperref was loaded with unicode support
%% if not the field is garbled in the output for characters like ČŽĆŠĐ
\hypersetup{
pdftitle={\mytitle}, %
pdfauthor={\myauthor}, %
pdfsubject={\mysubject}, %
pdfcreator={Accomplished with: pdfLaTeX, biber, and hyperref-package. No animals, MS-EULA or BSA-rules were harmed.},
pdfproducer={\myauthor},
pdfkeywords={\mykeywords}
}

%\DeclareGraphicsExtensions{.pdf}

%%%% END
%%% Local Variables:
%%% TeX-master: "../main"
%%% mode: latex
%%% mode: auto-fill
%%% mode: flyspell
%%% eval: (ispell-change-dictionary "en_US")
%%% End:
%% vim:foldmethod=expr
%% vim:fde=getline(v\:lnum)=~'^%%%%'?0\:getline(v\:lnum)=~'^%doc.*\ .\\%(sub\\)\\?section{.\\+'?'>1'\:'1':
  %% should be *last* definitions in preamble!
%% ========================================================================
%%%% begin{document}
%% ========================================================================
\begin{document}

\frontmatter                    %% KOMA: roman page numbers and such; only available in scrbook

%%%% Time-stamp: <2013-03-18 14:35:00 vk>
%% ========================================================================
%%%% Disclaimer
%% ========================================================================
%%
%% created by
%%
%%      Karl Voit


\newcommand{\mycolophon}{%%
  This document 
  %% was written with \myacro{GNU}~Emacs, 
  is set in Palatino, compiled with
  \href{http://LaTeX.TUGraz.at}{pdf\LaTeX2e} and
  \href{http://en.wikipedia.org/wiki/Biber_(LaTeX)}{\texttt{Biber}}.

  The \LaTeX{} template from Karl Voit is based on
  \href{http://www.komascript.de/}{KOMA script} and can be found 
  online: \href{https://github.com/novoid/LaTeX-KOMA-template}{https://github.com/novoid/LaTeX-KOMA-template}
}


%%% Local Variables: 
%%% mode: latex
%%% mode: auto-fill
%%% mode: flyspell
%%% eval: (ispell-change-dictionary "en_US")
%%% TeX-master: "../main"
%%% End: 


%% Choose your desired title page:
\input{\mytitlepage}            %% include title page


%%%% Time-stamp: <2017-02-05 16:00:44 vk>
%% ========================================================================
%%%% Disclaimer
%% ========================================================================
%%
%% created by
%%
%%      Karl Voit
%%

\section*{Affidavit}

I declare that I have authored this thesis independently, that I have
not used other than the declared sources/resources, and that I have
explicitly indicated all material which has been quoted either
literally or by content from the sources used. The text document
uploaded to \myacro{TUGRAZ}online is identical to the present master‘s
thesis.

\vfill

%% definition of the block tat contains date and signature
\newcommand{\mysignatureblock}[3]{%
  %% Sorry, this is a "bit" of a hack. Maybe someone knows a more elegant method?
  \begin{tabular}{llp{2em}l}
  #1 & \hspace{5cm}        & & \hspace{6cm} \\\cline{2-2}\cline{4-4}
     &                     & & \\[-3mm]
     & {\footnotesize #2}  & & {\footnotesize #3}
  \end{tabular}
}

\mysignatureblock{Graz,}{Date}{Signature}

\vfill
\vfill
\vfill
\vfill

\section*{Eidesstattliche Erklärung}

\foreignlanguage{ngerman}{%
  Ich erkläre an Eides statt, dass ich die vorliegende Arbeit
  selbstständig verfasst, andere als die angegebenen
  Quellen/Hilfsmittel nicht benutzt, und die den benutzten Quellen
  wörtlich und inhaltlich entnommenen Stellen als solche kenntlich
  gemacht habe. Das in \myacro{TUGRAZ}online hochgeladene Textdokument
  ist mit der vorliegenden Dissertation identisch.}

\vfill

\mysignatureblock{Graz, am}{Datum}{Unterschrift}


\newpage
\newpage

%%% Local Variables:
%%% mode: latex
%%% mode: auto-fill
%%% mode: flyspell
%%% TeX-master: "../main"
%%% End:
  %% Statutory Declaration
% \input{thanks}                %% this is a suggestion: you have to create this file on demand
% \input{foreword}              %% this is a suggestion: you have to create this file on demand


%% include the abstract without chapter number but include it on table of contents:
\cleardoublepage
\addcontentsline{toc}{chapter}{Abstract}
%%%% Time-stamp: <2013-02-25 10:31:01 vk>


\chapter*{Abstract}
\label{cha:abstract}


This is a placeholder for the abstract. It summarizes the whole thesis
to give a very short overview. Usually, this the abstract is written
when the whole thesis text is finished.



%\glsresetall %% all glossary entries should be used in long form (again)
%% vim:foldmethod=expr
%% vim:fde=getline(v\:lnum)=~'^%%%%\ .\\+'?'>1'\:'='
%%% Local Variables:
%%% mode: latex
%%% mode: auto-fill
%%% mode: flyspell
%%% eval: (ispell-change-dictionary "en_US")
%%% TeX-master: "main"
%%% End:
              %% Abstract


\tableofcontents                %% this produces the table of contents - you might have guessed :-)

\listoffigures
\listoftables

%% if myaddlistoftodos is set to "true", the current list of open todos is added:
\ifthenelse{\boolean{myaddlistoftodos}}{
  \newpage\listoftodos          %% handy if you are using todonotes with \todo{}
}{}                             %% with todonotes-package option "disable" you can get rid of any todo in the output

\mainmatter                     %% KOMA: marks main part using arabic page numbers and such; only available in scrbook

%% include tex file chapters:
\chapter{Introduction}
\label{cha:introduction}


\section{Motivation}
\label{sec:motivation}

Plenty of problems that arise in the real world can be solved by first abstracting them in a more general form and solve them by using already established and well-known tools afterwards.
One very famous example for this process is attributed to the Swiss mathematician Leonhard Euler.
The city Königsberg (now called Kaliningrad) was split into four parts in the 18\textsuperscript{th} century, due to the pathway of the river Pregel.
These four areas of the city were connected by seven bridges.
The residents of Königsberg had an ongoing challenge to find a way through the city that crosses each bridge exactly once and ends once more at the starting point of the tour.
This is known as the Königsberg bridge problem~\cite{Paoletti2011, Cook2012}.
Euler tackled this challenge by eliminating irrelevant details (e.g., the length of the brides, or the size of the size of the areas) and abstracting the problem to its essence.
The regions of the city became points and the bridges between two areas became line segments that connect the corresponding points.
This abstracted topological view on the problem allowed Euler to solve the Königsberg problem and show that in fact no such tour through the city (also known as an Eulerian cycle) exists.
For it to be possible requires the number of bridges that are accessible in each region to be even, which is not the case for the Königsberg problem, where the number is odd in each region.

This particular way of abstracting real world objects and their connections into points and lines and applying mathematical methods and reasoning to it laid the foundation for the mathematical field of graph theory.
It is used today in countless applications, from finding efficient ways to manufacture circuit boards~\cite{Cook2012} to calculating fast routes for the data packets that are sent over the internet~\cite{Wang1999}.
Graph theory also spawned the field of networks science~\cite{Newman2010}, which deals with large real world systems that can be modeled using graphs.
For instance, large infrastructure networks, such as power grids can be represented by graphs.
These graphs can relate parts of the infrastructure, like power stations or transformers, that are connected to each other via power lines~\cite{Watts1998}.
Another field that benefits from this as well is sociology, and in particular social network analysis, which investigates the complex behavior of people and groups in the context of social interactions~\cite{Newman2010}.
Sociological studies are often performed with small sample sizes, since the collection of the required data is time consuming and usually done by hand using methods like questionnaires, interviews, or by simply observing the people that are part of the study~\cite{Wasserman1994}.
However, ever since the emerge of the web, people can interact with each other more easily.
Websites like online social networks (e.g., Facebook, Twitter, Reddit,\ldots), or collaboration forums (e.g., StackOverflow) seem to be very popular.
Most of these websites are ranked on top positions on Alexas' list of top 500 visited websites globally~\cite{Alexa2017}.
\citet{Wellman2001} showed that these websites (and the internet in general) does not increase or decrease the social capital of people (i.e., their relationships with friends and family or their commitment to participate in organizations), but supplements it by providing easier ways to organize and plan real-world activities in an online setting.

This availability of large amounts of data that are generated on these websites can be beneficial in the context of social network analysis.
For example, StackExchange, a website that maintains a variety of communities in which users can ask and answer questions to different topics (e.g., StackOverflow for programming related questions), provides an easy access\footnote{\url{https://data.stackexchange.com/}} to the user data.
This data is used in many studies on a wide range of topics (e.g.,~\cite{Danescu2013,Walk2016, Hasani-Mavriqi2016}).
A large mobile phone calls (MPC) data set is used in a variety of papers~\cite{Onnela2007, Karsai2014, Murase2015, Laurent2015} as well.
It was collected by an European mobile phone provider with approximately 20\% market share and consists of over 630 million logged phone calls between more than six million people.
Another interesting data set was used in the study by \citet{Sekara2016}, in which they propose a framework to describe gatherings of people and their (temporal) properties.
The data set consists of data from various sources for 1,000 people that was collected over a period of 36 months in short intervals (i.e., with a five minute resolution).
The data set contains information about the phone calls, text messages, social media activity, geolocation, and the proximity to other study participants for every subject.

All of these large data sets share another crucial feature: they all include temporal information.
Every post, tweet, phone call, or text message has a timestamp attached to it.
This allows to pin these activities to users at specific points in time and to infer a chronological order between them.
However, not all studies include this additional information into their work.
The reasons for this can vary.
For instance, some use-cases only require quantitative information (i.e., how often something happens between two objects) and not exactly when it did  (e.g.,~\cite{Kumpula2007, Bagler2008}).
This corresponds to the elimination of irrelevant details in the abstraction of a problem.
Another reason is that graphs by them self are not able to include the temporal information, since they only represent static relationships between objects.
However, there is an extension that provides the possibility to integrate the time information when required.
This type of graphs are called temporal or time-varying networks~\cite{Holme2012, Holme2015} and extent graphs in a way that relationships between objects become time depended.
This basically means that the connection between objects is only present at certain points in time, which leads to a more realistic but also more complex abstraction.

An area that definitively requires the incorporation of time information is the modeling of human behavioral patterns.
It has been shown that activities performed by people (e.g., the writing of e-mails, text messages, tweets,\ldots) are not randomly distributed in time, but follow certain patterns instead~\cite{Barabasi2005}.
Human activity can usually be described as bursty.
For instance, it is evident that people tend to write multiple e-mails in a relatively short period of time.
This high activity phase is then generally followed by longer periods of inactivity, in which, for example, no e-mails are written at all.
The best way to describe these patterns is by using a probability distribution of the times between two consecutive activities (i.e., the inter-event time distribution).
The distribution is characterized by its high probabilities for short inter-event times and its long tail that allows for the longer phases with no activity, which can be modeled using a power-law distribution of the form \( p(\tau) \sim \tau^{-\gamma} \).

Inter-event time distributions are also relevant in the context of time-varying networks.
\citet{Lambiotte2013} study in their work the effects of the inter-event time distribution of the link activations on dynamic spreading processes in temporal networks.
The framework proposed by \citet{Perra2012a} shifts the focus from the activation of the connections to the activations of the objects.
It is based on the simple idea that each entity in the network can become active based on some inherent activity potential and subsequently connect to others in each time step.
Therefore, it can, for example, be used to model the activity of users in an entire social network.
However, this model has some disadvantages and due to its simplicity and unrealistic assumptions.
For example, it is not able to reproduce inter-event time distributions of the activations and topological properties of the network that have the same characteristics as observed in so many real-world networks.
Nevertheless, there are a lot of extensions to this basic framework that address these issues (e.g.,~\cite{Laurent2015, Moinet2015, Moinet2016}).
Overall initiated this activity-driven temporal network model a wide range of followup work in many different areas.

It was the foundation that started this thesis as well.
All models that originated from the original paper by \citet{Perra2012a} so far are based on the idea that entities can only become active due to their intrinsic activity potential.
 However, in the real world people are often heavily influenced by their peers and friends~\cite{Walk2016}.
They are not doing things solely because of their own determination to do so, but also because their friends and the associated peer pressure.
Therefore, in the context of social and collaboration networks could more active user help to motivate their peers to become active as well.
The aim of this work is it to define a model based on the activity-driven time-varying network framework that is able to include peer influence effects in the activation process of entities and to study the implication of this mechanism of the network in general.


%% ========================================================================
%% ========================================================================


\section{Outline}
\label{sec:outline}

The content of this thesis is structured as follows.
\Cref{cha:related-work} starts with some basic graph-theoretic definitions (\cref{sec:graph-theory-basics}) and a detailed introduction into the topics of social and temporal networks (\cref{sec:social-networks} and \cref{sec:time-varying-networks} respectively).
Additionally, an overview of some important generative network models and their properties is given in \cref{sec:network-models}.
 \Cref{sec:user-activity-models} discusses related work in the context of user behavioral models.
 It provides a possible explanation for the origin of bursty human behavior with the queuing model and provides an overview of activity models that are based on time-varying networks, with focus on work based on the activity-driven temporal network framework.
The last section (\cref{sec:peer-influence}) of this chapter is related to peer influence and its ramifications and applications in different fields.

In \cref{cha:model} the proposed time-varying peer influence network model is discussed in great detail.
First, the model and the ideas on which it is based are outlined in \cref{sec:base-model}.
The extension that incorporates peer influence effects into the base model and the idea behind it is described in \cref{sec:peer-influence-model}.
\Cref{cha:results} contains the evaluation and analysis of the proposed model on synthetic networks.

The last chapter of this document (\cref{cha:conclusion}) includes a summary of the archived results and a conclusion.
Furthermore, different applications and possible extensions of the peer influence model are discussed in the last section of this thesis.

\chapter{Related Work}


\section{Graph Theory Basics}
\label{sec:graph-theory-basics}

In this section some graph theory related notation is defined which is used during this thesis.
It is for the most part based on~\cite{Thulasiraman1992} and on~\cite{Diestel2012}.
A graph is a mathematical construct that can be used to model and explore the relationship between objects.
More formally, a graph is a ordered pair of finite sets \(G = (V, E)\), whereas \(V\) denotes the set of \emph{vertices} (i.e., the objects) and \(E \subseteq [V]^{2} \) the set of \emph{edges} (i.e., the relationships between the objects).
It is common to write \(V(G)\) and \(E(G)\) to refer to the set of vertices and the set of edges, respectively, that are associated with a graph \(G\).
An edge \(\{v_1, v_2\} \in E(G)\) is an unordered pair of two vertices.
This means that there is no distinction between the two edges \(\{v_1, v_2 \}\) and \(\{v_2, v_1\}\).
A graph with this property is called a \emph{undirected} graph.
However, it is also possible to define edges as ordered pairs, so that each edge does have a start- and endpoint.
Such a graph is called a \emph{directed} graph.
An edge of the form \(\{v_i, v_i\} \in E(G)\) is called a \emph{self-loop} of the vertex \(v_i\).
Furthermore, it is possible that two distinct vertices are joined by multiple edges.
Such edges are referred to as \emph{parallel} edges.
A graph that has no parallel edges and no self-loops is called a \emph{simple} graph.
Figure~\ref{fig:example_graphs} depicts a example for a simple graph and for a graph with multiple edges and a self-loop vertex.
It is also possible to assign each edge in a graph a real number, that adds additional information to it~\cite{Cormen2009}.
This number is usually called \emph{weight}, and is determined by the function \(w : E(G) \rightarrow \mathbb{R}\).
This type of graph is called weighted graph.
However, all further mentions and definitions for graphs are referring to undirected simple graphs unless stated otherwise.
It is also possible to perform operations on graphs.
For instance, the union of two graphs \(G_{1} = (V_{1}, E_{1})\) and \(G_{2} = (V_{2}, E_{2})\) results in a graph \(G = (V_{1} \cup V_{2}, E_{1} \cup E_{2})\).
Other binary operations, such as the intersection of two graphs, can be done analogous.
There are unary operations on graphs (e.g., the removal of vertices and edges) as well.

\begin{figure}[h]
   \centering
   \begin{subfigure}[t]{0.45\textwidth}
     \centering
     \begin{tikzpicture}[node/.style={circle,fill=red!70,minimum size=1em,inner sep=3pt]}]
       \node[node] (1) at (0, 0) {};
       \node[node] (2) at (-1, -1.5)  {};
       \node[node] (3) at (1, -1.5) {};
       \node[node] (4) at (-1, -3) {};
       \node[node] (5) at (1, -3) {};

       \draw (1) -- (3);
       \draw (1) -- (2) -- (4) -- (5) -- (3) -- (2);
     \end{tikzpicture}
     \caption{A undirected simple graph.}
   \end{subfigure}
   ~
   \begin{subfigure}[t]{0.45\textwidth}
     \centering
     \begin{tikzpicture}[every loop/.style={}, node/.style={circle,fill=red!70,minimum size=1em,inner sep=3pt]}]
       \node[node] (1) at (0, 0) {};
       \node[node] (2) at (-1, -1.5)  {};
       \node[node] (3) at (1, -1.5) {};
       \node[node] (4) at (0, -3) {};

       \draw (1) -- (2);
       \draw (1) -- (3) -- (4);
       \path (2) edge [bend left] (4);
       \path (2) edge [bend right] (4);
       \path (3) edge [bend left] (1);
       \path (3) edge [bend right] (1);
       \draw (4) edge [in=-50,out=-130,loop] (4);
     \end{tikzpicture}
     \caption{A undirected graph with a self-loop and parallel edges.}
   \end{subfigure}

   \caption[Examples for graphs]{Graphical representation of two graphs with different properties.
   The vertices are represented by red dots and the edges are the line segments between them.}
\label{fig:example_graphs}
\end{figure}

The \emph{order} of a graph is its number of vertices (i.e., the cardinality of the vertex set) and is denoted as \(n = |V(G)|\).
The neighborhood of a vertex \(v_i\) is defined as \(N(v_i) = \{v_j \in V(G) : \{v_i, v_j \} \in E(G)\}\).
It is the set of vertices that are \emph{adjacent} to the vertex \(v_i\).
The cardinality of this set is called the \emph{degree} of the vertex and is denoted as \(d(v_i) = |N(v_i)|\).
A vertex without any neighbors (i.e., with a degree of zero) is called \emph{isolated}.
It is also often very useful to measure degree properties for the graph.
For example, the \emph{minimum degree} \(\delta(G) = \min\{d(v_i) : v_i \in V(G)\}\), the \emph{maximal degree} \(\Delta(G) = \max\{d(v_i) : v_i \in V(G)\}\), and the \emph{average degree} \(d(G) = \frac{1}{n} \sum_{v_i \in V(G)} d(v_i)\).
Another way to calculate the average degree is \(d(G) = \frac{2|E|}{n}\), due to the fact that each edge is counted twice during the summation of the vertex degrees.
These global measures can be used to get an insight in the basic structure of the graph.
Another global property related to the degree is the \emph{degree distribution} \(p\)~\cite{Barabasi2016} of a graph.
It yields the probability that that a randomly selected vertex has a degree of \(k\).
Since it is a probability distribution \(\sum_{k=0}^\infty p(k) = 1\) must hold.
The degree distribution for a given graph can be calculated by using \(p(k) = \frac{|\{v \in V(G) \,:\, d(v) = k\}|}{n}\) (i.e., calculating the normalized histogram).

A \emph{path} on a graph can be defined as a finite sequence of vertices \(v_1,v_2,\dots,v_k\), such that between any consecutive pair of vertices exist a edge in the graph.
Furthermore, all edges between the vertices and the vertices itself must be distinct.
The first and the last vertices in the sequence are called the \emph{end vertices} or \emph{terminal vertices} of the path.
The \emph{path length} is the number of edges on the path.
Two vertices are \emph{connected} if it is possible to find a path with these two vertices as end points.
A vertex is, by definition, connected to itself.
If there exists a path between all pairs of vertices, then the graph is called connected.
It is possible to partition the vertex and edge set of a not connected graph in such a way that there are no edges between vertices in different partitions.
These partitions are called the \emph{components} of the graph.

The \emph{clustering coefficient} of a vertex is a measure for the cliquishness of its neighborhood, and was introduced by \citet{Watts1998}.
A \emph{clique} in a graph is a subset of vertices, such that there exists an edge between every pair of vertices in this set.
The clustering coefficient \(C(v_i)\) is defined as the fraction of possible edges between the neighbors of the vertex \(v_i\).
There are at most \(\binom{d(v_i)}{2} = \frac{d(v_i)(d(v_i) - 1)}{2}\) possible edges between vertices in the neighborhood.
Therefore, the clustering coefficient can be calculated using \autoref{eq:clustering-coefficient}.

\begin{equation}
 C(v_i) = \frac{2 \, |\{\{v_j, v_k\} \in E(G) : v_j \in N(v_i) \wedge v_k \in N(v_i)\}| }{d(v_i)(d(v_i) - 1)}
 \label{eq:clustering-coefficient}
\end{equation}

This is, of course, a local property of one vertex and is, therefore, sometimes called \emph{local clustering coefficient}.
However, it is also often useful to consider the average clustering coefficient \(\bar{C} = \frac{1}{n} \sum_{v(i) \in V(G)} C(v_i)\) of the graph.
Figure~\ref{fig:clustering-coefficient-examples} shows some examples for neighborhoods with different clustering coefficients.
There is another definition of a \emph{global clustering coefficient}, which is also often called \emph{transitivity}~\cite{Boccaletti2006}.
It is the ratio of triangles (i.e., cliques consisting of exactly three vertices) to the number connected triples in the graph (see \autoref{eq:global-clustering-coefficient}).
A connected triple is made up of three vertices as well, but does only have two edges.
Hence, a triangle consists of exactly three triples.
The global clustering coefficient is a measure the extent of transitive connections in the graph (i.e., if there is a edge between vertices A and B, and between B and C, how likely is it that there is also an edge between A and C).
The two global clustering measures are, however, not equivalent to each other and may yield very different values~\cite{Newman2010}.
The definition of the local clustering clustering coefficient introduces a bias towards vertices with smaller degree, due to smaller values in the denominator.
Therefore, the value for the average local clustering coefficient maybe larger than the global clustering coefficient for the same graph, if the number of low-degree vertices is sufficiently large.

\begin{equation}
 T = \frac{3 \times \text{\# of triangles}}{\text{\# of connected triples of vertices}}
 \label{eq:global-clustering-coefficient}
\end{equation}

\begin{figure}[h]
   \centering
   \begin{subfigure}[t]{0.31\textwidth}
     \centering
     \begin{tikzpicture}[node/.style={circle,fill=red!70,minimum size=1em,inner sep=3pt]}, neighbor/.style={circle,fill=blue!70,minimum size=1em,inner sep=3pt]}]
       \node[text width=6em, align=center] at (0, 0.75)  {\(C(red) = 0\)};
       \node[node] (1) at (0, 0) {};
       \node[neighbor] (2) at (-1, -1)  {};
       \node[neighbor] (3) at (1, -1) {};
       \node[neighbor] (4) at (-1, -2)  {};
       \node[neighbor] (5) at (1, -2) {};

       \foreach \p in {2,3,4,5}{\draw (\p) -- (1); }
     \end{tikzpicture}
     \caption{In this example none of the four neighbors shares a edge with any other neighbor of the red vertex.
     Therefore, the clustering coefficient of the red vertex is \(\frac{0}{6} = 0\).}
   \end{subfigure}
   ~
   \begin{subfigure}[t]{0.31\textwidth}
     \centering
     \begin{tikzpicture}[node/.style={circle,fill=red!70,minimum size=1em,inner sep=3pt]}, neighbor/.style={circle,fill=blue!70,minimum size=1em,inner sep=3pt]}]
       \node[text width=6em, align=center] at (0, 0.75)  {\(C(red) = 0.5\)};
       \node[node] (1) at (0, 0) {};
       \node[neighbor] (2) at (-1, -1)  {};
       \node[neighbor] (3) at (1, -1) {};
       \node[neighbor] (4) at (-1, -2)  {};
       \node[neighbor] (5) at (1, -2) {};

       \foreach \p in {2,3,4,5}{\draw (\p) -- (1); }
       \draw (2) -- (4);
       \draw (3) -- (5);
       \draw (3) -- (2);
     \end{tikzpicture}
     \caption{Here are half of the possible edges between the neighbors are present.
     The clustering coefficient of the red vertex is \(\frac{3}{6} = \frac{1}{2}\).}
   \end{subfigure}
   ~
   \begin{subfigure}[t]{0.31\textwidth}
     \centering
     \begin{tikzpicture}[node/.style={circle,fill=red!70,minimum size=1em,inner sep=3pt]}, neighbor/.style={circle,fill=blue!70,minimum size=1em,inner sep=3pt]}]
       \node[text width=6em, align=center] at (0, 0.75)  {\(C(red) = 1\)};
       \node[node] (1) at (0, 0) {};
       \node[neighbor] (2) at (-1, -1)  {};
       \node[neighbor] (3) at (1, -1) {};
       \node[neighbor] (4) at (-1, -2)  {};
       \node[neighbor] (5) at (1, -2) {};

       \foreach \p in {1,2,3,4,5}{ \foreach \q in {1,2,3,4,5}{\draw (\p) -- (\q); }}
     \end{tikzpicture}
     \caption{The neighbors of the red vertex form a clique.
     Hence, the clustering coefficient of the red vertex is \(\frac{6}{6} = 1\).}
   \end{subfigure}

   \caption[Clustering coefficient examples]{Examples for the local clustering coefficient of a vertex with a small neighborhood.
   The blue vertices are the neighbors of the red vertex.
   The possible number of edges between the four neighbors is \(\binom{4}{2} = 6\).}
\label{fig:clustering-coefficient-examples}
\end{figure}


%% ========================================================================
%% ========================================================================


\section{Social Networks}
\label{sec:social-networks}

In the real-world it is common to use graphs to model the complex systems that are arising.
For example, the web can be represented as a graph, where the vertices correspond to websites and the edges are the hyperlinks between them.
Another use case are large infrastructure networks, such as power grids.
In a graph that represents such a power grid network vertices represent things like power stations or transformers and edges the power lines between them.
However, when modeling these large networks it common to use a slightly different terminology~\cite{Barabasi2016}.
Vertices are often called \emph{nodes} and edges are called \emph{links} in the context of networks.

Social networks~\cite{Newman2010} are another type of network that can benefit from the usage of graph theory methods.
The study of these networks is considered to be a part of the field of sociology and researchers may also use slightly different terminology for the vertices and edges in their work.
Nodes (or vertices) often represent people in social networks and are also sometimes referred to as \emph{actors}.
However, it is also possible that nodes depict other entities, such as a departments, companies, or countries (i.e., larger groups of people).
The links (or edges) between these entities can mean, depending of the context, different things as well and are sometimes also referred to as \emph{ties}.
For example, links between persons can show social relationships (e.g., friendships), collaborations in projects (e.g., co-authorship of a scientific papers), or other social interactions.
Links between companies could represent trading relationships or the like.
Often social networks are only mentioned in relation to large online communities, such as Facebook or Twitter, but there is no necessity that a social network must exist in an online form.
The network of acquaintances or friends in a school is also considered as a social network.
Figure~\ref{fig:karate-club-network} shows a famous example of a small real-world social network.

\myfig{karate_club_network}
      {width=0.6\textwidth,height=0.6\textheight}
      {Zachary's karate club network~\cite{Zachary1977} is a social network that shows the relationship between 34 members of a university-based karate club in the US in the early 1970's.
      There exists an edge between two members if there were social interactions outside of the normal club activities (i.e., two members are considered as friends).
      The graph shows a separation into two groups (red and blue nodes) due to a dispute in the club.}
      {Zachary's karate club network}
      {fig:karate-club-network}

Two fundamental terms in social network analysis are \emph{dyads} and \emph{triads}~\cite{Wasserman1994}.
These two concepts describe the relationship between multiple actors.
A \emph{dyad} denotes the linkage between a pair of actors.
It is a very common research topic to understand the pairwise relationships in social networks.
A \emph{triad} on the other hand describes triples of actors and the ties between them.
This concept is especially important for the question of transitivity of certain relationships.
A example for this would be the question \enquote{is the friend of my friend also my friend?}.
Another important concept is the \emph{strength of ties}, which was introduced by \citet{Granovetter1973}.
The strength of a tie is influenced by factors like the time that two actors spend together, or the intimacy between them.
Depended on the strength, a tie can either be strong, weak, or absent.
Absent does not only include non-existing ties, but also ties that are below some threshold of social interactions.
This concept of strong and weak ties can be used to explain the formation of triads.
If there already exists a strong tie between the two actors A and B, and between B and C, then there exists at least a weak tie between A and C.
This is true, due to the opportunities for the formation of a tie between A and C, that will result from a common strong tie to the actor B.
Furthermore, it is possible to explain the spreading of information in a social network in a better way by considering the strength of ties.
Weak ties can help the flow of information by acting as a bridge between sparsely connected parts of the network.

A property of many real-world social networks is a \emph{community structure}~\cite{Girvan2002}.
Communities are groups of actors that are more connected to actors in the same group, than to actors in a different one.
This basically means that there exists subsets of densely linked nodes in the network with very few links to other subsets.
The network shown in \autoref{fig:karate-club-network} has two known communities characterized by different node colors.
The detection of community structures in networks is a very import research topic, since the identified communities may correspond to actual social groupings.
For example, the detected communities in a social network that models the friendship between students may represent the real corresponding social groups (i.e., the circles of friends).
There is a variety of different methods and approaches to perform this task.
Examples are approaches based on hierarchical clustering, or edge betweenness (i.e., the number of shortest paths going through an edge)~\cite{Fortunato2010}.
However, often is it not possible to study the network and its structure on a fully global scale, due to processing limitations.
One alternative way is to look at single actors and their immediate neighborhood.
These small sub-networks are called \emph{ego-centered} or \emph{personal} networks~\cite{Newman2010}.
Often is some number of these ego-centered networks sampled from the complete networks and examined for local properties, such as the degrees or the local clustering coefficient.

Another attribute that many social networks share is that they are \emph{scale-free networks}~\cite{Barabasi2016}.
Formally, a network is a scale-free network if its degree distribution is a \emph{power-law distribution} (i.e., \(p(k) \sim k^{-\gamma}\), where \(\gamma\) is the parameter of the distribution that denotes the degree exponent).
The value for \(\gamma\) for most real-world (social) networks is in the range between 2 and 3.
A consequence of the power-law distribution is that the distribution of the degrees is right-skewed with a long tail.
This means that there is a large number of nodes in the network that have only a few links (i.e., a small degree) but there is also the chance that there exists a few nodes with a very high degree.
Such nodes are usually called \emph{hubs} and may correspond in the context of social networks to very influential actors that can play an important role in the network.


%% ========================================================================
%% ========================================================================


\section{Time-varying Networks}
\label{sec:time-varying-networks}

This section contains an overview on the concept of time-varying networks~\cite{Holme2012, Holme2015}.
Since this type of network is used in many different scientific fields it also has a variety of names.
For example, temporal networks, dynamic networks, evolving graphs, or the name that is mainly used in this thesis, time-varying networks.
As already mentioned in the section about social networks, the structure, or topology, of networks can be used to understand dynamic processes and their behavior.
However, there are many dynamical processes that are modeled using networks, in which the links are not active all the time.
One example would be a communication network, such as the network of phone calls between users.
Another example would social or collaboration network, where actors do not interact constantly but in irregular intervals.
These link activation at certain times can, however, be very important to explain the dynamic process and are simply lost when approximated by a static graph.
So the idea of time-varying networks is to introduce another dimension (i.e., time) to the network and move the information of when something happens from the dynamic process to the network itself.
As a general rule, a time-varying network is applicable when the structure (i.e., the topology) of the system and the temporal process are connected to each other.
This means that the time scale on which the network itself evolves should be similar to the time scale of the dynamic process that takes place on it.
For example, a time-varying network is not a suitable model for the internet, since the the infrastructure (i.e., the topology) changes very slowly in comparison to the transmission of the packages that are routed through the network (i.e., the dynamic process).

The underlying concept of time-varying networks is called \emph{contacts} and can be seen as interactions between two nodes at a certain time.
The duration of the interaction is negligible and thus assumed to be instantaneous.
Contacts can be interpreted as an extension of links in the static network.
The unordered pair \(\{v_{i}, v_{j}\}\) becomes a ordered triple \((v_{i}, v_{j}, t)\).
The order is in this case important, since the third object in the triple must refer to the interaction time \(t\) between the nodes \(v_{i}\) and \(v_{j}\).
The usage of static networks in models for dynamic processes, that represents some time-depended sequence of contacts between pairs of nodes, often results in a loss of information, that can be regained using time-varying networks.
However, these temporal networks also introduce more complexity to the model and one have to weight the gain of information versus the extra effort.
Especially in the case that the knowledge of how often something between two actors in the network happens is more important than when exactly something happens is a use case for weighted graph.
See \autoref{fig:weighted-network-example} for an illustration of a weighted graph.
However, a simple graph without weights can also be used to approximate the interaction sequence of a time-varying network~\cite{Holme2013}.
The idea is to calculate a total weight for each pair of nodes in the network.
If the weight exceeds a certain threshold \(\Omega \in \mathbb{R}_{0}^{+}\) then there will be an edge between the two nodes in the static approximation.
Each contact between two contacts in the sequence \(C\) contributes a part to the total weight.
However, the weight of the contribution to the total weight decays exponentially.
Hence, the total weight \(\omega_{i,j}\) between two nodes \(v_{i}\) and \(v_{j}\) is \(\omega_{i,j} = \sum_{(v_{i}, v_{j}, t) \in C} \exp(-t / \tau)\), where \(\tau\) is an additional parameter that controls the exponential decay.
For a static approximation that should contain an edge if there was at least one contact between the two nodes the threshold can be set to \(\Omega = 0\).
Networks that are generated using this approach are called exponential-threshold networks.

There are also many different possibilities to represent a temporal network without the loss of information.
The simplest way to to this is by using the actual contact sequences.
A contact sequence is basically a list that contains all the contact triples.
This is a very raw form data (i.e., in essence a spreadsheet with three columns) and is therefore very easy to parse and to use in algorithms, but is not very well suited for the analysis of the underlying dynamic process by humans due to the lack of illustrations.
Another way to represent time-varying networks are graph sequences.
The idea here is to generate a static graph that contains all contacts between nodes for a given time step.
This method has the advantage that all tools that work for static graphs can be applied to each of the graphs in the sequence.
The problem with this representation is, that the time resolution should be rather low to avoid the creation of graphs with no, or only a few, edges for most time steps.
\autoref{fig:graph-sequence-example} depicts an example of a graph sequence for a time-varying network with four nodes.
There are also more visual-focused representation methods.
One example would be assigning a time series of time stamp of the contacts between two nodes to the corresponding link in the static network (see \autoref{fig:time-stamp-edges-example} for an example).
This allows the usage of the variety of graph layout algorithms to visualize the network, but does not work very well for large networks due to the lack of space for the time stamps on the links and the large numbers of nodes.
Another idea is to visualize the time-varying network using a timeline of contacts.
The interactions between nodes (i.e., the tuple of nodes that are interacting with each other) are placed on one axis and the time is placed on the other one.
A marker is placed for each pair that interacts at a certain time.
This allows the visual detection of interaction patterns.
However, similar to the last representation methods, this one also only reasonable for small networks.
\autoref{fig:timeline-example} shows an example for this type of representation.

\begin{figure}
   \centering
   \begin{subfigure}[t]{0.39\textwidth}
     \centering
     \begin{tikzpicture}[node/.style={circle,fill=red!70,minimum size=1em,inner sep=0pt,align=center,text width=14pt]}]

       \node[node] (1) at (0, 2) {1};
       \node[node] (2) at (2, 2) {2};
       \node[node] (3) at (0, 0) {3};
       \node[node] (4) at (2, 0) {4};

       \draw[line width=5.00pt] (1) -- (2);
       \draw[line width=2.50pt] (1) -- (4);
       \draw[line width=1.25pt] (2) -- (4);
       \draw[line width=1.25pt] (3) -- (4);
     \end{tikzpicture}

   \caption{Static approximation of the contact sequence as a weighted graph.
   The width of the lines between the nodes represent the the weight (i.e., the number of interactions) of the edges.
   The thicker the line the higher the weight.}
   \label{fig:weighted-network-example}
   \end{subfigure}
   ~
   \begin{subfigure}[t]{0.58\textwidth}
     \centering
     \begin{tikzpicture}[node/.style={circle,fill=red!70,minimum size=1em,inner sep=2pt]}]

       \foreach \t in {0, 1, 2} {
           \node[text width=6em, align=center] at (0.5+\t*2.5, -0.75)  {\(t=\t\)};
           \node[node] (1_\t) at (0+\t*2.5, 1) {\scriptsize 1};
           \node[node] (2_\t) at (1+\t*2.5, 1) {\scriptsize 2};
           \node[node] (3_\t) at (0+\t*2.5, 0) {\scriptsize 3};
           \node[node] (4_\t) at (1+\t*2.5, 0) {\scriptsize 4};
       }

       \draw (1_0) -- (2_0);
       \draw (1_0) -- (4_0);
       \draw (3_0) -- (4_0);
       \draw (2_1) -- (4_1);
       \draw (1_1) -- (2_1);
       \draw (1_2) -- (2_2);
       \draw (1_2) -- (4_2);
     \end{tikzpicture}

   \caption{Visualization of the graph sequence representation of the three time steps of a time-varying network.
   There exists an edge in a graph at time \(t\) if there was a contact between the two nodes at this time step.}
   \label{fig:graph-sequence-example}
   \end{subfigure}

   \begin{subfigure}[t]{0.39\textwidth}
     \centering
     \begin{tikzpicture}[node/.style={circle,fill=red!70,minimum size=1em,inner sep=0pt,align=center,text width=14pt]}]

       \node[node] (1) at (0, 2) {1};
       \node[node] (2) at (2, 2) {2};
       \node[node] (3) at (0, 0) {3};
       \node[node] (4) at (2, 0) {4};

       \draw (1) -- (2) node[midway, above] {0,1,2};
       \draw (1) -- (4) node[midway, right] {0,2};
       \draw (2) -- (4) node[midway, right] {1};
       \draw (3) -- (4) node[midway, above] {0};
     \end{tikzpicture}

   \caption{A graph that contains an edge between two nodes if there was at least one contact between them.
   Furthermore, the edges are annotated with a time series of time steps that indicate when the interactions took place.}
   \label{fig:time-stamp-edges-example}
   \end{subfigure}
   ~
   \begin{subfigure}[t]{0.58\textwidth}
     \centering
     \begin{tikzpicture}[contact/.style={rectangle,fill=black,inner sep=0pt,minimum size=4pt]}]

       \draw (0, 2) node[left] {(1,2)} to (6.5, 2);
       \draw (0, 1.5) node[left] {(1,4)} to (6.5, 1.5);
       \draw (0, 1) node[left] {(2,4)} to (6.5, 1);
       \draw (0, 0.5) node[left] {(3,4)} to (6.5, 0.5);

       \draw[->, line width=1.5pt] (0, 0) to (7, 0) node[below] {t};
       \node[below] at (1, 0) {0};
       \node[below] at (3, 0) {1};
       \node[below] at (5, 0) {2};

       \node[contact] at (1, 2) {};
       \node[contact] at (1, 1.5) {};
       \node[contact] at (1, 0.5) {};
       \node[contact] at (3, 2) {};
       \node[contact] at (3, 1) {};
       \node[contact] at (5, 2) {};
       \node[contact] at (5, 1.5) {};
     \end{tikzpicture}

   \caption{Visualization of the contact sequence as a timeline of contacts.
   The vertical axis shows the interactions that happened between the nodes in the network and the horizontal axis shows the three time steps.
   There is a marker, depicted as a black rectangle, if there was a contact between the pair at a given time \(t\).}
   \label{fig:timeline-example}
   \end{subfigure}

   \caption[Graphical representations of time-varying networks]{Figures \subref{fig:graph-sequence-example}--\subref{fig:timeline-example} show different visualizations for the contact sequence \((1, 2, 0), (1, 4, 0), (3, 4, 0), (1, 2, 1), (2, 4, 1), (1, 2, 2), (1, 4, 2)\).
   Figure \subref{fig:weighted-network-example} shows a static network of the same contact sequence where the number of contacts between two nodes is reflected by the edge weights.}
\end{figure}

It is also noteworthy that most of the introduced measures for networks do not apply for temporal networks or must be redefined, respectively extended.
For example, the concept of degree distributions does not exists in this context.
But there are new measures like the \emph{inter-contact time distributions}, which describe the frequency of the time between contacts between either a specific pair of nodes or any two nodes.
Paths also cannot be used in temporal networks and are replaced by measures like \emph{latency} (i.e., how long since the last contact between two nodes) or \emph{temporal distance} (i.e., how long does it take to get from one node to another while taking the contacts into account).
There is also the idea, and many approaches, to extend community detection mechanisms for the usage in time-varying networks by running community detection algorithms for a static approximation of the network at time \(t\) and then including community information from previous time steps.


%% ========================================================================
%% ========================================================================


\section{Network Models}
\label{sec:network-models}

This section contains descriptions of different network models.
These models can be used to build graphs that fulfill different properties.
They are often a useful tool to model some type of real-world network (e.g., a social network) to gain a deeper understanding on the processes that create these networks.

\subsection{The Erdős-Rényi Model}

The Erdős–Rényi (ER) model~\cite{Erdos1959, Newman2010} was in its first form described by the two famous mathematicians Paul Erdős and Alfréd Rényi in 1959.
The model generates a random graph with \(n\) nodes and \(m\) links.
It chooses one of the \(\binom{\binom{n}{2}}{m}\) possible graphs of this size with equal probability at random.
The idea is that large, complex networks often seem random and can be examined using random graphs and statistical methods~\cite{Barabasi2002}.
Therefore,this is a very simple, yet powerful, model to explore some effects that take place in real-world systems.
It is also often called \(G(n, m)\).
There is another model that is very similar to the \(G(n, m)\) model, called the \(G(n, p)\) model.
Here is the number of links not fixed beforehand, but determined by \(p\), the probability of the presence of a link between any pair of nodes in the network.
Hence, the edges of a random network are determined by flipping a biased coin for each of the \(\binom{n}{2}\) possible edges.

The probability for an arbitrary network with exactly \(m\) links under this model is \(p^{m} (1-p)^{\binom{n}{2} - m}\).
Therefore, the probability that the model will generate a network with \(m\) links is \(\prob{m} = \binom{\binom{n}{2}}{m} p^{m} (1-p)^{\binom{n}{2} - m}\) (i.e., the probability of such network times the number of possible networks).
This corresponds to  a binomial distribution \(B(\binom{n}{2}, p)\).
One simple consequence of that is that the expected value for the number of links \(\expval{m} = \sum_{m=0}^{\binom{n}{2}} m \prob{m} = \binom{n}{2} p\).
The expected value for the average degree of a network generated with this model is deduced in equation~\ref{eq:avg-degree-erdos-model}.
This value is also often called \(c\).
The probability that an arbitrary node has a degree of exactly \(k\) is given by \(p(k) = \binom{n-1}{k} p^{k} (1-p)^{n-1-k}\) (i.e., \(k\) of its possible \(n-1\) links must exist and there are \(\binom{n-1}{k}\) possible combinations for the \(k\) links).
Therefore, the degree distribution of this model is a binomial distribution as well.
However, it is also possible to approximate the degree distribution with an Poisson distribution \(p(k) = \exp(-c) \frac{c^{k}}{k!}\) for large values of \(n\).

\begin{equation}
  c = \expval{d(G)} = \sum_{m=0}^{\binom{n}{2}} \frac{2m}{n} \prob{m} = \frac{2}{n} \sum_{m=0}^{\binom{n}{2}} m \prob{m} = \frac{2}{n} \binom{n}{2} p = (n-1) p
  \label{eq:avg-degree-erdos-model}
\end{equation}

A nice property of the Erdős–Rényi model is that it can be used to study the formation of giant components in networks.
A \emph{giant component} is a component that contains a large fraction of the nodes.
It is interesting that even such a simple model can be used to study a phenomenon that is part of many real-world networks.
However, there are also quite a few problems with this model.
The degrees in real-world networks are usually not binomial, respectively Poisson, distributed.
The formation of hubs in networks usually requires a power-law degree distribution.
Other examples for a shortcomings of the model is the inability to generate community structures and inadequate average path lengths.


\subsection{The Barabási-Albert model}

One model that addresses the problem of missing power-law degree distributions is the Barabási-Albert (BA) model~\cite{Barabasi2002}.
The model is named after its creators Réka Albert and Albert-László Barabási.
It tries to emulate the dynamic process that is responsible for the creation of scale-free degree distributions and yields a the generated network as result.
One of the main differences to the ER model is that the size of the network is not fixed.
The model starts with an small number of nodes and adds new nodes to the network over time (i.e., the network grows).
A newly added node then forms links with already existing nodes.
Which other nodes are chosen depends on how important the other nodes are.
The more important a node is, the more likely it is that the new nodes forms a connection with it.
This process is called \emph{preferential attachment}.
It can, for example, be used to explain the evolution of the online encyclopedia Wikipedia~\cite{Caldarelli2006}.
\citet{Eisenberg2003} showed that preferential attachment is the mechanism behind the evolution of protein networks.
New websites on the World-Wide-Web follow the same pattern, they tend to link to already popular websites, since they are maybe easier to find~\cite{Barabasi1999}.
More formally, the model starts with a small number \(m_{0}\) of nodes and the following two steps are repeated in every time step:

\begin{enumerate}
    \item add a new node to the network
    \item choose \(m \leq m_{0}\) already existing nodes at random proportional to their degree and form a link with them
\end{enumerate}

This means that in this model the degree of a node is a measure of its popularity  and a existing node \(v_{i}\) will be selected with a probability \(\Pi(v_{i})=\frac{d(v_{i})}{\sum_{j} d(v_{j})}\) to form a link.
The process yields after \(t\) time steps a network that consists of \(m_{0} + t\) nodes and \(mt\) links.

It can be shown, using different methods (e.g., continuum theory, master equations, and numerically), that the node degrees follow a power-law distribution with an degree exponent of \(\gamma = 3\).
Furthermore, asymptotically does \(\gamma\) not depend on \(m\), the number of links that are generated in each iteration.
The degree distribution is also (asymptotically) independent on the time, and therefore on the size of the network.
This is also a property of the model and reflects the fact that there exists real-world networks with a power-law degree distribution with different sizes.
However, like the ER model, the BA model has its shortcomings as well.
For example, the average path length of generated networks does not comply with real-world networks.
However, they are more realistic than the path lengths generated by the ER model.
Another property that cannot be reproduced by this model are the community structures that many real-world, and especially social, networks have~\cite{Reid2011}.

Another interesting question regarding the Barabási-Albert model is if both used mechanisms, the network growth and the preferential attachment, are necessary to produce a scale-free network.
The result of numerical simulations and formal tests of the model with either of the two mechanisms missing is that both are required.
Missing preferential attachment results in exponentially distributed node degrees and the missing network growth leads to a power-law degree distribution in the beginning, but it changes to a normal distribution over time.
This indicates that in fact both mechanisms are required to yield a network with the scale-free property.


%% ========================================================================
%% ========================================================================


\section{User Activity Models}
\label{sec:user-activity-models}

The modeling of the activity of users in complex systems (e.g., in social networks) can be a challenging task.
Usually the actual activities, such as the writing of an e-mails, posts on  Facebook, or tweets, are not of particular interest.
More relevant is how are these events or activities laid out in time and what  the distribution of intervals between two consecutive events is (i.e., the inter-event time distribution).
Multiple models try to capture patterns of human activities based on different approaches are discussed in this section.

\subsection{Stochastic Models}

One of the simplest methods to model user activity is by using a Poisson process to describe the inter-activity times~\cite{Masuda2016, Vazquez2006}.
This stochastic process is defined by the event rate \( \lambda \), which states how often a event should occur in a given time window.
Two important properties of the Poisson process are that the inter-activity times are exponentially distributed and independent of each other.
This leads to the effect that events take place in regular intervals (i.e., at the given rate) and that it is almost impossible to have long periods of time between to consecutive activities.
However, it has been shown, that human activity patterns (e.g., email communication) cannot be modeled very accurately by a Poisson process due to its assumptions.
Most activities are executed in bursts, followed by longer periods of inactivity.
For example, a person may have a dedicated time in the day for answering emails.
This behavior can be much better explained by a power-law distribution of the inter-activity times because its long tail allows for longer inactivity periods.
There are, however, approaches that try to tackle this problem that are using extensions of Poisson processes.
For example, \citet{Malmgren2008} use a mixture of homogeneous and non-homogeneous Poisson processes to model user e-mail activity more precisely.
The rate of a non-homogeneous Poisson process does depend on the time, whereas the rate of a homogeneous process is constant.

A approach that does not only generate power-law distributed inter-event times, but also captures other patterns of human behavior, such as periodic spikes (e.g., higher activity every 24 hours) or a bimodal distribution of inter-event times (e.g., phases of high activity that are separated by phases of rest) was proposed by \citet{Costa2015}.
Their Rest-Sleep-and-Comment (RSC) model is based on the idea that a user can be in one of multiple states.
In the active state, a user generates events with a certain probability at some rate, which depends on how much time has passed since the last event.
The rest and sleep states are used to model the inactivity of a user.
The difference is that the rest state produces null-events (i.e., it increments the time) at a certain rate, whereas the sleep state is used to increment only once, but by a larger amount.
A active user can become inactive (i.e., go to the rest state) or stay active.
A user in the rest state can either become active, stay inactive, or go to the sleep state.
The user can not stay in the sleep state, only go into the rest state again.
The state transition probabilities are parameters of the model.
This model was the foundation for a classifier, that is able to detect whether a activity sequence was generated by a bot or by a human with very high accuracy.

Another possibility to model the user activity in social networks is by using coupled Hidden Markov Models (CHMMs)~\cite{Raghavan2013}.
The Markov model has two hidden states that describe the user activity (active or inactive) and yields inter-activity times with respect to the current state.
The CHMM model takes the social network influence of other users into account by explicitly coupling the stochastic processes of groups of people.
This is done by letting the transition probabilities between the states of the HMM for a single user be dependent on the activity of other users, that are in the circle of friends of the user (i.e., neighbors in the social network).
If the activity of the neighbors exceeds a certain threshold the probability that a user becomes active is larger.
The results show that this model is able to learn the complex human activity patterns and allows predictions with high accuracy compared to other methods.

\subsection{Queuing Models}

One approach that archives to model human activity with a more suitable power-law distributions are queuing models~\cite{Vazquez2006}.
The idea here is to think of the user as a queue that is constantly filled with new tasks (e.g., answer a email, go shopping, do the dishes,\ldots).
Each task takes a certain amount of time to finish and is prioritized by the user on arrival.
Furthermore, the queue is usually bounded in size, since the user can only keep track of a certain amount of tasks at a time.
At each time step the user selects the task with the highest priority from the queue and executes it.
It can be shown that the time it takes for a task to be handled (i.e., the waiting time) follows a power-law distribution.
There is evidence that the waiting time distribution of the queuing model is responsible for the inter-activity time power-law distribution of a specific activity, due to the fact that persons tend to group tasks in categories and reinserting them into the queue with a lower priority after they are done.
For example, a person does not keep track of every unanswered email in the queue, but has a \enquote{answer email} task that contains all emails that need to be replied-to.
Therefore, answering multiple emails in a short period of time, followed by a longer period of no email correspondence due to other tasks with higher priority.

\subsection{Time-varying Network Models}
\label{subsec:time-varying-network-models}

The prior discussed user activity models are designed to solely describe the activity profile of a single user, or require at least a separate stochastic process for each one.
However, there are models that allow the description of the activities for multiple users at once using time-varying networks (i.e., as contact sequences).
The difference between models for temporal networks and static network models as described in~\autoref{sec:network-models} is, that the latter are purely connectivity driven~\cite{Perra2012a}.
This means that they are laid out to generate specific topological properties in the networks (e.g., the formation of community structures, or short average path lengths), but do not consider the dynamic processes, which are forming these structures.

One of the simplest methods to generate a time-varying network of user activity was proposed by \citet{Holme2013}.
The idea is to generate a static network and assign each link a (possibly empty) set of time stamps, that represents the contact sequence between the two nodes.
To archive this, the first step is to generate a static network using the configuration model~\cite[cf. sec. 13.2]{Newman2010}.
This model assigns a number, drawn from a probability distribution (e.g., a power-law distribution), to each node in the network.
This number represents the number of \enquote{half edges} of the node.
These \enquote{half edges} can be seen as dangling edges that will be connected at random to other \enquote{half edges} of different nodes.
Therefore, this number corresponds to the degree of the node.
Self-loops and parallel edges are avoided during the matching process to generate a simple static network.
In the next step, each link in network is assigned a time window at random in which contacts are possible (i.e., the activity interval).
In the last step of this approach, a time series of events is generated by drawing inter-event times from a power-law distribution.
This time series is then split into parts and mapped onto the activity windows of the nodes, thus, generating the contact sequences.

A different approach by~\citet{Perra2012a} is based on the idea of activity potentials.
Each node \(v_{i}\) in the network of size \( n \) is assigned a quantity called the activity potential \(x_{i}\), which denotes the probability that the node will be active in a time window \(\Delta t\).
The activity potentials of real-world networks can be determined by calculating the ratio of the number of interactions of a user to the total number of interactions in a time window for every user.
This usually yields long tailed activity potential probability distributions, which corresponds to typical heterogeneous human activity patterns~\cite{Vazquez2006, Jo2012}.
Furthermore, the size of the time window, that is used to estimate the probabilities, seems not to effect the resulting distribution in a significant way.
The first step of this activity-driven model is to initialize each node in the network by assigning it a activity/firing rate \(a_{i} = \eta x_{i}\), where the activity potential is drawn from a suitable probability distribution \( f(x) \) and \( \eta \) is a re-scaling factor that is chosen in such a way that the expected number of active node in the time window is \(\eta \expval{x} n\).
The range of possible activity potential values is \(x_{i}\ \in [\varepsilon, 1]\).
The activity potential has a lower bound \( \varepsilon \) to avoid possible divergences of the distribution for values that are very close to zero.
After the setup phase the time-varying network can be generated.
In this model the network is represented as a sequence of graphs (see \autoref{sec:time-varying-networks}), which are called instantaneous networks in the context of this model.
For each time step \(t\) the following steps are repeated:

\begin{enumerate}
    \item Create a new network \(G_{t}\), that contains all \( n \) nodes but has no links yet.
    \item Every node \(v_{i}\) becomes active with probability \(a_{i} \Delta t\). Active nodes choose \(m\) distinct other nodes uniformly at random and form a link with them.
    \item Increment the current time \(t \rightarrow t + \Delta t\).
\end{enumerate}

Since every active node creates \(m\) links, the cardinality of the edge set is given by \(|E_{t}| = m \eta \expval{x} n\).
Therefore, the average degree of the instantaneous network at time \(t\) is \(d(G_{t}) = \frac{2|E_{t}|}{n} = \frac{2 m \eta \expval{x} n}{n} = 2 m \eta \expval{x}\).
The probability that a node becomes active does not change over time and is, therefore, also independent of previous activities.
This resembles the problem already encountered with Poisson processes models.
The inter-event times for the node \(v_{i}\) will eventually be exponential distributed \(\varphi_{i}(\tau) = a_{i} \exp(-a_{i} \tau)\)~\cite{Moinet2016}.
Additionally, it is not very realistic in the sense that every node selects its neighbors in each iteration uniformly at random, which leads to networks with a random structure.
Users are usually prone to repeat previous communication~\cite{Karsai2014}.
Nevertheless, this model possesses a few considerable advantages.
First, and most important, it produces not only activities for a single user, but it generates snapshots of the user activities in the network for each time window.
Additionally, it is a very simple model.
It only requires a few parameters and the activity potential distribution is governs the dynamical behavior in the network.
Furthermore, this model can be used to explain the formation heterogeneous structures (i.e., hubs) in networks over longer periods of time.
This is done by examining the integrated network \(G_{T} = \bigcup_{t=0}^{T} G_{t}\), which the union of all instantaneous networks up to the time stamp \(T\).
It can be shown that the  degree distribution of this integrated network has the form \(p_{T}(k) \sim f(\frac{k}{T m \eta})\).
Therefore, it is up to a re-scaling factor, the same as the activity potential distribution.
Hence, the model is able to generate scale-free networks, by drawing the activity potentials from a power-law distribution.
The resulting hubs are not caused by preferential attachment like in other models, but due to the heterogeneous activity profiles of the nodes.
The re-scaling factor emerges from the fact that the model does not capture all features of real-world networks.
For instance, memory effects could allow links, that were formed in earlier instantaneous networks, to be formed later again with higher probability.

Regardless of its simplicity, this activity-driven model can help to understand topological patterns and the dynamics of systems (e.g., epidemic spreading processes).
\citet{Starnini2013} study the topological properties of this model in a more formal way by mapping it onto a hidden variables network model.
The idea behind hidden variables models is, that the probability of the formation of a link between two nodes depends on some underlying characteristic of the nodes (i.e., the hidden variables).
In this case the hidden variable is the activity potential.
A network that was generated using this model fully depends on probability distributions that are related to the hidden variables.
Therefore, topological properties also depend on these probability distributions and can be expressed with respect to them.
The properties of the degree distribution of the integrated network could be verified using this more formal approach.
In addition, they showed that the clustering coefficient of the network is rather small and comparable to the clustering coefficient of random networks.
They also propose to use the hidden-variable approach to study possible extensions of the model, which is, for example, done for the NoPAD model~\cite{Moinet2015}, that is briefly discussed later in this section.

The activity-driven model is also a fundamental framework for many other studies that use it as start point for their work.
For example, \citet{Perra2012b} use this model to study random walks on temporal networks, or \citet{Rizzo2016} are applying this model for their research about the spreading the of the infectious Ebola disease in Liberia.
Another paper by \citet{Rizzo2014} uses the activity-driven network framework to study how an epidemic effects the behavior of persons.
They show that a reduction of the activity potential of persons, due to the fact that there are already ill or because they are trying to protect them self from the disease, may help the slow down the spreading process.
A paper on a similar topic by \citet{Liu2014} proposes a framework to develop strategies on how to contain the spreading of diseases.
\citet{Mistry2015} use the model to explore the spreading of opinions in social networks.
They show that activists are able to spread messages across the population more effectively and can help to reduce the cost of campaigns.

On the other hand, others try to improve the model by making it more realistic by implement additional mechanisms.
\citet{Laurent2015} adds additional social mechanisms (e.g., memory) to the model to allow for the formation of communities on the integrated network.
Another extension by \citet{Moinet2015, Moinet2016} solves the problem of inaccurate inter-events times by making the activity potential of each node time depended.
This model is known as the non-poissonian activity-driven (NoPAD) model and can generate inter-event time distributions that can also be observed in real-world networks.
\citet{Wang2016} proposed the Activity-Security-Trust (AST) model, which not only considers activity as the explicit driving force behind dynamic processes, but also incorporates the implicit factors security and trust.
Trust can be seen as the belief in honesty or fairness between two nodes and influences the possible link formation between them.
The second extension, the security level, is like the activity potential a property of each node that determines how well a node is prepared against possible attacks.
It can be seen as the probability of a node to accept the interaction initiated by another one.
\citet{Sunny2015} adds link lifetimes to the activity-driven framework.
Every time a link is formed it is assigned a lifetime, that is drawn at random from a probability distribution.
This link may then be part of multiple consecutive instantaneous networks
and is only removed after its lifetime has decayed, in contrast to the simple activity-driven model, where links are meant to be instantaneous and are deleted after every time step.
The authors use this new introduced mechanism to study how well link lifetimes are suited to model disease spreading processes.


%% ========================================================================
%% ========================================================================


\section{Peer Influence}
\label{sec:peer-influence}

\chapter{Model}

The proposed user activity model with peer influence effects is in detail described in this chapter.
It is yet another extension of the activity-driven network framework by \citet{Perra2012a}, that was discussed in \autoref{subsec:time-varying-network-models}.
However, it is not directly based on it, but on the work of \citet{Laurent2015}, which also relies on the the activity-driven framework.
This underlying model, which is explained in the first section of this chapter, allows the formation of community structures and the formation of strong an weak ties in the network, which is a crucial condition for the occurrence of peer influenced activity.
The additional introduced mechanisms, that allow for peer influenced activity to actually happen in the network, are then discussed in \autoref{sec:peer-influence-model} of this chapter.


%% ========================================================================
%% ========================================================================


\section{The Base Model}
\label{sec:base-model}


\subsection{Description}

Since this model~\cite{Laurent2015} is based on the activity-driven framework by \citet{Perra2012a}, a activity potential \( a_{i} = \eta x_{i}\),  which is drawn from a suitable distribution, is assigned to each node.
To reflect the heterogeneous activity patterns of people, a power-law distribution \( f(x) \sim x^{-\gamma}\) is selected.
The exponent for the activity potential distribution is fixed to \( \gamma = 2.7 \), which is a value that is similar to the exponent observed in real world communication networks.
The lower bound for the activity potential is set to \( \varepsilon = 10^{-3} \), and the time scaling parameter is set to \( \eta = 1\), so that \( a_{i} = x_{i} \in [\varepsilon, 1] \).
Furthermore, for the sake of simplicity and without loss of generality are the time window \( \Delta t \), and the number of generated links at each activation \( m \) all set to the value one.

The dynamic part of the model is basically identical to the activity-driven framework.
A node becomes active in every time window with probability equal to its activity potential and selects other nodes to interact with.
The difference is how the interaction partners are selected once a node becomes active.
The goal of this model is it to produce adjustable community structures and weight-topological correlations in the integrated network, which can be archived by changing the way the other nodes are selected.
More specifically, to archive these more realistic topological properties, the following two additional social mechanisms are introduced:

\begin{enumerate}
    \item Memory effects
    \item Closure processes
\end{enumerate}

The first mechanism introduces memory to the nodes, in the sense that nodes remember all previous interactions with other nodes.
This idea was adapted from the work of \citet{Karsai2014}, and enables the formation of strong ties (i.e., interactions that are repeated often) and weak ties (i.e., interactions that area repeated infrequently) between the nodes.
This heterogeneity of tie strengths is an important role for processes that take place in many real world networks.
For instance, they showed in their work, that strong ties are able to slow down the spreading of rumors in networks of social interaction (e.g., a network of phone calls).
This counter intuitive result can be explained by the observation that most activity is concentrated in strongly tied groups, which prevents a fast spreading of the rumor into other parts of the network.

The memory of a node is represented by a weighted egocentric network, that includes all other nodes, which were already part of one or more interactions in the past.
These previous communication partners are also called neighbors.
The weight represents the number of previous interactions, scaled by the link-reinforcement constant \( \delta \).
\autoref{fig:egocentric-network} shows an exemplary egocentric network of the node \( v_{i} \) and its neighbors.
The choice of forming a new tie or reinforcing an existing one depends on the number of neighbors of a node.
This corresponds to the observations of social interaction dynamics, where actors tend to communicate almost exclusively within their social cycle, which has a limited size, due to cognitive capacities of the actors~\cite{Dunbar1992}.
Let \( k_{i} = d(v_{i}) \) be the degree of the node \( v_{i} \) in its egocentric network, then the probability to form a new tie is \( p(k_{i}) = \frac{c}{k_{i} + c} \), and the probability to interact with an already established tie is \( \bar{p}(k_{i}) = 1 - p(k_{i}) = 1 - \frac{c}{k_{i} + c} = \frac{k_{i}}{k_{i} + c} \), where the constant \(c\) determines how strong the memory of an actor is (cf.  \autoref{fig:reinforcement-process-prob-plot} for details).
Therefore, the probability for the formation of a new tie decays very fast with increasing size of the egocentric network.
The memory strength constant is fixed to \( c = 1 \) in the context of this work.
If an active node decides to reinforce an existing tie, the neighbor is selected proportional to the current tie strength.
Therefore, the probability for node \( v_{j} \) to be selected as communication partner of node \( v_{i} \) is given by
\( p_{i,j} = \sfrac{w_{i,j}}{\sum_{k \in N(v_{i})} w_{i, k}} \), where \( w_{i,j} \) denotes the tie strength between node \( v_{i} \) and \( v_{j} \) in the egocentric network of \( v_{i} \).
This reinforcement process allows the introduction of dependencies between successive interactions of node pairs and replaces the approach of selecting a communication partner uniformly at random when a node becomes active in the original activity-driven framework.


\begin{figure}
    \centering

    \begin{tikzpicture}[node/.style={circle,fill=red!70,minimum size=1em,inner sep=3pt]}, neighbor/.style={circle,fill=blue!70,minimum size=1em,inner sep=3pt]}]
      \node[node] (1) at (-1, -1)  {i};
      \node[neighbor] (2) at (2.5, 1.5) {j};
      \node[neighbor] (3) at (2.5, -1) {k};
      \node[neighbor] (4) at (2.5, -3.5) {l};

      \draw (1) -- (2) node [midway, above, sloped] (a) {$w_{i,j} = 3$};
      \draw (1) -- (3) node [midway, above, sloped] (b) {$w_{i,k} = 5$};
      \draw (1) -- (4) node [midway, above, sloped] (c) {$w_{i,l} = 2$};
    \end{tikzpicture}

    \caption[Egocentric network example]{Egocentric network of the node \(v_{i} \) (red node) and its neighbors \( v_{j} \), \( v_{k} \), and \( v_{l} \) (blue nodes).}
    \label{fig:egocentric-network}
\end{figure}


\myfig{reinforcement-process-prob-function}
      {width=0.75\textwidth}
      {Plots of the function that determines the probability for the formation of a new tie based on the degree of a node \( p(k) \) for different values of the memory strength \( c \). This constant can help to model different types of users. Larger values may correspond to social explorers, that are more prone to form new ties, and smaller values are related to social keepers, which communicate almost exclusively to peers in their social circle.}
      {Probability distribution for the formation of new ties.}
      {fig:reinforcement-process-prob-plot}


The second mechanism introduces two different closure processes to the model.
The first one, cyclic closure, assures the formation of triangles (i.e., cliques between three nodes), which were linked to the formation of community structures in the network by \citet{Bianconi2014}.
They showed that this mechanism is sufficient to generate networks with complex topological structures (e.g., long-tailed degree distributions), where the strength of communities depends on the cyclic closure probability.
If a node wants to form a new tie, it tries to perform with probability \( p_{\Delta} \) a cyclic closure, by interacting with a randomly selected neighbor of a neighbor.
The second closure process, focal closure, tries to emulate the social dynamic that users tend to form ties with other users that are similar to them (e.g., they have common interests).
This process is performed when a new tie should be created with a probability of \( 1 - p_{\Delta} \) or if there are no suitable candidates for a cyclical closure available.
This is, for instance, the case if a node becomes active the first time.
The weight of a new tie is always initialized to a value of one, regardless of the type of closure that was used to establish the new tie.
The actual implementation of these two closure mechanisms was adapted from the work of \citet{Kumpula2007}, who used the same closure mechanisms to study the formation of community structures in weighted static networks.
They model cyclic closure as a biased local search (cf. \autoref{fig:cyclic-closure} for details) and focal closure as an unbiased global search, which means selecting a new node uniformly at random from the entire network.
Furthermore, the introduced a node deletion mechanism in their model, which was adapted here as well.


\begin{figure}
    \centering

    \begin{subfigure}[t]{0.3\textwidth}
    \begin{tikzpicture}[active/.style={circle,fill=red!70,minimum size=1em,inner sep=3pt]}, neighbor/.style={circle,fill=blue!70,minimum size=1em,inner sep=3pt]}]
      \node[active]   (1) at (0, 1) {};
      \node[neighbor] (2) at (1, 0) {};
      \node[neighbor] (3) at (1, 2) {};
      \node[neighbor] (4) at (2, 1) {};
      \node[neighbor] (5) at (3, 0) {};
      \node[neighbor] (6) at (3, 2) {};

      % dangling edges
      \draw (3) -- (1.3, 2.3);
      \draw (3) -- (0.7, 2.3);
      \draw (2) -- (0.7, -0.3);
      \draw (6) -- (2.7, 2.3);
      \draw (5) -- (3.4, 0);
      \draw (6) -- (3.4, 2);

      \draw (1) -- (2) node [midway, below] (a) {2};
      \draw (1) -- (3) node [midway, above] (b) {3};
      \draw (1) -- (4) node [midway, below] (c) {5};
      \draw (4) -- (6) node [midway, below] (d) {3};
      \draw (4) -- (5) node [midway, above] (e) {4};
      \draw (2) -- (5) node [midway, below] (f) {7};
    \end{tikzpicture}
    \caption{}
    \label{subfig:cyclic-closure-a}
    \end{subfigure}
    \begin{subfigure}[t]{0.3\textwidth}
    ~
    \begin{tikzpicture}[active/.style={circle,fill=red!70,minimum size=1em,inner sep=3pt]}, neighbor/.style={circle,fill=blue!70,minimum size=1em,inner sep=3pt]}, selected/.style={circle,fill=green!70,minimum size=1em,inner sep=3pt]}]
      \node[active]   (1) at (0, 1) {};
      \node[neighbor] (2) at (1, 0) {};
      \node[neighbor] (3) at (1, 2) {};
      \node[selected] (4) at (2, 1) {};
      \node[neighbor] (5) at (3, 0) {};
      \node[neighbor] (6) at (3, 2) {};

      % dangling edges
      \draw (3) -- (1.3, 2.3);
      \draw (3) -- (0.7, 2.3);
      \draw (2) -- (0.7, -0.3);
      \draw (6) -- (2.7, 2.3);
      \draw (5) -- (3.4, 0);
      \draw (6) -- (3.4, 2);

      \draw (1) -- (2) node [midway, below] (a) {2};
      \draw (1) -- (3) node [midway, above] (b) {3};
      \draw[line width=0.5mm] (1) -- (4) node [midway, below] (c) {5};
      \draw (4) -- (6) node [midway, below] (d) {3};
      \draw (4) -- (5) node [midway, above] (e) {4};
      \draw (2) -- (5) node [midway, below] (f) {7};
    \end{tikzpicture}
    \caption{}
    \label{subfig:cyclic-closure-b}
    \end{subfigure}
    ~
    \begin{subfigure}[t]{0.3\textwidth}
    \begin{tikzpicture}[active/.style={circle,fill=red!70,minimum size=1em,inner sep=3pt]}, neighbor/.style={circle,fill=blue!70,minimum size=1em,inner sep=3pt]}, selected/.style={circle,fill=green!70,minimum size=1em,inner sep=3pt]}]
      \node[active]   (1) at (0, 1) {};
      \node[neighbor] (2) at (1, 0) {};
      \node[neighbor] (3) at (1, 2) {};
      \node[selected] (4) at (2, 1) {};
      \node[neighbor] (5) at (3, 0) {};
      \node[selected] (6) at (3, 2) {};

      % dangling edges
      \draw (3) -- (1.3, 2.3);
      \draw (3) -- (0.7, 2.3);
      \draw (2) -- (0.7, -0.3);
      \draw (6) -- (2.7, 2.3);
      \draw (5) -- (3.4, 0);
      \draw (6) -- (3.4, 2);

      \draw (1) -- (2) node [midway, below] (a) {2};
      \draw (1) -- (3) node [midway, above] (b) {3};
      \draw[line width=0.5mm] (1) -- (4) node [midway, below] (c) {5};
      \draw[line width=0.5mm] (4) -- (6) node [midway, below] (d) {3};
      \draw (4) -- (5) node [midway, above] (e) {4};
      \draw (2) -- (5) node [midway, below] (f) {7};
      \draw[line width=0.5mm] (1) -- (6) node [midway, above] (g) {1};
    \end{tikzpicture}
    \caption{}
    \label{subfig:cyclic-closure-c}
    \end{subfigure}

    \caption[Cyclic closure mechanism example]{This is an illustration of the cyclic closure mechanism in the model. The network depicted in these figures is part of the union of all egocentric networks (i.e., the integrated network). \autoref{subfig:cyclic-closure-a} shows the active node in red. In the first step, this node has to select one of his neighbors. This is done at random with respect to the tie strengths. Therefore, the probabilities for the three neighbors to be selected are \(\sfrac{3}{10}\), \(\sfrac{5}{10}\), and \(\sfrac{2}{10}\) respectively. In this example the neighbor with the highest probability was selected, which is depicted in \autoref{subfig:cyclic-closure-b}. Since the selected neighbor has neighbors himself that do not share a link with the active node yet, the cyclic closure can be completed. This is done once more by selecting one of the candidates at random with respect to the weight of the ties and creating a new tie with unit strength with probability \(p_{\Delta}\). \autoref{subfig:cyclic-closure-c} shows the newly formed triangle in the network.}
    \label{fig:cyclic-closure}
\end{figure}


In the activity-driven framework, nodes live forever and are, therefore, forever part of the network.
Here, nodes have an intrinsic probability \(p_{d}\) to be removed in every time step, which is the same for every node in the network.
This ensures that the network can reach a stable state in which the structural characteristics (e.g., the community structures) become invariant in time.
However, every time a node is removed from the network, a new one joins to keep the size of the network constant.
The deletion probability of nodes can determine how fast the network reaches its equilibrium.
A small value for \(p_{d}\) allows nodes to stay a long time in the network and even nodes with relatively small activity potential can become fully integrated in the community structures.
Therefore, it takes longer to reach the time invariant state of the network, when the low activity nodes are not removed fast enough.
The expected time that a node will be part of the network can be determined by viewing it as a simple Bernoulli process.
In each iteration a biased coin is tossed for every node.
The outcome of this Bernoulli random experiment determines if the nodes stays in the network or is replaced in the next round.
The probability for a node to be deleted after exactly \( x \) iterations is \( \prob{x} = (1-p_{d})^{x-1} p_{d} = \bar{p}_{d}^{x-1} p_{d}\).
Hence, the expected value for the lifetime of a node is given by \( \expval{x} = \sum_{x=1}^{\infty} x \prob{x} = \sum_{x=1}^{\infty} x \bar{p}_{d}^{x-1} p_{d} = p_{d} \sum_{x=1}^{\infty} x \bar{p}_{d}^{x-1} \).
This sum is related to the sum of the geometric series \( \sum_{x=0}^{\infty} r^{x} = \frac{1}{1 - r} \), for \(|r| < 1 \), by being its first derivative\footnote{The first derivative of the sum of the geometric series is \( \frac{\mathrm{d} \sum_{x=0}^{\infty} r^{x}}{\mathrm{d} r} = \sum_{x=0}^{\infty} \frac{\mathrm{d} r^{x}}{\mathrm{d} r} = \sum_{x=1}^{\infty} x r^{x-1} = \frac{\mathrm{d} \frac{1}{1-r}}{\mathrm{d} r} = \frac{1}{(1 - r)^{2}} \), for \(|r| < 1\).}.
Therefore, the expected value for the lifetime of a node is \( \expval{x} = \frac{p_{d}}{(1 - \bar{p}_{d})^{2}} = \frac{1}{p_{d}} \).
This means that, for example, nodes with a deletion probability of \(p_{d} = 5e-5\) will be on average deleted after 20,000 iterations.


\subsection{Properties}

The properties of this model are examined by analyzing a extended version of the integrated network.
This is very similar to the basis framework, in which the temporal network is represented as a sequence of graphs.
These graphs are denoted as instantaneous networks, and the union of these networks up to a time step \( T \) is called the integrated network.
This is also true for this extension, however, the links in the integrated network have additionally a weight assigned to them, which corresponds to the tie strength in the egocentric networks of the nodes.
Another equivalent way to define the integrated network is the union of all egocentric networks up to some time step \( T \).

These newly introduced mechanisms have interesting effects on how the structures on the integrated network evolve over time.
In the beginning, after nodes form their first ties, they start to close triangles and reinforce the ties in their egocentric network.
This means that strong community structures are formed early in the model.
However, after a while more weak ties are introduced and fewer triangles are closed so that the strength of the communities declines and the network reaches its equilibrium state.
As mentioned earlier, the node deletion probability can be used to control the time until the network converges, but it can also be used to tune the strength of the communities and the average degree in the network.
A smaller value for \( p_{d} \) decreased the average local clustering coefficient and increases the average degree.

The cyclic closure probability \( p_{\Delta} \) and the reinforcement increment \( \delta \) control the formation of communities as well (see \autoref{fig:community-structures-in-model}).
Furthermore, like the node deletion probability, the two parameter have an effect on the average degree of the converged network.
Higher values for the cyclic closure probability or the tie reinforcement increment result in a smaller average degree.
However, the two parameter do not influence the actual (heterogeneous) distribution of the degrees in a significant way.
The tie strengths, which are power-law distributed, are not effected by \( p_{\Delta} \), and higher values for \( \delta \) only influence the length of the tail.
Another property is the that the cyclic closure probability has an larger impact on the emerging community structures than the tie reinforcement increment, since it simply leads to more triangles.
Whereas \( \delta \) is responsible for the creation of strong ties, which increases the bias in the local search, and helps to find suitable nodes for the triangle formation.
Additionally, the model is able to produce higher-order correlations, that are also observable in real-world networks.
For example, weight-topology correlations (i.e., stronger ties within groups) are measurable and are depended on \( p_{\Delta} \) and \( \delta \).


\myfig{community-structures-model}
      {width=0.9\textwidth}
      {Depiction of the influence of \( p_{\Delta} \) and \( \delta \) on the resulting community structures (image borrowed from~\cite{Laurent2015}). The networks in the first row (a--c) were generated with a fixed value for the link reinforcement increment \(\delta = 1\) and varying values for the cyclic closure probability (from left to right: \( p_{\Delta} = 0.5 \), \( p_{\Delta} = 0.9 \), and \( p_{\Delta} = 0.995 \)). This shows that \( p_{\Delta} \) directly influences the strength of the communities. Furthermore, tie strength heterogeneities are observable, with strong ties within communities (darker link color) and weak ties between them (brighter link color).The second row shows networks with a fixed cyclic closure probability \( p_{\Delta} = 0.995 \) and different reinforcement constants (from left to right: \(\delta = 0\), \(\delta = 0.5\), and \(\delta = 1.5\)). This shows that a high probability for the formation of triangles is not enough for the formation of communities. The reinforcement process that helps to develop strong ties is required as well.}
      {Influence of \( p_{\Delta} \) and \( \delta \) on the community structures}
      {fig:community-structures-in-model}


%% ========================================================================
%% ========================================================================


\section{Peer Influence Extension}
\label{sec:peer-influence-model}

So far the activity in the temporal network was entirely determined by the activity-potential distribution.
Each node is assigned an intrinsic probability to become active in each round, which is drawn from this distribution.
The assignment is done once for every node when it gets created and does not change afterwards.
A node can become active in an iteration of the model either by himself or by being contacted by another active node.
The process of becoming active on one's own accord does not depended on whether or not the node was active in previous time steps.
This corresponds to a memory-less Poisson process and leads necessarily to exponentially distributed times between two consecutive activations of a node (i.e., inter-event times).
The significance of the inter-event time distribution in human behavior was already discussed in \autoref{sec:user-activity-models}.
However, it is evident that activations caused by other nodes are not necessarily independent of previous events, due to the memory effects introduced in the model.
For example, lets assume a node with very low activity potential is part of group of high-activity nodes and already established strong ties.
The self-activation rate of the low-activity node will be quite low, with a mean value and standard deviation for the inter-event times that is equal to the inverse of its activity probability.
Nevertheless, the other nodes in the group will fairly often select the low-active node as communication partner when they become active, due the biased local search.
This can, of course, alter the inter-event time distribution of the node with the small activity potential in a significant way, compared to distributions that are generated by the activity-driven framework, where the activation through other nodes happens only at random.
This, in this case implicit, influence that nodes have on the activity on their neighbors is an interesting effect and was a starting point for this thesis.
In this section, a extension of the prior discussed base model with memory and closure effects is presented, which tries to model the influences of peers in the local network in a more explicit way.

The ideas for this model were heavily influenced by the work of \citet{Walk2016}.
They proposed a model that includes peer-influence and its impact on the state of activity in collaboration networks.
A more detailed description of their work is located in \autoref{sec:peer-influence}.
The gist of the here proposed extension to the model is that a node can not only become active on its own based on its activity potential, but also by being motivated to become active by its neighbors, that were active in the previous iteration.
Another way to look at it would be that active nodes can influence their neighbors to become active as well in the next round.
Therefore, introducing a more explicit peer influence mechanism to the model.
This also means that a node now can become active in three different ways.
First, it can become active by himself either due to its intrinsic fixed activity probability or due to the influence of the nodes in its egocentric network, or it can become active in a passive way by being contact by another active node.

The peer influence that a node \(v_{i} \) receives from its neighbors is denoted as \( p_{i} \).
Like the activity potential, \( p_{i} \) is a probability for an activation as well, but instead of being fixed, it may vary in each iteration based on the number of active neighbors in the last round.
Therefore, a more appropriate notation is \( p_{i}(t) \).
To regulate how much peer influence a node can receive at most, an upper bound \( q \) is defined, such that \( \forall t: \, 0 \leq p_{i}(t) \leq q \leq 1 \).
It denotes the maximal probability for an activation motivated by the neighbors of a node.

Since the peer influence probability depends on the neighbors that were active in the last round, the information of the last activations of the nodes must be stored.
This is done by extending the egocentric networks to save the timestamp of the last activation for each node.
The time of last activation of a node \( v_{i} \) is called \( t_{i} \) and is updated independent of the type of activation (i.e., due to self activation or by being contacted by another active node).
Furthermore, the peer influence probability should not only depend on how many of a nodes neighbors were active in the round before, but also on the strength of the ties between them.
A neighbor that shares a strong connection with a node should be more influential than, for example, neighbors that were only recently introduced to the egocentric network.
This can easily be described by a weighted fraction of the last active neighbors.
However, to make the model more general and adaptable the following version of the weighted fraction of active neighbors at time \(t - 1\) is used, that transforms the weights beforehand.
Each weight in the egocentric networks of \( v_{i} \) is transformed and normalized by

\begin{equation}
    w'_{i,j} = \frac{\exp(\beta w_{i,j})}{\sum_{k \in N(i)} \exp(\beta w_{i,k})},
\end{equation}

where \( \beta \) is a free parameter.
This corresponds to applying the softmax or normalized exponential function~\cite{Bishop2006} to each weight.
The softmax function is strongly related to the Boltzmann distribution~\cite{vanLaarhoven1987}, which describes the probability for states in a physical system (e.g., particles in a magnetic field) with respect to the systems temperature and the energy of its states.
Low energy states have a higher probability in the distribution, and as the temperature of the system gets close to zero, the probability for the state with the lowest energy is almost equal to one.
Whereas, in a high temperature system all states are nearly equiprobable.
The free parameter \( \beta \) in the softmax function is called inverse temperature and replicates this behavior.
It is defined as the reciprocal value of the temperature.
This softmax function is used in many applications in different fields besides physics as well.
For instance, it is used in the machine learning area of reinforcement learning, where the actions of agents in some setting are based on the probabilities yielded by the softmax function.
\citet{Crites1998} apply this method to control a group of elevators using multiple agents.
It is also, for example, used in the modeling of the decision making behavior of humans in economic settings~\cite{Ray2008}.


\begin{figure}
    \centering
    \begin{tikzpicture}[node/.style={circle,fill=red!70,minimum size=1em,inner sep=3pt]}, neighbor/.style={circle,fill=blue!70,minimum size=1em,inner sep=3pt]}]
      \node[node, label=left:{$t_{i} = 43$}] (1) at (-1, -1)  {i};
      \node[neighbor, label=right:{$t_{j} = 42$}] (2) at (3, 2.0) {j};
      \node[neighbor, label=right:{$t_{k} = 23$}] (3) at (3, -1) {k};
      \node[neighbor, label=right:{$t_{l} = 39$}] (4) at (3, -4.0) {l};

      \draw (1) -- (2) node [midway, above, sloped, align=right] (a) {$w_{i,j} = 3$ \\ $w'_{i,j} = 0.114$};
      \draw (1) -- (3) node [midway, above, sloped, align=right] (b) {$w_{i,k} = 5$ \\ $w'_{i,k} = 0.844$};
      \draw (1) -- (4) node [midway, above, sloped, align=right] (c) {$w_{i,l} = 2$ \\ $w'_{i,l} = 0.042$};
    \end{tikzpicture}

    \caption[Extended egocentric network example]{Extended egocentric network of the node \(v_{i} \) (red node) and its neighbors \( v_{j} \), \( v_{k} \), and \( v_{l} \) (blue nodes). Each node stores additionally the point in time on which it was last active. For instance, node \(v_{j} \) was last active at \( t = 42 \). Furthermore, the scaled and normalized weights \( w'_{i,j} \) are part of the network. For example, the scaled weight for the tie between \( v_{i} \) and \( v_{j} \) can be calculated by \(\sfrac{\exp(3)}{\exp(3) + \exp(5) + \exp(3)} = 0.114 \). For the sake of simplicity is \( \beta = 1 \) in this example.}
    \label{fig:extended-egocentric-network}
\end{figure}


The usage of the softmax function for this model allows different influence scenarios.
For example, a value of \( \beta = 0 \) (this would correspond to a very high temperature) would scale every weight to the same value, which means that every active neighbor influences a node equally, regardless of the tie strength.
However, it also possible to make \( \beta \) depended on the time (i.e., \( \beta(t) \)), similar to a physical system that is cooling off or heating up.
For this thesis, the temperature at time \( t \) is set to the average weight in the integrated network \( G_{T} = \sum_{i=0}^{t} G_{i}\) (i.e., \( \beta(t) = (\frac{1}{m} \sum_{(i,j) \in E(G_{T})} w_{i,j})^{-1} \)).
The average weight, similar to the local clustering coefficient and average degree, converges to a value after some time as well.
The normalized weights \( w'_{i,j} \) are also part of the extended egocentric (c.f. \autoref{fig:extended-egocentric-network}) and must be updated at the end of every iteration.
The fraction of active neighbors \( \alpha_{i}(t) \) of node \(v_{i} \) at time \( t - 1 \) is then given by

\begin{equation}
    \alpha_{i}(t) = \frac{\sum_{j \in N(i)} \mathbf{1}_{\{t_{j} = t-1\}} \exp(\beta(t) w_{i, j})}{\sum_{j \in N(i)} \exp(\beta(t) w_{i, j})} = \frac{\sum_{j \in N(i)} \mathbf{1}_{\{t_{j} = t-1\}} w'_{i, j}}{\sum_{j \in N(i)} w'_{i, j}},
\end{equation}

where \( \mathbf{1} \) is the indicator function, which yields the value one every time the predicate is true, otherwise it is zero.
In the next step this weighted fraction of prior active neighbors must be mapped into a peer influence probability in the range \( [0, q] \) using a monotonically increasing function.
One possibility would be the usage of a linear function \( g(\alpha) = q \alpha \) defined for values \(0 \leq \alpha \leq 1 \).
However, this function seems to not describe the peer influence mechanism very well.
One one hand should another active neighbor not be heavily influential when there is already a large portion of neighbors active.
The peer influence should saturate after some fraction of neighbors is active.
On the other hand, should the peer influence for a node become noticeable after some threshold of active neighbors is reached.
This can be archived using an sigmoid function.
Similar to~\cite{Walk2016} the following algebraic sigmoid  function \( g \) is used to determine the peer influence for the node \( v_{i} \)

\begin{equation}
    p_{i}(t) = g(\alpha_{i}(t)) = \frac{\alpha_{i}(t) q}{\sqrt{\alpha_{i}^{2}(t) + \theta^2}},
\end{equation}

where the parameter \( q \) is the maximum peer influence, as discussed prior, and \( \theta > 0 \) denotes a critical threshold, which determines the required (weighted) fraction of active neighbors to set the peer influence probability close to its maximum.
Therefore, active neighbors always influence the peer influence probability, but only after a certain point in a significant way.
The satisfaction of the prior described requirements for the function can be verified by studing its first derivative \( \frac{\mathrm{d} g}{\mathrm{d} \alpha} = \frac{q \theta^{2}}{(\theta^2 + \alpha^2)^{\sfrac{3}{2}}}  \), which approaches zero very fast for values grater than \( q \).
Sigmoid functions are usually defined for all real values since there are often used to re-scale values to the range of \([-1, 1] \) or \( [0, 1] \).
However, for this work only input values on the unit interval are relevant and the two parameter \( q \) and \( \theta \) should be selected carefully to archive the desired peer influence behavior.
For instance, too large values for \( \theta \) may obstruct the peer influence mechanism crtically, since the maximum influence may never be reached, even for \( \alpha = 1 \).
\autoref{fig:peer-influence-sigmoid} shows the discussed sigmoid function with different values for the critical threshold.


\myfig{peer-influence-sigmoid}
      {width=0.75\textwidth}
      {Depiction of the sigmoid function that is used to calculate the peer influence probability \(p_{i} \) of a node based on its (weighted) fraction of active neighbors \( \alpha \) for different values of the critical threshold \( \theta \). The maximum possible peer influence probability in this example is fixed to \( q = 0.10 \). The critical threshold determines how fast the maximum peer probability can be reached. Values in the range between 5\% and 20\% seem to be a sound choice, since already a small number of influential neighbors should suffice to have an notable effect on a user.}
      {Peer influence sigmoid function examples}
      {fig:peer-influence-sigmoid}


The process of a node becoming active by himself must be adapted as well.
The simple biased coin flip becomes a more sophisticated two-step random experiment.
First, like in the base model, a biased coin is tossed to determine if a node becomes active with probability that corresponds to its activity potential.
If this first random experiment fails the node gets a second chance to become active by himself.
This is done by calculating its peer influence probability and tossing another biased coin.
Therefore, the total probability of a node \( v_{i} \) becoming active can be expressed by

\begin{equation}
    \prob{v_i \text{ becomes active}} = a_{i} + (1 - a_{i}) p_{i}.
\end{equation}



\begin{table}[]
\centering
\begin{tabular}{lp{10cm}}
\hline
\textbf{parameter} & \textbf{description} \\ \hline \hline
\multicolumn{2}{c}{activity-driven framework} \\ \hline
\( n \) & The number of nodes in the network. \\
\( f(x) \) & The probability distribution for the activity potentials of the nodes. In this work a power-law distribution is selected \( f(x) \sim x^{-2.7} \). It has a positive probability for values in the range of \( [\varepsilon, 1] \). \\
\( \varepsilon \) &  The lower bound of the activity potential. It should be \( 0 < \varepsilon \ll 1 \). \\
\( \Delta t \) &  The length of the time window in each iteration. \\
\( \eta \) &  A re-scaling factor for the activity potential to adjust the average number of active nodes in each iteration. \\
\( m \) & The number of contacts a node initiates, once it become active. \\ \hline \hline

\multicolumn{2}{c}{community structure extension} \\ \hline
\( p_{\Delta} \) & The probability to form a triangle when establishing a new tie. \\
\( p_{d} \) & The probability for node to get deleted in a iteration. \\
\( \delta \) & The constant factor that is added to the tie strength when it is reinforced. \\
\( c \) &  The memory constant, which influences the probability for a new tie.\\ \hline \hline

\multicolumn{2}{c}{peer influence extension} \\ \hline
\( \beta \) & Inverse temperature parameter for the softmax re-scaling of the weights. This parameter may also be time depended. \\
\( q \) & The maximum possible peer influence probability that a node can receive. \\
\( \theta \) & The critical threshold for the peer influence.  \\ \hline
\end{tabular}

\caption{An overview of the parameter set of the model.}
\label{tbl:all-model-parameter}
\end{table}

\chapter{Results}
\label{cha:results}


This chapter contains the results for the analysis of networks that were generated using the proposed peer influence model.
All results are obtained from synthetic networks that were generated over \( T = 75,000 \) iterations.
The size of the networks was fixed to \(n = 5,000 \) nodes and, since the model heavily depends on events that happen at random, the reported properties of the time-varying networks are obtained by averaging the results of 40 independent runs.
The model parameter that are responsible for the formation of the community structures are set to \( p_{\Delta} = 0.90 \) for the triadic closure probability, \( \delta = 1 \) for the link reinforcement constant, and \( p_{d} = \num{5e-05} \) for the node deletion probability for every experiment.
Furthermore, the critical peer influence threshold was fixed to \( \theta = 0.10 \).
This reflects the idea that only a relatively small number of active neighbors is sufficient to increase the activity of a node significantly.

The topological properties of the integrated network, which are discussed in \cref{sec:integrated-network-properties}, are measured only for nodes that are part of the temporal network.
This means that nodes that were removed earlier due to the node deletion process do not influence the properties of the integrated network any more.
\Cref{sec:network-activity} contains an overview of the overall network activity with respect to different levels of peer influence.
The effect of the peer influence mechanism on the inter-event time distribution in the network is examined in \cref{sec:inter-event-time-dists}.
All this experiments are performed for different values for the maximum peer influence probability \( q \).
However, the last section (\cref{sec:softmax-rescaling}) keeps the peer influence level constant and discusses how different values for \( \beta \), the inverse temperature for the softmax weight re-scaling, change the peer influence effects in the network.


%% ========================================================================
%% ========================================================================


\section{Integrated Network Properties}
\label{sec:integrated-network-properties}


%% ========================================================================
%% ========================================================================


\section{Network Activity}
\label{sec:network-activity}


%% ========================================================================
%% ========================================================================


\section{Inter-event Time Distributions}
\label{sec:inter-event-time-dists}
% definition burstiness and parameter B is invariant wrt to activity homogeneity in \cite{Goh2008}
% nice explanation for B in \cite{Masuda2016}


%% ========================================================================
%% ========================================================================


\section{Softmax Weight Re-scaling}
\label{sec:softmax-rescaling}

\chapter{Conclusion and Future Work}
\label{cha:conclusion}

% make critical peer influence threshold dependend on the egocentric network size and strucutre?


\appendix                       %% closes main document, appendix follows until end; only available in book-classes
\addpart*{Appendix}             %% adding Appendix to tableofcontents

\chapter{Power-law Probability Distribution}


This thesis often refers to power-law probability distributions, especially in the context of activity-driven models.
However, for the sake of simplicity the exact formula of the distribution is never explicitly stated.
This appendix contains the derivation for the exact density, distribution function, and the mean value of the probability distribution.
Additionally, it contains an example and a short description of the inverse transform sampling method used in this thesis.
However, this requires a more formal definition of the distribution, so let \(X\) be a continuous random variable with the probability density function \(f(x) \sim x^{-\gamma}\), taking values in the range \(X \in [\epsilon, 1]\), and with \(\gamma > 0\) and \(0 < \varepsilon \leq 1\).


%% ========================================================================
%% ========================================================================


\section{Probability Density Function}
\label{sec:pdf}

The probability density function (PDF) of the power-law distributed random variable \(X\), which is defined above, is stated in a more detailed form in \cref{eq:pdf}.

\begin{equation}
	f(x) =
	\begin{cases}
		c x^{-\gamma} & \varepsilon \leq x \leq 1 \\
		0             & \text{otherwise.}
	\end{cases}
	\label{eq:pdf}
\end{equation}

The factor \(c\) is a normalizing constant to ensure that the function fulfills the properties of a probability density function (i.e., \(\int_{-\infty}^{\infty} f(x) \, \mathrm{d}x = 1\)).
To calculate the normalizing constant the equation must be solved for \(c\) (see~\cref{eq:normalizing-const}).

\begin{align}
	& \int_{-\infty}^{\infty} cx^{-\gamma} \, \mathrm{d}x = c \int_{\varepsilon}^{1} x^{-\gamma} \, \mathrm{d}x = c \, \frac{x^{1-\gamma}}{1-\gamma}  \bigg |_{\varepsilon}^{1} = c \, \frac{1 - \varepsilon^{1-\gamma}}{1-\gamma} = 1 \\
	& \Leftrightarrow \, c = \frac{1-\gamma}{1 - \varepsilon^{1-\gamma}}
	\label{eq:normalizing-const}
\end{align}


%% ========================================================================
%% ========================================================================


\section{Cumulative Distribution Function}
\label{sec:cdf}

The cumulative distribution function (CDF) is used to calculate the probability that a random variable with a probability distribution \(f\) takes a value less then \(x\) (see \cref{eq:cdf-def} for the definition).

\begin{equation}
	F(x) = \int_{-\infty}^{x} f(t) \, \mathrm{d}t
	\label{eq:cdf-def}
\end{equation}

The derivation of the cumulative distribution function for the power-law distribution defined above is done in \cref{eq:cdf}.
Note that this only holds for values of \(\gamma \neq 1\).

\begin{equation}
	F(x) = \int_{-\infty}^{x} ct^{-\gamma} \, \mathrm{d}t = c \, \frac{t^{1-\gamma}}{1-\gamma}  \bigg |_{\varepsilon}^{x} = \frac{c}{1-\gamma} \Big( x^{1 - \gamma} - \varepsilon^{1 - \gamma} \Big)
	\label{eq:cdf}
\end{equation}

Due to the range of possible values that can be taken by the probability density with positive probability, the CDF can be written as a piece-wise function (\cref{eq:cdf-pieces}).

\begin{equation}
	F(x) =
	\begin{cases}
		0                                                                        & x < \varepsilon        \\
		\frac{c}{1-\gamma} \Big( x^{1 - \gamma} - \varepsilon^{1 - \gamma} \Big) & \varepsilon \leq x < 1 \\
		1                                                                        & x \geq 1
	\end{cases}
	\label{eq:cdf-pieces}
\end{equation}


%% ========================================================================
%% ========================================================================


\section{Expected Value}
\label{sec:expected-value}

The expected value for the random variable, which was defined in the beginning, is derived in \cref{eq:exp-val}.


\begin{align}
	\expval{X} & = \int_{-\infty}^{\infty} x f(x) \, \mathrm{d}x = \int_{\varepsilon}^{1} x f(x) \, \mathrm{d}x = c \int_{\varepsilon}^{1} x x^{-\gamma} \, \mathrm{d}x \\
	    & = c \int_{\varepsilon}^{1} x^{-\gamma+1} \, \mathrm{d}x = c \frac{x^{2-\gamma}}{2-\gamma}  \bigg |_{\varepsilon}^{1} = \frac{c}{2-\gamma} \Big(1 - \varepsilon^{2-\gamma}\Big)
\label{eq:exp-val}
\end{align}

%% ========================================================================
%% ========================================================================


\section{Example: \(\gamma = 2.5\) and \(\varepsilon = 10^{-2}\)}
\label{sec:example}

For this example, the exponent parameter of the distribution is set to \(\gamma = 2.5\) and the lower bound is \(\varepsilon = 0.01\).
These parameters require a normalizing constant of \(c = \frac{1}{666}\).
Therefore, the density is \(f(x) = \frac{x^{-2.5}}{666}\) and has a expected value of \(0.027\).
\cref{fig:example-pdf} and \cref{fig:example-cdf} show the plots for the probability distribution function and the cumulative distribution function, respectively, for this example.

\myfig{example-pdf}
      {width=0.75\textwidth}
      {Log-log plot of the probability density function \(f(x) = \frac{x^{-2.5}}{666}\) taking values in the range \([0.01, 1]\).}
      {Example probability density function of a power-law distribution.}
      {fig:example-pdf}

\myfig{example-cdf}
      {width=0.75\textwidth}
      {Plot if the cumulative distribution function of the power-law distribution described by the PDF \(f(x) = \frac{x^{-2.5}}{666}\) taking values in the range \([0.01, 1]\).}
      {Example cumulative distribution function of a power-law distribution.}
      {fig:example-cdf}


%% ========================================================================
%% ========================================================================


\section{Inverse Transform Sampling}
\label{sec:inverse-transform-sampling}

To generate samples from the prior defined power-law distribution the inverse transform sampling method is used in this thesis.
This algorithm is based on the inversion principle~\cite{Devroye1986}.
It states that, if \(U\) is a uniform random variable on the unit interval (i.e., \(U \sim \mathcal{U}(0,1)\)), then the random variable \(Y = F^{-1}(U)\) has the probability distribution function \(F\), where \(F^{-1}\) is the inverse distribution function.
The proof for this theorem is very short (see \cref{eq:proof-inversion-principle}) and uses the fact that \(\prob{U \leq x} = x\) for a random variable \(U \sim \mathcal{U}(0,1)\).

\begin{equation}
    \prob{F^{-1}(U) \leq x} = \prob{U \leq F(x)} = F(x)
    \label{eq:proof-inversion-principle}
\end{equation}

The actual algorithm is also very short and simple.
To draw a sample from a distribution with a cumulative distribution function \(F\), simply execute the following two steps:

\begin{enumerate}
    \item draw a number \(r\) uniformly at random from the unit interval \([0, 1]\)
    \item calculate \(F^{-1}(r)\) to obtain the sample
\end{enumerate}

However, the inverse transform sampling method requires the inverse of the cumulative distribution function \(F^{-1}\).
This can, for example, be done by solving \(F(x) = y\) for \(x\).
The inverse CDF for the power-law distribution is

\begin{equation}
    F^{-1}(x) = \Big( \frac{x(1-\gamma)}{c} + \varepsilon^{1-\gamma} \Big)^{1/1-\gamma}
\end{equation}

See \cref{eq:inverese-cdf} for the derivation.
\Cref{fig:example-sampled-pdf} depicts a approximation of a power-law distribution that was generated using the inverse transfom algorithm.

\begin{align}
	\frac{c}{1-\gamma} \Big( x^{1 - \gamma} - \varepsilon^{1 - \gamma} \Big) & = y \\
    x^{1-\gamma} - \varepsilon^{1-\gamma} & = \frac{y(1-\gamma)}{c} \\
    x^{1-\gamma} & = \frac{y(1-\gamma)}{c} + \varepsilon^{1-\gamma} \\
    x & = \Big( \frac{y(1-\gamma)}{c} + \varepsilon^{1-\gamma} \Big)^{1/1-\gamma}
\label{eq:inverese-cdf}
\end{align}

\myfig{example-sampled-pdf}
      {width=0.75\textwidth}
      {The estimated probability density function of \(f(x) = \frac{x^{-2.5}}{666}\), taking values in the range \([0.01, 1]\) (i.e., the PDF from the example in \cref{sec:example}). The approximation is archived using \(n = 5000\) samples, obtained using the inverse transform sampling method.}
      {Inverse transform sampling example.}
      {fig:example-sampled-pdf}


\printbibliography              %% remove, if using BibTeX instead of biblatex
% \include{further_ressources}  %% this is a suggestion: you have to create this file on demand


%%%% end{document}
\end{document}
