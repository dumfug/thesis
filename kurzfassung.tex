\begin{otherlanguage}{ngerman}

\chapter*{Kurzfassung}
\label{cha:kurzfassung}

Das Aufkommen von zeitlich veränderlichen Netzwerken hat maßgeblich zur Vereinfachung der zeitlichen Analyse von dynamischen Systemen beigetragen.
Systeme in denen menschliche Interaktionen stattfinden sind dabei von besonderen Interesse.
Die Interaktionsmuster der Benutzer solcher Systeme können zumeist als stoßweise beschrieben werden.
Das vor kurzem vorgestellte aktivitätsgesteuerte Framework, dass auf zeitlich veränderlichen Netzwerken basiert, hat zu Bestrebungen geführt, diese Systeme möglichst realitätsnah zu modellieren.
Allerdings vernachlässigen alle bisherigen darauf basierenden Ansätze externe Einflussquellen auf die Aktivität von Benutzern und konzentrieren sich ausschließlich auf deren inhärentes Verhalten.
In dieser Arbeit stellen wir ein aktivitätsgesteuertes Netzwerk Modell vor, dass die Aktivität von anderen in Beziehung stehenden Benutzern im Netzwerk miteinbezieht.
Dies ermöglicht es, Benutzern auf die Aktivität ihrer Nachbarn im sozialen Netzwerk Einfluss zu nehmen.
Wir untersuchen die Auswirkungen dieses Mechanismus auf die topologischen und aktivitätsbezogenen Eigenschaften von künstlich erzeugten Netzwerken.
Zudem diskutieren wir die Auswirkungen auf dynamische Prozesse im Netzwerk, insbesondere auf die auftreten Interaktionsmuster von Benutzern.

\end{otherlanguage}
